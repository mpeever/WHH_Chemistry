\documentclass[11pt,addpoints]{exam}  	% use "amsart" instead of "article" for AMSLaTeX format
\usepackage{geometry}                		% See geometry.pdf to learn the layout options. There are lots.
\geometry{letterpaper}                   		% ... or a4paper or a5paper or ... 
%\geometry{landscape}                		% Activate for rotated page geometry
%\usepackage[parfill]{parskip}    		% Activate to begin paragraphs with an empty line rather than an indent
\usepackage{graphicx}				% Use pdf, png, jpg, or eps§ with pdflatex; use eps in DVI mode
								% TeX will automatically convert eps --> pdf in pdflatex		
\usepackage{amssymb}

%SetFonts

%SetFonts


\title{Metric Prefix Quiz}
\author{Mark Peever\\ \texttt{mpeever@gmail.com}}
\date{2025-09-05}							% Activate to display a given date or no date

\begin{document}
%\maketitle

%\begin{table}[h]
%\centering
%\begin{tabular}[b]{ l | c | l  | l}
%\hline
%Metric Prefix   & Abbreviation & Power of $10$ & Rational Number \\
%\hline
%\emph{tera}     & T             & $ \times 10^{12} $  & $ 1,000,000,000,000 $ \\
%\emph{giga}    & G             & $ \times 10^{9} $   & $ 1,000,000,000 $   \\
%\hline
%\emph{mega}  & M            & $ \times 10^{6} $   & $ 1,000,000 $   \\
%\emph{kilo}      & k             & $ \times 10^{3} $   & $ 1,000 $   \\
%\emph{hecta}   & H            & $ \times 10^{2} $   & $ 100 $   \\
%\emph{deca}    & D            & $ \times 10^{1} $   & $ 10 $   \\
%\emph{unit}      & ---           & $\times 10^{0} $    & $ 1 $ \\
%\emph{deci}     & d            & $ \times 10^{-1} $  & $ 0.1 $   \\
%\emph{centi}    & c            & $ \times 10^{-2} $  & $ 0.01 $   \\
%\emph{milli}      & m          & $ \times 10^{-3} $  & $ 0.001 $  \\
%\emph{micro}   & $\mu$   & $ \times 10^{-6} $  & $ 0.000001 $  \\
%\hline
%\emph{nano}    & n           & $ \times 10^{-9} $  & $ 0.000000001 $  \\
%\emph{pico}     & p           & $ \times 10^{-12} $  & $ 0.000000000001 $  \\
%\end{tabular}
%\caption{Metric Prefixes}
%\label{table:mprefixes}
%\end{table}

\makebox[\textwidth]{Name:\enspace\hrulefill}
\vspace{0.2in}

\begin{center}
\fbox{\fbox{\parbox{5.5in}{\centering
Fill in the blanks.
}}}
\end{center}
\vspace{0.1in}

\begin{questions}
\question[1] what is the abbreviation for \emph{micro}? \fillin[ $\mu$ ]
\vspace{0.2 in}

\question[1] what is the abbreviation for \emph{milli}? \fillin[ $m$ ]
\vspace{0.2 in}

%\question[1] what is the abbreviation for \emph{centi}? \fillin[ $c$ ]
%\vspace{0.2 in}

\question[1] what is the abbreviation for \emph{kilo}? \fillin[ $k$ ]
\vspace{0.2 in}

\question[1] what is the abbreviation for \emph{Mega}? \fillin[ $M$ ]
\vspace{0.2 in}

\question[1] which metric prefix means $10^{-2}$?  \fillin[centi]
\vspace{0.2 in}

%\question[1] which metric prefix means $10^{3}$?  \fillin[kilo]
%\vspace{0.2 in}

\question[1] which metric prefix means $10^{-3}$? \fillin[milli]
\vspace{0.2 in}

\question[1] which metric prefix means $10^{-6}$? \fillin[micro]
\vspace{0.2 in}

\question[1] $1 km$ = \fillin[$1000$] m
\vspace{0.2 in}

%\question[1] $1 kg$ = \fillin[$1000$] g
%\vspace{0.2 in}

\question[1] $1 \mu s$ = \fillin[$0.000001$] s
\vspace{0.2 in}

\question[1] $1 ML$ = \fillin[$1,000,000$] L
\vspace{0.2 in}

\question[1] $1 ms$ = \fillin[$0.001$] s
\vspace{0.2 in}

\question[1] $1 DN$ = \fillin[$10$] N
\end{questions}





\end{document}  