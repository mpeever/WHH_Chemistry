\documentclass[10pt, oneside]{article}   	% use "amsart" instead of "article" for AMSLaTeX format
\usepackage{geometry}                		% See geometry.pdf to learn the layout options. There are lots.
\geometry{letterpaper}                   		% ... or a4paper or a5paper or ... 
%\geometry{landscape}                		% Activate for rotated page geometry
%\usepackage[parfill]{parskip}    		% Activate to begin paragraphs with an empty line rather than an indent
\usepackage{graphicx}				% Use pdf, png, jpg, or eps§ with pdflatex; use eps in DVI mode
								% TeX will automatically convert eps --> pdf in pdflatex		
\usepackage{amssymb}

\usepackage{hyperref}


\title{Module 1, Part 2}
\author{Mark Peever\\ \texttt{mpeever@gmail.com}}
\date{August 29, 2025}

\begin{document}
\maketitle



\section*{Objectives}
\marginpar{0 minutes}
Refer to \href{https://drive.google.com/file/d/1p-i3eBQ1MXrmtNDu9kW5U6uMlN-BSEsR/view?usp=sharing}{Module 1 Notes.pdf}.\\

By the end of this class, the students should be able to\ldots
\begin{itemize}
\item use fraction multiplication to convert between consistent units of measure
\item describe the difference between \emph{weight} and \emph{mass}
\item explain how to determine the correct number of significant figures in a measurement or calculated quantity
\item describe \emph{density}
\item use the definition of \emph{density} to calculate mass, volume, and density of substances
\end{itemize}

\section*{Welcome \& Devotion}
\marginpar{5 minutes}
\begin{itemize}
\item have one student read \href{https://www.biblegateway.com/passage/?search=Psalm\%20107\&version=LSB}{Psalm 107:1--22}
\end{itemize}

\section*{Unit Conversion}
\marginpar{10--15 minutes}
\begin{itemize}
\item Review: we covered this last week on our Baseline Quiz
\item Example: convert 3 Cups to Gallons
\item Example: convert 300m to km
\item Example: convert 12mL to Litres
\end{itemize}

\section*{Mass and Weight}
\marginpar{5 minutes}
\begin{itemize}
\item Review: you almost certainly were taught this in  ``General Science"
\item \emph{mass} is how much of you there is, \emph{weight} is how hard gravity pulls you down
\item mass is measured in grams or slugs (Imperial/English)
\item weight is measured in Newtons or pounds (Imperial/English)
\item Example: your weight on the moon \emph{vs.} your weight on the earth 
\end{itemize}

\section*{Significant Figures}
\marginpar{10--15 minutes}
\begin{itemize}
\item go through rules in \href{https://drive.google.com/file/d/1p-i3eBQ1MXrmtNDu9kW5U6uMlN-BSEsR/view?usp=sharing}{Module 1 Notes.pdf}, p. 2
\item Examples:
\begin{enumerate}
\item $2.80$ has three significant figures
\item $0.0028$ has two significant figures
\item $28000$ has two significant figures
\item $5.56$ has three significant figures
\item $0.44$ has two significant figures
\end{enumerate}
\end{itemize}

\section*{Density}
\marginpar{20 minutes}
\begin{itemize}
\item \emph{density} is how tightly packed mass is
\item cover actual formula: $\rho = \frac{m}{V} $
\item side-note: \emph{density} is what controls floatation/buoyancy
\end{itemize}

\section*{Questions}
\marginpar{10 minutes}

\section*{Assignment}
\marginpar{2 minutes}
\begin{itemize}
\item \emph{pp.} 35 \#1--10 (not to turn in)
\item \emph{pp.} 36 \#1--10 due 2025-09-05
\end{itemize}



\end{document}  