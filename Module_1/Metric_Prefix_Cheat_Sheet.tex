\documentclass[11pt, oneside]{article}   	% use "amsart" instead of "article" for AMSLaTeX format
\usepackage{geometry}                		% See geometry.pdf to learn the layout options. There are lots.
\geometry{letterpaper}                   		% ... or a4paper or a5paper or ... 
%\geometry{landscape}                		% Activate for rotated page geometry
%\usepackage[parfill]{parskip}    		% Activate to begin paragraphs with an empty line rather than an indent
\usepackage{graphicx}				% Use pdf, png, jpg, or eps§ with pdflatex; use eps in DVI mode
								% TeX will automatically convert eps --> pdf in pdflatex		
\usepackage{amssymb}

%SetFonts

%SetFonts


\title{Metric Prefixes (Prefices?)}
\author{Mark Peever\\ \texttt{mpeever@gmail.com}}
\date{2025-09-01}							% Activate to display a given date or no date

\begin{document}
\maketitle



\begin{table}[h]
\centering
\begin{tabular}[b]{ l | c | l  | l}
\hline
Metric Prefix   & Abbreviation & Power of $10$ & Rational Number \\
\hline
\emph{tera}     & T             & $ \times 10^{12} $  & $ 1,000,000,000,000 $ \\
\emph{giga}    & G             & $ \times 10^{9} $   & $ 1,000,000,000 $   \\
\hline
\emph{mega}  & M            & $ \times 10^{6} $   & $ 1,000,000 $   \\
\emph{kilo}      & k             & $ \times 10^{3} $   & $ 1,000 $   \\
\emph{hecta}   & H            & $ \times 10^{2} $   & $ 100 $   \\
\emph{deca}    & D            & $ \times 10^{1} $   & $ 10 $   \\
\emph{unit}      & ---           & $\times 10^{0} $    & $ 1 $ \\
\emph{deci}     & d            & $ \times 10^{-1} $  & $ 0.1 $   \\
\emph{centi}    & c            & $ \times 10^{-2} $  & $ 0.01 $   \\
\emph{milli}      & m          & $ \times 10^{-3} $  & $ 0.001 $  \\
\emph{micro}   & $\mu$   & $ \times 10^{-6} $  & $ 0.000001 $  \\
\hline
\emph{nano}    & n           & $ \times 10^{-9} $  & $ 0.000000001 $  \\
\emph{pico}     & p           & $ \times 10^{-12} $  & $ 0.000000000001 $  \\
\end{tabular}
\caption{Metric Prefixes}
\label{table:mprefixes}
\end{table}

This table of metric system prefixes expands on Table 1.2, p. 7 of our text.
Please note the table in our text only goes from $\mu$ to $M$ ($10^{-6}$ -- $10^{6}$).

\end{document}  