\documentclass[11pt, oneside]{article}   	% use "amsart" instead of "article" for AMSLaTeX format
\usepackage{geometry}                		% See geometry.pdf to learn the layout options. There are lots.
\geometry{letterpaper}                   		% ... or a4paper or a5paper or ... 
%\geometry{landscape}                		% Activate for rotated page geometry
\usepackage[parfill]{parskip}    		% Activate to begin paragraphs with an empty line rather than an indent
\usepackage{graphicx}				% Use pdf, png, jpg, or eps§ with pdflatex; use eps in DVI mode
								% TeX will automatically convert eps --> pdf in pdflatex		
\usepackage{amssymb}
\usepackage{cite}

% numbered examples
\usepackage{gb4e}
\usepackage{enumitem}
\usepackage{cancel}
\usepackage{amsmath}

\title{Module 1: Measurement and Units}
\author{Mark Peever \texttt{mpeever@gmail.com}}


\begin{document}
\maketitle
%\section{}
%\subsection{}

\begin{center}
Psalm 107:23--32
\end{center}

\section{Overview}

\begin{enumerate}
\item \textbf{Matter} is anything that has \emph{mass} and takes up space
\item \textbf{Units of Measurement} make numbers meaningful
\item \textbf{The Metric System} is designed to make units measurement consistent and simple
\item \textbf{Unit Conversion} is based on multiplying fractions
\item \textbf{Significant Figures} are a convention for maintaining precision in measurements
\item \textbf{Scientific Notation} allows us to represent numbers ``naturally"   
\end{enumerate}



\section{The Metric System}
The metric system is an attempt to make measurements more standardized and easier to use in mathematical equations than our ``English" or ``Imperial" system.

\begin{itemize}
\item each physical quantity has a \emph{base unit} for that quantity \cite[p. 5]{wile-chem-2} (see Table \ref{table:mbaseunits})
\item each base unit can be scaled up or down through \emph{metric prefixes} that represent powers of $10$ \cite[p. 7]{wile-chem-2} (see Table \ref{table:mprefixes})
\item note: don't make the mistake of thinking that \emph{mass} and \emph{weight} are the same thing
\end{itemize}

\begin{table}
\centering
\begin{tabular}[b]{l | l| l}
\hline
Metric Base Unit & Physical Quantity & Approximate Size \\
\hline
meter     & length    & just over 1 yard (39.37 inches) \\
gram      & mass     & one penny is about 2 grams \\
liter         & volume & just over 1 quart \\
second   & time      & about 1 second \\
\end{tabular}
\caption{Common metric base units}
\label{table:mbaseunits}
\end{table}

\begin{table}
\centering
\begin{tabular}[b]{l|l}
\hline
Metric Prefix & Power of $10$ \\
\hline
\emph{giga} (G)           & $ 1,000,000,000 $   \\
\emph{mega} (M)         & $ 1,000,000 $   \\
\emph{kilo} (k)              & $ 1,000 $   \\
\emph{hecta} (H)          & $ 100 $   \\
\emph{deca} (D)           & $ 10 $   \\
\emph{deci} (d)             & $ 0.1 $   \\
\emph{centi} (c)            & $ 0.01 $   \\
\emph{milli}  (m)           & $ 0.001 $  \\
\emph{micro} ($\mu$)  & $ 0.000001 $  \\
\emph{pico}  (p)           & $ 0.000000001 $  \\
\end{tabular}
\caption{Most common metric prefixes}
\label{table:mprefixes}
\end{table}


\section{Significant Figures}
\begin{itemize}
\item we need a way to communicate uncertainty in our measurements
\item any number we record contains its own precision claims
\item when we record any measurements, we need to be careful of our significant figures
\end{itemize}

\subsection{Rules for Significant Figures}
Per our textbook\cite[p. 21]{wile-chem-2}, a digit is a significant figure if:
\begin{enumerate}
\item it is non-zero, or
\item it is a zero between two significant figures, or
\item it is a zero to the right of a decimal point
\end{enumerate}

\subsection{Examples}
\begin{enumerate}[label=Example \arabic*]
\item $2.80$ has three significant figures
\item $0.0028$ has two significant figures
\item $28000$ has two significant figures
\item $5.56$ has three significant figures
\item $0.44$ has two significant figures
\end{enumerate}


\subsection{Mathing with Significant Figures}
\begin{itemize}
\item when adding or subtracting, round your answer to the same precision as your least precise measurement (\cite[p. 25]{wile-chem-2})
\item when multiplying or dividing, round your answer to the same number of significant figures as your least precise measurement (\cite[p. 26]{wile-chem-2})
\end{itemize}

\subsection{Examples}
\begin{enumerate}[label=Example \arabic*]
\item What is $5.56 mm + 9 mm$?
\begin{equation} 
\boxed{
\begin{split}
    5.56 mm + 9 mm &= 14.56 mm \\
                                &= 15 mm \\
 \end{split}
 }
 \end{equation}
 
\item  What is $2.80 cm + 3.6678 cm$?
\begin{equation} 
\boxed{
\begin{split}
    2.80 cm + 3.6678 cm &= 6.4678 cm \\
                           &= 6.47 cm \\
 \end{split}
 }
\end{equation}

\item What is $17.000 ft \times 3.001 lb$?
\begin{equation} 
\boxed{
\begin{split}
    17.000 ft \times 3.001 lb &= 51.017 ft \cdot lb \\
                                           &= 51.02 ft \cdot lb  \\
 \end{split}
 }
 \end{equation}

\item What is $ 5.243 mol \div 17.32543 L$?
\begin{equation} 
\boxed{
\begin{split}
        5.243 mol \div 17.32543 L &=  \frac{5.243 mol}{17.32543 L} \\
                                                  &= 0.30261875 \frac{mol}{L} \\
                                                  &= 0.3026 \frac{mol}{L} 
 \end{split}
 }
 \end{equation}

\end{enumerate}





\section{Scientific Notation}
\begin{itemize}
\item we can represent numbers as a number $\{ x | 1 \leq x < 10 \}$ multiplied by a power of $10$
\item this is particularly helpful with very large or very small numbers.
\item the rules for scientific notation are:
\begin{enumerate}
\item the first number is between $ 1 $ and $ 10 $
\item  the power of $ 10 $ is the number of places you move the decimal to the \emph{left} to get to $1$ (\emph{e.g.} $1000$ is $10^{3}$)
\item if you move the decimal to the \emph{right}, then the power of $ 10 $ is negative (\emph{e.g.} $0.01$ is $10^{-2}$)
\end{enumerate}
\item notice that it's very easy to track significant figures with scientific notation!
\item notice that the metric system is just scientific notation with pretentious names! (see Table \ref{table:pretentiousprefixes})
 \end{itemize}

\begin{table}
\centering
\begin{tabular}[c]{l|c}
\hline
Pretentious Name & Scientific Notation Equivalent \\
\hline
\emph{giga} (G)  & $ \times 10^{9} $   \\
\emph{mega} (M)  & $ \times 10^{6} $   \\
\emph{kilo} (k)      & $ \times 10^{3} $   \\
\emph{hecta} (H)      & $ \times 10^{2} $   \\
\emph{deca} (D)      & $ \times 10^{1} $   \\
\emph{deci} (d)      & $ \times 10^{-1} $   \\
\emph{centi} (c)     & $ \times 10^{-2} $   \\
\emph{milli}  (m)    & $ \times 10^{-3} $  \\
\emph{micro} ($\mu$)  & $ \times 10^{-6} $  \\
\emph{pico}  (p) & $ \times 10^{-9} $  \\
\end{tabular}
\caption{Scientific Notation and Metric Prefixes}
\label{table:pretentiousprefixes}
\end{table}

\subsection{Examples}
\begin{enumerate}[label=Example \arabic*]
\item we can write $1000$ as $1 \times 10^{3}$
\item we can write $256$ as $2.56 \times 10^{2}$
\item we can write $0.000002341$ as $ 2.341 \times 10^{-6} $
\end{enumerate}



\section{Unit Conversion}
\begin{itemize}
\item we use the idea of faction multiplication to convert measurements between units. 
\item \textbf{we can multiply any number by $1$ without changing it!}
\end{itemize}

\subsection{Examples}
 \begin{enumerate}[label=Example \arabic*]
 \item
How many yards in a mile?

We begin with what we know:
\begin{itemize}
\item $ 1 mile = 5280 ft $
\item $ 1 yd  = 3 ft $
 \end{itemize}

\begin{equation} \label{Yards per Mile}
\boxed{
\begin{split}
    1 mile  & = (\frac{1 mile}{1}) (\frac{5280 ft}{1 mile}) ( \frac{1 yd}{3 ft})  \\ \\
                & = \frac{(1 \xcancel{mile}) (5280 \xcancel{ft}) (1 yd)} {(1 \xcancel{mile}) (3 \xcancel{ft})}  \\ \\
                & = \frac{5280 yd}{3} \\ \\
                & = \frac{5280}{3} yd \\ \\
                & = 1760 yd 
 \end{split}
 }
 \end{equation}

\item
How many cups are in 5 liters?

We begin with what we know:
\begin{itemize}
\item $1 quart = 2 pints $
\item $ 1 pint = 2 cups $
\item $ 1 quart = 0.946353 L $ 
 \end{itemize}

\begin{equation} \label{Cups per 5 L}
\boxed{
\begin{split}
         5 L  & = ( \frac{5 L}{1} ) ( \frac{ 1 qt }{ 0.946353 L} ) ( \frac{ 2 pint} {1 qt } ) ( \frac{2 cup} {1 pint} )   \\ \\
                & = ( \frac{5 \xcancel{L}}{1} ) ( \frac{ 1 \xcancel{qt} }{ 0.946353 \xcancel{L} } ) ( \frac{ 2 \xcancel{pint}} {1 \xcancel{qt} } ) ( \frac{2 cup} {1 \xcancel{pint}} )  \\ \\
                & = \frac{5 \cdot 2 \cdot 2 cup}{0.946353} \\ \\
                & = \frac{5 \cdot 4 \cdot 2 }{0.946353} cup \\ \\
                & = \frac{20}{0.946353} cup \\ \\
                & = 21.1338 cup
 \end{split}
 }
 \end{equation}

\end{enumerate}




\bibliography{../Chemistry}{}
\bibliographystyle{apalike}
\end{document}  