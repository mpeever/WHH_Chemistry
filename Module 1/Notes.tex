\documentclass[11pt, oneside]{article}   	% use "amsart" instead of "article" for AMSLaTeX format
\usepackage{geometry}                		% See geometry.pdf to learn the layout options. There are lots.
\geometry{letterpaper}                   		% ... or a4paper or a5paper or ... 
%\geometry{landscape}                		% Activate for rotated page geometry
\usepackage[parfill]{parskip}    		% Activate to begin paragraphs with an empty line rather than an indent
\usepackage{graphicx}				% Use pdf, png, jpg, or eps§ with pdflatex; use eps in DVI mode
								% TeX will automatically convert eps --> pdf in pdflatex		
\usepackage{amssymb}

% numbered examples
\usepackage{gb4e}
\usepackage{enumitem}
\usepackage{cancel}
\usepackage{amsmath}

\title{Module 1: Measurement and Units}
\author{Mark Peever \texttt{mpeever@gmail.com}}


\begin{document}
\maketitle
%\section{}
%\subsection{}

\begin{center}
Psalm 107:23--32
\end{center}

\section{Overview}

\begin{enumerate}
\item \textbf{Matter} is anything that has \emph{mass} and takes up space
\item \textbf{Units of Measurement} make numbers meaningful
\item \textbf{S. I.} (or \textbf{the metric system}) is designed to make units measurement consistent and simple
\item \textbf{Unit Conversion} is based on multiplying fractions
\item \textbf{Significant Figures} are a convention for maintaining precision in measurements
\item \textbf{Scientific Notation} allows us to represent numbers ``naturally"   
\end{enumerate}

\section{Scientific Notation}
\begin{itemize}
\item we can represent numbers as multiples of factors of 10.
\item this is particularly helpful with very large or very small numbers.
\item the rules for scientific notation are:
\begin{enumerate}
\item the first number is between $ 1 $ and $ 10 $
\item  the power of $ 10 $ is the number of places you move the decimal to the \emph{left}
\item if you move the decimal to the \emph{right}, then the power of $ 10 $ is negative
\end{enumerate}
\item notice that the metric system is just scientific notation with pretentious names!
 \end{itemize}

\begin{center}
\begin{tabular}[hbpt]{|l|c|}
\hline
Pretentious S. I. Name & Scientific Notation Equivalent \\
\hline
\emph{mega}  &  $ \times 10^{6} $   \\
\emph{kilo}      &  $ \times 10^{3} $   \\
\emph{milli}     &  $ \times 10^{-3} $  \\
\emph{micro}  &  $ \times 10^{-6} $.  \\
\hline
\end{tabular}
\end{center}

\subsection{Examples}
\begin{enumerate}[label=Example \arabic*]
\item we can write $1000$ as $1 \times 10^{3}$
\item we can write $256$ as $2.56 \times 10^{2}$
\item we can write $0.000002341$ as $ 2.341 \times 10^{-6} $
\end{enumerate}


\section{Unit Conversion}
\begin{itemize}
\item we use the idea of faction multiplication to convert measurements between units. 
\item \textbf{Remember: we can multiply any number by $1$ without changing it!}
\end{itemize}

\subsection{Examples}
 \begin{enumerate}[label=Example \arabic*]
 \item
How many yards in a mile?

We begin with what we know:
\begin{itemize}
\item $ 1 mile = 5280 ft $
\item $ 1 yd  = 3 ft $
 \end{itemize}

\begin{equation} \label{Yards per Mile}
\boxed{
\begin{split}
    1 mile  & = (\frac{1 mile}{1}) (\frac{5280 ft}{1 mile}) ( \frac{1 yd}{3 ft})  \\ \\
                & = \frac{(1 \xcancel{mile}) (5280 \xcancel{ft}) (1 yd)} {(1 \xcancel{mile}) (3 \xcancel{ft})}  \\ \\
                & = \frac{5280 yd}{3} \\ \\
                & = \frac{5280}{3} yd \\ \\
                & = 1760 yd 
 \end{split}
 }
 \end{equation}

\item
How many cups are in 5 liters?

We begin with what we know:
\begin{itemize}
\item $1 quart = 2 pints $
\item $ 1 pint = 2 cups $
\item $ 1 quart = 0.946353 L $ 
 \end{itemize}

\begin{equation} \label{Cups per 5 L}
\boxed{
\begin{split}
         5 L  & = ( \frac{5 L}{1} ) ( \frac{ 1 qt }{ 0.946353 L} ) ( \frac{ 2 pint} {1 qt } ) ( \frac{2 cup} {1 pint} )   \\ \\
                & = ( \frac{5 \xcancel{L}}{1} ) ( \frac{ 1 \xcancel{qt} }{ 0.946353 \xcancel{L} } ) ( \frac{ 2 \xcancel{pint}} {1 \xcancel{qt} } ) ( \frac{2 cup} {1 \xcancel{pint}} )  \\ \\
                & = \frac{5 \cdot 2 \cdot 2 cup}{0.946353} \\ \\
                & = \frac{5 \cdot 4 \cdot 2 }{0.946353} cup \\ \\
                & = \frac{20}{0.946353} cup \\ \\
                & = 21.1338 cup
 \end{split}
 }
 \end{equation}

\end{enumerate}





\end{document}  