\documentclass[11pt, oneside]{article}   	% use "amsart" instead of "article" for AMSLaTeX format
\usepackage{geometry}                		% See geometry.pdf to learn the layout options. There are lots.
\geometry{letterpaper}                   		% ... or a4paper or a5paper or ... 

\usepackage[parfill]{parskip}    		% Activate to begin paragraphs with an empty line rather than an indent
\usepackage{graphicx}				% Use pdf, png, jpg, or eps§ with pdflatex; use eps in DVI mode
								% TeX will automatically convert eps --> pdf in pdflatex		
\usepackage{amssymb}
\usepackage{cite}

% numbered examples
\usepackage{gb4e}
\usepackage{enumitem}
\usepackage{cancel}
\usepackage{amsmath}

\usepackage{mhchem}
\usepackage{lewis}
%\usepackage{urlbst}
\usepackage{url}
\usepackage{hyperref}


% Scientific Laws
\newtheorem{definition}{Definition}
\newtheorem{law}{Law}
\newtheorem{theory}{Theory}
\newtheorem{hint}{Hint}

\title{Module 10: Acids and Bases }
\author{Mark Peever\\ \texttt{mpeever@gmail.com}}
\date{February 6 -- 13, 2025}

\begin{document}
\maketitle

\begin{center}

\end{center}

\section{Overview}
\begin{enumerate}
\item \textbf{Electrolytes} are compounds that produce ions when dissolved in water
\item \textbf{Acids} are electrolytes that are proton donors
\item \textbf{Bases} are electrolytes that are proton acceptors
\item \textbf{Salts} are electrolytes that are ionic compounds 
\end{enumerate}

\section{Acids and Bases}
\begin{definition}[Electrolyte]\label{definition:electrolyte}
Electrolytes are substances that give ions in aqueous solution.
\end{definition}

\begin{definition}[Indicator]\label{definition:indicator}
A substance that turns one color in the presence of acids and another color in the presence of bases.
\end{definition}

\begin{itemize}
\item ``electrolytes are substances that give ions in aqueous solution" (see \href{https://chem.libretexts.org/Bookshelves/Inorganic_Chemistry/Supplemental_Modules_and_Websites_(Inorganic_Chemistry)/Chemical_Reactions/Chemical_Reactions_Examples/Electrolytes}{``Electrolytes"} \cite{chem:libretexts:electrolytes})
\item acids, bases, and salts are all electrolytes (see Definition \ref{definition:electrolyte} , p. \pageref{definition:electrolyte}) 
\item acids share these common properties:
\begin{enumerate}
\item acids taste sour 
\footnote{\emph{Don't} taste a substance to see what it is. Don't do that. Just don't.}
\item acids are covalent molecules that conduct electricity when dissolved in water
\footnote{\emph{I.e.} they are electrolytes.}
\item acids turn litmus paper red (see Definition \ref{definition:indicator}, p. \pageref{definition:indicator})
\end{enumerate}

\item bases share these common properties:
\begin{enumerate}
\item bases taste bitter
\footnotemark[1]
\item bases are slippery when dissolved in water
\item bases turn litmus paper blue (see Definition \ref{definition:indicator}, p. \pageref{definition:indicator})
\end{enumerate}
\end{itemize}

\section{The Chemical Definitions of Acids and Bases}
\begin{definition}[Acid]\label{definition:acid}
A molecule that donates \ce{H^{+}} ions.
\end{definition}

\begin{definition}[Base]\label{definition:base}
A molecule that accepts \ce{H^{+}} ions.
\end{definition}

\begin{definition}[Amphiprotic Compounds]\label{definition:amphiprotic}
Compounds that can act as acid or base, depending on the situation.
\end{definition}

\begin{itemize}
\item acids are \emph{proton donors}
\footnote{Remember that \ce{H} is just one proton and one electron, so \ce{H^{+}} is basically a proton. I \emph{told} you \ce{H} is barely an atom!}
\item bases are \emph{proton acceptors}
\item when acids and bases are dissolved in water, there is a sneaky sort of reaction that takes place
\item \emph{e.g.} \ce{HCl (aq) + H2O (l) -> H3O^{+} (aq) + Cl^{-} (aq)}
\item \emph{e.g.} \ce{H2O (l) + NH3 (aq) -> NH4^{+} (aq) + OH^{-} (aq)}

\item it's important to note acids and bases only do their thing in aqueous solution: if you had a bottle of pure \ce{HCl}, the \ce{H^{+}} magic wouldn't be happening
\item this is a result of the electronegativity of \ce{Cl} compared to the electronegativity of \ce{O}
\item since the \ce{H} atom in \ce{HCl} doesn't get a very big share of the electrons in the covalent bond, it can join with a \ce{H2O} molecule to form \ce{H3O^{+}}
\item notice we have this new polyatomic ion: \ce{H3O^{+}}, called ``hydronium"
\end{itemize}

\begin{hint}\label{hint:determining-acids-and-bases}
Whenever a reaction involves a \ce{H^{+}} ion moving, then we can think in terms of acid and base.
\begin{itemize}
\item \ce{HCl (aq) + H2O (l) -> H3O^{+} (aq) + Cl^{-} (aq)}, here \ce{HCl} acts as an acid and \ce{H2O} acts as a base
\item \ce{H2O (l) + NH3 (aq) -> NH4^{+} (aq) + OH^{-} (aq)}, here \ce{NH3} acts as a base and \ce{H2O} acts as an acid
\end{itemize}
\end{hint}	

\section{Ionic Compounds in Ionic Solution}
\begin{itemize}
\item when ionic compounds dissolve, the split into their ions in aqueous solution
\item \emph{e.g.} \ce{NaCl + H2O (l) -> Na^{+} (aq) + Cl^{-} (aq) + H2O (l)}
\item this isn't truly a chemical reaction\footnote{Remember that dissolution is a \emph{physical} change.}, but it might be an easy way to think about it
\item so ionic compounds are also electrolytes (see Definition \ref{definition:electrolyte} , p. \pageref{definition:electrolyte}), we often refer to them as ``salts"
\end{itemize}


\section{Recognizing Acids and Bases from their Formulae}

\begin{table}[p]
\centering
\begin{tabular}[p]{| l | l |}
\textbf{Acid Name} & \textbf{Formula} \\
\hline
hydrochloric acid  & \ce{HCl} \\
hydrobromic acid  & \ce{HBr} \\
hydrofluoric acid   & \ce{HF} \\
nitric acid              & \ce{HNO3} \\
sulfuric acid          & \ce{H2SO4} \\
phosphoric acid    & \ce{H3PO4} \\
carbonic acid        & \ce{H2CO3} \\
acetic acid.           & \ce{C2H4O2} \\
\end{tabular}
\caption{Common Acids}
\label{table:common-acids}
\end{table}

\begin{itemize}
\item acids are covalent compounds that generally start with an \ce{H} 
\item most bases have an \ce{OH^{-}} in them
\item many acids (and some bases) have common names, see Table \ref{table:common-acids}, p. \pageref{table:common-acids} gives several common acids, reproduced from your text (see \cite[Table 10.1 (p. 328)]{wile-chem-2}) 
\end{itemize}

\section{Predicting Acid-Base Reactions}

\begin{definition}[Polyprotic Acid]\label{definition:polyprotic-acid}
An acid that can donate more than one \ce{H^{+}} ion.
\end{definition}

\begin{hint}\label{hint:acid-and-base-reaction}
In general, \ce{acid + base -> salt + water}
\end{hint}	

\begin{itemize}
\item in general, when an acid reacts with a base, we'll get an ionic compound and water as a result (see Hint \ref{hint:acid-and-base-reaction}, p. \pageref{hint:acid-and-base-reaction}) 
\item let's think about a super simple case: \ce{HCl + NaOH -> NaCl (aq) + H2O (l)}
\begin{enumerate}
\item in aqueous solution:\\ \ce{HCl + H2O (l) -> H3O^{+} (aq) + Cl^{-} (aq) }
\item in aqueous solution:\\  \ce{NaOH (aq) -> Na^{+} (aq) + OH^{-} (aq) }  
\item so together we have:\\ \ce{HCl + NaOH + H2O (l) -> H3O^{+} (aq)  + Cl^{-} (aq)  Na^{+} (aq) + OH^{-} (aq) }  
\item which then becomes:\\  \ce{HCl + NaOH + H2O (l) -> H3O^{+} (aq) + OH^{-} (aq)  + Cl^{-} (aq)  Na^{+} (aq) }  
\item which then becomes:\\  \ce{HCl + NaOH + H2O (l) -> H3O^{+} (aq) + OH^{-} (aq)  + Cl^{-} (aq)  Na^{+} (aq) -> H2O (l) + Cl^{-} (aq)  Na^{+} (aq)}  
\item which then becomes:\\  \ce{HCl + NaOH + H2O (l) -> H3O^{+} (aq) + OH^{-} (aq)  + Cl^{-} (aq)  Na^{+} (aq) -> H2O (l) + Cl^{-} (aq)  Na^{+} (aq) -> NaCl (aq) + H2O (l)} 
\footnote{Please forgive the ugliness here!}
\end{enumerate}
\item so generally speaking, \ce{acid + base -> salt + water}\footnote{Remember ``salt" means any ionic compound, not merely \ce{NaCl}.}
\item there is a caveat to this, which we'll see when we discuss Chemical Equilibrium (Module 15)
\item the salt will always be the ``leftovers" from the \ce{H3O^{+} + OH^{-} -> 2H2O} reaction
\item so we might predict something like \ce{H2SO4 (aq) + 2KOH (aq) -> K2SO4 (aq) + 2H2O (l)}, here the salt is \ce{K2SO4}
\end{itemize}

\subsection{Acids and Covalent Bases}
\begin{itemize}
\item in the above examples, we've considered acids reacting to ionic bases (\ce{NaOH}, \ce{KOH}, etc.)
\item but we can also consider an acid dissolving in water as its own reaction\footnote{In this case, it really is a chemical reaction.}
\item so when we dissolve \ce{H2SO4} in water, we get something like: \ce{H2SO4 + H2O (l) -> H3O^{+} + SO4^{2-}}
\item notice we end up with ions on the right-hand-side
\item note, too, that when we balance these equations, the charges should balance as well as the elements: \ce{H2SO4 + 2H2O (l) -> 2H3O^{+} + SO4^{2-}}
\end{itemize}

\section{Concentration and Dilution}

\begin{definition}[Concentration]\label{definition:concentration}
The concentration of a substance in solution is the amount of that substance in a given volume:
$$ [ \hspace{0.1 in} ] = \frac{n}{V} $$	
\end{definition}

\begin{definition}[Molarity]\label{definition:concentration}
\textbf{Molarity} is the number of moles of a substance per liter of solution:
$$ 1 M = \frac{1 mole}{1 liter} $$
\end{definition}

\begin{definition}[Dilution]\label{definition:dilution}
Adding water to a solution to decrease its concentration.
\end{definition}

\begin{itemize}
\item electrolytes do their thing in aqueous solution
\item we remember that all chemical reactions go ``by the mole"
\item when we mix things in aqueous solution, we want to be able to think of our chemical reactions in terms of volumes (``$2 L$ of this acid added to $1.5 L$ of that base")
\item so we have to introduce the idea of \emph{Concentration}, which we'll measure in terms of amount per volume
\item because we really want to know how many moles of a substance we have, we'll express our concentrations in \textbf{Molarity}, which is just ``moles per liter"
\item because any given solution might contain many different substances, we write the concentration of any substance as $[substance]$ (\emph{e.g.} \ce{[HCl]} is the concentration of \ce{HCl})
\item so if we dissolved $1.000 mol$ of \ce{NaCl} in $2.35 L$ of water, we could calculate the concentration of \ce{NaCl} as:
\begin{equation} 
\boxed{
\begin{split}
    [ NaCl ] &= \frac{n}{V} \\
                 &= \frac{1.000 mol}{2.35 L}\\
                 &= 0.4255139 \frac{mol}{L}\\
                 &= 0.425 M
\end{split}
 }
 \end{equation}
\item we can add water\footnote{Or whatever the solvent is: it's only water for aqueous solutions.} to an aqueous solution to decrease the concentration of the solute
\item so if we have saltwater, we can make it ``less salty" by adding water
\item of course we can't ever get it to be non-salty, at least not in the absolute sense, but we can reduce the concentration of the salt (\ce{[NaCl]}) by ``watering it down"
\item from a Conservation of Mass perspective, we can see that salt is neither created nor destroyed, but it can be made less concentrated
\item so if we start with $[NaCl] = \frac{n}{V}$, then we could say that the number of moles must be given by: $n = V \cdot [NaCl]$
\item and since diluting our solution doesn't change the number of moles of \ce{NaCl} we have, then we can say:
\begin{equation} 
\boxed{
\begin{split}
    [\hspace{0.1 in}]_{1} \cdot V_{1} &= [\hspace{0.1 in}]_{2} \cdot V_{2}               
\end{split}
 }
 \end{equation}
 \item so we're back to cross-mulitplying!!!!
\end{itemize}


\section{Concentration and Stoichiometry}

\begin{itemize}
\item if we know the concentration of a substance in a solution, then we can use the volume of a solution as a proxy for the number of moles we have\ldots
\item this really isn't any different from using the mass of a substance to find the moles
\item we can then calculate our reactants or our products stoichiometrically, just as we did before
\end{itemize}

\section{Acid/Base Titration}

\begin{definition}[Titration]\label{definition:titration}
The process of slowly reacting a base of unknown concentration with an acid of known concentration (or vice-versa) until just enough acid has been added to react with all the base.
This process determines the concentration of the unknown base or acid.
\end{definition}

\begin{itemize}
\item \emph{titration} is when we use an acid to neutralize a base in order to determine the concentration of the base
\item the idea is that if we carefully measure how much volume of acid we add we know how many moles we added (as long as we know the concentration)
\item we add an indicator (see Definition \ref{definition:indicator}, p. \pageref{definition:indicator}) so we can know when the solution is no longer basic
\item when the solution is neutral (\emph{i.e.} it's no longer basic), then we can calculate how many moles we added
\item we can use stoichiometry to know how many moles of the base were neutralized
\item then we can calculate the concentration of the base
\item and of course, we can work in the other direction too: using a base to neutralize an acid
\item we'll do a titration lab to hammer this idea home
\end{itemize}

\section{Homework}
Review Problems: p. 351 \# 1--10 (not to be turned in)\\
Practice Problems: p. 352 \# 1--10 (due 2026-02-20)\\

\nocite{wile-chem-2}
\bibliography{../Chemistry}
\bibliographystyle{apalike}

%\clearpage
\subsection{Answers to Practice Problems}
Answers to Practice Problems p. 352 \# 1--10:
\begin{enumerate}
\item \ce{3H2SO4 + 2Al(OH)3 -> Al2(SO4)3 + 6H2O}
\item \ce{2HNO3 + Ca(OH)2 -> Ca(NO3)2 + 2H2O}
\item \ce{H2CO3 + 2NH3 -> + 2NH4^{+} + CO3^{2-}} (\emph{Hint}: the charges need to balance too!)
\item \ce{HBr + H2O (l) -> H3O^{+} (aq) + Br^{-} (aq)} 
\item 
\begin{enumerate}
\item $2.9M$
\item $12.1M$
\item $0.93M$
\end{enumerate}
\item $150 mL$
\item $0.84 mL$
\item $32 g$
\item $2.4 M$
\item $0.42 M$
\end{enumerate}
\end{document}  