\documentclass[10pt, oneside]{article}   	% use "amsart" instead of "article" for AMSLaTeX format
\usepackage{geometry}                		% See geometry.pdf to learn the layout options. There are lots.
\geometry{letterpaper}                   		% ... or a4paper or a5paper or ... 
%\geometry{landscape}                		% Activate for rotated page geometry
%\usepackage[parfill]{parskip}    		% Activate to begin paragraphs with an empty line rather than an indent
\usepackage{graphicx}				% Use pdf, png, jpg, or eps§ with pdflatex; use eps in DVI mode
								% TeX will automatically convert eps --> pdf in pdflatex		
\usepackage{amssymb}
\usepackage{mhchem}
\usepackage{hyperref}

\title{Module 6, Part 2}
\author{Mark Peever\\ \texttt{mpeever@gmail.com}}
\date{November 14, 2025}

\begin{document}
\maketitle

\section*{Objectives}
\marginpar{0 minutes}
Refer to \href{https://drive.google.com/file/d/1oxm09jENQAA8VtBM0JR3grig1tUsXAT7/view?usp=sharing}{Module 6 Notes.pdf}.\\

By the end of this class, the students should be able to\ldots
\begin{itemize}
\item use stoichiometric coefficients correctly
\item describe \emph{limiting} and \emph{excess} reactants
\item describe Gay-Lussac's Law
\item correctly describe Empirical and Molecular Formulae 
\end{itemize}

\section*{Welcome \& Devotion}
\marginpar{5 minutes}
\begin{itemize}
\item have one student read \href{https://www.biblegateway.com/passage/?search=Colossians\%201\&version=LSB}{Colossians 1:9--17}
\end{itemize}

\section*{How we're going to homework}
\marginpar{15 minutes}
\begin{itemize}
\item I'll ensure we all have the correct answers to each homework problem
\item I want you to show as much work as you possibly can
\item I'll give all the feedback I can, but will likely grade in terms of pass/fail
\end{itemize}


\section*{Stoichiometric Coefficients}
\marginpar{20 minutes}
\begin{itemize}
\item the coefficients in a chemical reaction are the ratios of substances \emph{in moles}
\item burning hectane: \ce{C6H14 + O2 -> CO2 + H2O}
\begin{itemize}
\item balance on the board
\item if we have 1.0 kg of hectane, how much oxygen do we need?
\item if we have 1.0 kg of hectane, how much \ce{CO2} do we produce?
\end{itemize}
\item what is our limiting reactant?
\item what is our excess reactant?
\end{itemize}

\section*{Gay-Lussac's Law}
\marginpar{10 minutes}
\begin{itemize}
\item turns out all gases like to take up the same amount of space \emph{per mole}
\item we can use our stoichiometric coefficients to represent volumes of gas in a chemical reaction
\item preview $PV = nRT$
\end{itemize}

\section*{Empirical and Molecular Formulae}
\marginpar{20 minutes}
\begin{itemize}
\item when we come up with an experimental formula from our stoichiometry, it can be difficult or impossible to tell what a molecule actually looks like:
\item \ce{OH-} and \ce{H2O2} would look the same, but they're not the same thing
\item they both have an empirical formula of \ce{HO}, but their molecular formulae are different
\item if we know the molecular mass, we can figure out which one is which 
\end{itemize}

\section*{Questions}
\marginpar{5 minutes}

\section*{Assignment}
\begin{itemize}
\item Experiment 6.1, p. 149 (due 2025-11-21)
\item Review Problems: p. 198 \# 1--10 (not to be turned in)
\item Practice Problems: p. 199--200 \# 1--10 (due 2025-11-21)
\item Take-Home Quiz (due 2025-11-21)
\end{itemize}



\end{document}  