\documentclass[10pt, oneside]{article}   	% use "amsart" instead of "article" for AMSLaTeX format
\usepackage{geometry}                		% See geometry.pdf to learn the layout options. There are lots.
\geometry{letterpaper}                   		% ... or a4paper or a5paper or ... 
%\geometry{landscape}                		% Activate for rotated page geometry
%\usepackage[parfill]{parskip}    		% Activate to begin paragraphs with an empty line rather than an indent
\usepackage{graphicx}				% Use pdf, png, jpg, or eps§ with pdflatex; use eps in DVI mode
								% TeX will automatically convert eps --> pdf in pdflatex		
\usepackage{amssymb}
\usepackage{mhchem}
\usepackage{hyperref}

\title{Module 6, Part 1}
\author{Mark Peever\\ \texttt{mpeever@gmail.com}}
\date{November 7, 2025}

\begin{document}
\maketitle

\section*{Objectives}
\marginpar{0 minutes}
Refer to \href{https://drive.google.com/file/d/1oxm09jENQAA8VtBM0JR3grig1tUsXAT7/view?usp=sharing}{Module 6 Notes.pdf}.\\

By the end of this class, the students should be able to\ldots
\begin{itemize}
\item describe molar mass
\item use stoichiometric coefficients correctly
\item describe \emph{limiting} and \emph{excess} reactants
\end{itemize}

\section*{Welcome \& Devotion}
\marginpar{5 minutes}
\begin{itemize}
\item have one student read \href{https://www.biblegateway.com/passage/?search=Colossians\%201\&version=LSB}{Colossians 1:9--17}
\end{itemize}

\section*{Review Module 5 Quiz}
\marginpar{15 minutes}
\begin{itemize}
\item work through the problems in Module 5 Quiz
\end{itemize}

\section*{Review Module 4 Homework}
\marginpar{15 minutes}
\begin{itemize}
\item work through the problems 6, 7, 8, 9, p. 132 
\end{itemize}

\section*{Molar Mass}
\marginpar{10 minutes}
\begin{itemize}
\item this is \emph{exactly} what we covered in Module 5
\item the mass of an atom or molecule in \emph{amu} is the mass of one \emph{mole} of the substance in \emph{g}
\item work out several on the board:
\begin{itemize}
\item  molar mass of \ce{H2O}
\item  molar mass of \ce{NaOH}
\item  molar mass of \ce{H2SO4}
\item  molar mass of \ce{CH4} (methane)
\item  molar mass of \ce{C4H10} (butane)
\item  molar mass of \ce{C3H8} (propane)
\item  molar mass of \ce{C6H14} (hectane)
\item  molar mass of \ce{C8H18} (octane)
\item  molar mass of \ce{C12H22O11} (sucrose)
\end{itemize}
\end{itemize}

\section*{Stoichiometric Coefficients}
\marginpar{20 minutes}
\begin{itemize}
\item the coefficients in a chemical reaction are the ratios of substances \emph{in moles}
\item burning hectane: \ce{C6H14 + O2 -> CO2 + H2O}
\begin{itemize}
\item balance on the board
\item if we have 1.0 kg of hectane, how much oxygen do we need?
\item if we have 1.0 kg of hectane, how much \ce{CO2} do we produce?
\end{itemize}
\end{itemize}


\section*{Questions for me}
\marginpar{10 minutes}

\section*{Assignment}
\marginpar{5 minutes}
\begin{itemize}
\item nothing due next week, but start on:
\item Experiment 6.1, p. 149 (due 2025-11-21)
\item Review Problems: p. 198 \# 1--10 (not to be turned in)
\item Practice Problems: p. 199--200 \# 1--10 (due 2025-11-21)
\end{itemize}



\end{document}  