\documentclass[11pt, oneside]{article}   	% use "amsart" instead of "article" for AMSLaTeX format
\usepackage{geometry}                		% See geometry.pdf to learn the layout options. There are lots.
\geometry{letterpaper}                   		% ... or a4paper or a5paper or ... 

\usepackage[parfill]{parskip}    		% Activate to begin paragraphs with an empty line rather than an indent
\usepackage{graphicx}				% Use pdf, png, jpg, or eps§ with pdflatex; use eps in DVI mode
								% TeX will automatically convert eps --> pdf in pdflatex		
\usepackage{amssymb}
\usepackage{cite}

% numbered examples
\usepackage{gb4e}
\usepackage{enumitem}
\usepackage{cancel}
\usepackage{amsmath}

\usepackage{mhchem}

% Scientific Laws
\newtheorem{definition}{Definition}
\newtheorem{law}{Law}
\newtheorem{theory}{Theory}
\newtheorem{hint}{Hint}

\title{Module 6: Stoichiometry }
\author{Mark Peever\\ \texttt{mpeever@gmail.com}}
\date{November 7 -- 14, 2025}

\begin{document}
\maketitle

\begin{center}

\end{center}

\section{Overview}
\begin{enumerate}
\item \textbf{Molar Mass} tells us the mass of one mole of a substance
\item \textbf{Stoichiometry} is relating the quantities of various substances in a chemical reaction
\item we can control chemical reactions by \textbf{limiting reactants} 
\item \textbf{Gay-Lussac's Law} allows to to use volumes of a gas as a proxy for the count of atoms in a gaseous substance
\item \textbf{Empirical Formulae} allow us to express \emph{possible} chemical formulae in terms of whole-number ratios
\end{enumerate}

\section{Mole Relationships in Chemical Equations}
\begin{itemize}
\item sometimes a package will display the text, ``this package is measured by weight, not by volume"\ldots what does that mean?
\item the idea there is that a box might appear to be partially, or even mostly empty, but you got what you paid for, \emph{because you paid for a specific mass (weight) of the product}
\item for example, if you buy a box of cereal\footnote{Don't judge me!} and the box advertises that it contains $10 oz$ of cereal, but it only fills half the box; the box is warning you that you paid for $10 oz$, as opposed to ``a boxful''
\item in exactly the same way\footnote{I sound like Homer here.}, we need to understand the chemical reactions don't occur \emph{per gram} or \emph{per liter}\footnote{Notwithstanding Law \ref{law:gay-lussac}, p. \pageref{law:gay-lussac}, but more on that later.}, but \emph{per atom}
\item and since we group atoms into moles, we can say that chemical reactions work \emph{per mole}
\item we can still relate the masses of the different substances, but we have to remember that a mole of one and a mole of another don't necessarily have the same mass:
\end{itemize} 

 \begin{enumerate}[label=Example \arabic*]
 \item burning propane: \ce{ C3H8 + 5O2 -> 3CO2 + 4H2O }
\begin{itemize}
\item every propane molecule (\ce{C3H8}) requires \emph{five} oxygen molecules to burn
\item every propane molecule we burn produces \emph{three} carbon dioxide (\ce{CO2}) molecules
\item every propane molecule we burn produces \emph{four} water molecules
\item but notice if we were to weigh out these substances, we wouldn't find we require five times as much oxygen as propane(!!)
\begin{itemize}
\item \ce{C3H8} has a mass of $44.0956 amu$
\item  \ce{O2} has a mass of $31.9988 amu$, but there are \emph{five} of them for a total mass of $159.994 amu$
\item so we'd find there is $3.628$ times as much much \ce{O2} as \ce{C3H8} by mass
\end{itemize}	
\end{itemize}

 \item decomposing water: \ce{ 2H2O -> 2H2 + O2 }
\begin{itemize}
\item every water molecule produces \emph{one} oxygen molecules
\item every water molecule produces \emph{two} hydrogen molecules
\item but if we were to weigh them out\ldots 
\begin{itemize}
\item \ce{H2O} has a mass of $18.01528 amu$, but there are \emph{two} of them for a total mass of $36.03056 amu$
\item \ce{O2} has a mass of $31.9988 amu$, and our reaction only contains \emph{one} 
\item so we'd find there is $1.126$ times as much water as oxygen, by mass in our decomposition reaction	
\end{itemize}
\end{itemize}
\end{enumerate}

\pagebreak
%\section{Recipes vs. Reactions}
%\begin{itemize}
%\item in the USA, most recipes are written in terms of \emph{volume}: $1 Tbsp$ of this, $2 tsp$ of that, etc.
%\item sometimes you find ``more serious" recipes written in terms of \emph{weight}: $1 lb$ of this, $3 oz$ of that, etc.\footnote{Although small quantities are frequently measured by \emph{volume} regardless, because kitchen scales don't typically have sufficient sensitivity.}
%\item sometimes ``more serious" recipes\footnote{Or recipes from countries that haven't managed to land someone on the moon.} will be written in terms of \emph{mass}: $1 kg$ of this, $2 g$ of that, etc.
%\item chemical reactions are written in terms of \emph{number of atoms}: $1 mole$ of this, $3 mole$ of that, etc.
%\item if we're actually putting together a chemical ``recipe," we'll end up calculating \emph{mass} from our moles, but we still represent chemical reactions in terms of moles
%\end{itemize}



\section{Limiting Reactants and Excess Components}
\begin{itemize}
\item let's go back to our burning of propane example:  \ce{ C3H8 + 5O2 -> 3CO2 + 4H2O }
\item we might imagine if we were to fill a balloon with propane, take it outside, and touch it with a cattle prod, it would go up in flame
\item but when would our reaction stop? 
\item it would stop when we ran out of propane\ldots why?
\item if we're outside, we have a virtually endless supply of \ce{O2}, but a very limited supply of \ce{C3H8}
\item so we can safely say we're going to run out of propane before we run out of oxygen
\item the reactant we run out of first is called the \textbf{limiting reactant}: that's the one that'll stop our reaction
\item so in our balloon example, we have an \emph{excess} of \ce{O2} and we're limited by our quantity of \ce{C3H8}
\end{itemize}

%\section{Fully Analyzing Chemical Equations}


\section{Stoichiometric Coefficients}

\begin{itemize}
\item the multipliers of each substance in a chemical equation are referred to as \textbf{stoichiometric coefficients} 
\item these define ratios between the number of molecules and/or atoms of each substance in a chemical reaction
\item we generally think of the as ratios between the number of \emph{moles} of each substance in a chemical reaction\footnote{Remember, a mole is just a bunch of things, so this is exactly the same thing as the point above.}
\item remember: chemicals react by the mole\footnote{The great Greg Mitchell said this a lot, I think of it as ``Mitchell's Law," although we probably won't live to see it enshrined as it ought to be. We live in a fallen world, after all.}
\end{itemize}

\section{Volume Relationships in Chemical Reactions}

\begin{law}[Gay-Lussac's Law]\label{law:gay-lussac}
The stoichiometric coefficients in a chemical equation related the volumes of gases in the equation as well as the number of moles of substances in the equation.
\end{law}

\begin{itemize}
\item we'll come back to this when we talk about the Ideal Gas Law later in the year
\item it turns out gases tend to fill the same amount of space per mole, regardless of what sort of gas they are: \ce{O2} and \ce{C3H8}, for example, will both tend to fill the same volume per mole
\item \emph{if a chemical equation consists only of gases}, then you can work in terms of volume (\emph{i.e.} litres) instead of moles
\end{itemize}


\section{Mass Relationships in Chemical Reactions}
\begin{itemize}
\item we've already learned that chemicals react \emph{by the mole}, not by the gram\footnote{See ``Mitchell's Law" above.}
\item so when we want to examine quantities in a chemical reaction, we need to convert mass to moles (grams to moles), which we do using our molar mass
\item when we finish analyzing a chemical reaction, we'll generally have to convert moles back to grams
\item remember: \emph{chemicals react by the mole}
\end{itemize}

 \begin{enumerate}[label=Example \arabic*]
 \item How much water is produced when we burn $2.00 kg$ of propane (\ce{C3H8})? 
\begin{itemize}
\item we know a \emph{combustion reaction} starts with a substance that contains both carbon (\ce{C}) and hydrogen (\ce{H}) and produces carbon dioxide and water
\item so we know burning propane looks like: \ce{ C3H8 + 5O2 -> 3CO2 + 4H2O }
\item the molar mass of propane can be calculated from the molar masses of carbon ($12.01 g$) and hydrogen ($1.01g$)
\item so the molar mass of propane is:
\begin{equation} 
\boxed{
\begin{split}
    m_{C} = 12.01 \frac{g}{mol} \\
    m_{H} = 1.01 \frac{g}{mol} \\
    m_{mol} &= 3 \times m_{C} + 8 \times m_{H} \\
                             &= 3(12.01\frac{g}{mol}) + 8(1.01 \frac{g}{mol}) \\
                             &= 36.01 \frac{g}{mol} + 8.08 \frac{g}{mol}\\
                             &= 44.09 \frac{g}{mol}
\end{split}
 }
 \end{equation}
 
 \item we're starting our with $2.00 kg$ of propane, so we need to figure out what that is in moles:
\begin{equation} 
\boxed{
\begin{split}
    moles &= \frac{m}{m_{mole}} \\ 
               &= (\frac{2.00 kg}{44.09 \frac{g}{mol}})(\frac{1000 g}{1 kg})\\
               &= (\frac{2.00 \xcancel{kg}}{44.09 \frac{\xcancel{g}}{mol}})(\frac{1000 \xcancel{g}}{1 \xcancel{kg}})\\
               &= 45.36176 mol \\
               &= 45.36 mol
\end{split}
 }
 \end{equation}

\item every propane molecule we burn produces \emph{four} water molecules:
\begin{equation} 
\boxed{
\begin{split}
    moles_{water} &= 4 \times moles_{propane} \\
               &= 4 \times 45.36 mol \\
               &= 181.447 mol \\
               &= 181.4 mol \\
\end{split}
 }
 \end{equation}
	
\item to get from the moles of water to a mass of water, we'll need to find the molar mass of water:
\begin{equation} 
\boxed{
\begin{split}
    m_{mole} &= 2 \times m_{H} + m_{O} \\
                    &= 2 \times 1.01 \frac{g}{mol} + 16.00 \frac{g}{mol}\\
                    &= 18.02 \frac{g}{mol}
\end{split}
 }
 \end{equation}
 
 \item so we'll multiply our number of moles by the molar mass of water to get a final mass:
 \begin{equation} 
\boxed{
\begin{split}
    m &= m_{mole} \times moles \\
        &= 181.4 mol \times 18.02 \frac{g}{mol}  \\
        &= 181.4 \xcancel{mol} \times 18.02 \frac{g}{\xcancel{mol}}  \\
        &= (3269.67566342 g)(\frac{1 kg}{1000 g}) \\
        &= (3269.67566342 \xcancel{g})(\frac{1 kg}{1000 \xcancel{g}}) \\
        &= 3.26967566342 kg \\
        &= 3.270 kg \\
\end{split}
 }
 \end{equation}
 \item so $2.00 kg$ of propane will produce $3.270 kg$ of water when burned
\end{itemize}

\end{enumerate}

\section{Using Stoichiometry to Determine Chemical Formulae}
\begin{itemize}
\item we can use what we know of stoichiometry so far to come up with a chemical formula for a substance by decomposition
\item so if we decompose a substance and identify its constituent parts, we can use our stoichiometry to work out what that substance was
\end{itemize}
\begin{enumerate}[label=Example \arabic*]
 \item an unknown substance was decomposed into $2.02 g$ of hydrogen and  $32.00 g$ of oxygen, what was the substance?
\begin{itemize}
\item we know our \emph{decomposition reaction} starts with a substance that contains both oxygen (\ce{O}) and hydrogen (\ce{H}) and produces \ce{H2} and \ce{O2}\footnote{Because both \ce{O} and \ce{H} are homonuclear diatomics.}
\item so we know our decomposition looks like: \ce{ H_{x}O_{y} -> H2 + O2 }
\item we know the molar mass of \ce{H} is $1.01 g$
\item we know the molar mass of \ce{O} is $16.00 g$
\item so we know we start out with:
\begin{equation} 
\boxed{
\begin{split}
    mm_{H} &= 1.01 \frac{g}{mol} \\
    moles_{H}   &= \frac{m_{H}}{mm_{H}} \\
                        &= \frac{2.02 g}{1.01 \frac{g}{mol}} \\   
                        &= \frac{2.02 \xcancel{g}}{1.01 \frac{\xcancel{g}}{mol}} \\ 
                        &= 2.00 mol 
\end{split}
 }
 \end{equation}

\begin{equation} 
\boxed{
\begin{split}
    m_{O} = 16.00 \frac{g}{mol} \\
    moles_{O} &= \frac{m_{O}}{mm_{O}} \\
                      &= \frac{32.00 g}{16.00 \frac{g}{mol}} \\
                      &= \frac{32.00 \xcancel{g}}{16.00 \frac{\xcancel{g}}{mol}} \\
                      &= 2.00 mol \\
\end{split}
 }
 \end{equation}
\item so we have $1.00 mol$ of \ce{H} and $1.00 mol$ \ce{O}
\item so in our chemical equation, we must have $ x = y$
\item so our equation becomes \ce{H2O2 -> H2 + O2}
\end{itemize}
\end{enumerate}

\section{Empirical and Molecular Formulae}

\begin{definition}[Molecular Formula]\label{definition:molecular-formula}
A chemical formula that provides the number of each type of atom in a molecule.
\end{definition}

\begin{definition}[Empirical Formula]\label{definition:empirical-formula}
A chemical formula that tells you a simple, whole-number ratio for the atoms in a molecule.
\end{definition}

\begin{itemize}
\item in our example above, I skipped a step: I jumped to the conclusion that we began with \ce{H2O2} (hydrogen peroxide)\footnote{And let's be honest, I used a contrived example to get the result I wanted.}
\item but at least in theory, the left-hand side of our equation could have been \ce{\frac{1}{3}H6O6}, right? or even \ce{2HO}
\item this makes brings us to the concept of an \emph{empirical formula}: a formula that gives us ratios of the atoms in a molecule, but not the specific number of all the different types of atoms
\item so in this example, the \emph{empirical formula} of our mystery substance is \ce{HO}: there's a $1:1$ ratio of \ce{H} to \ce{O}, but we don't know for sure how many atoms of each are in our substance\footnote{This isn't quite as contrived as I'm making it out to be: \ce{H2O2} is hydrogen peroxide, but \ce{OH-} is hydronium. They both exist in real life, although hydronium really only exists in aqueous solution.}
\item while this seems like a dead-end for serious inquiry, it's possible to work from one to the other if we can figure out a molecule's mass for our mystery substance
\item measuring molecular mass is a bit too complicated for us right now, so we'll treat it as a given in our homework\ldots
\end{itemize}

\section{Homework}
Review Problems: p. 198 \# 1--10 (not to be turned in)\\
Practice Problems: pp. 199--200 \# 1--10 (due 2025-11-21)\\
Experiment 6.1, p. 149 (due 2025-11-21)\\

\subsection{Answers to Practice Problems}
Answers to Practice Problems pp. 199--200 \# 1--10:
\begin{enumerate}
\item $6.39 mol$
\item $0.0012 mol$ (or $1.2 \times 10^{-3} mol$)
\item $6700 mol$ (or $6.7 \times 10^3 mol$)
\item $100.0 L$
\item $17900 g$ (or $1.79 \times 10^4 g$)
\item $13.6 g$
\item $540000 g$ (or $5.4 \times 10^5 g$)
\item \ce{C2H2Br4}
\item \ce{C2H2O}
\item \ce{C2Cl4}
\end{enumerate}



\nocite{wile-chem-2}
\bibliography{../Chemistry}{}
\bibliographystyle{apalike}
\end{document}  