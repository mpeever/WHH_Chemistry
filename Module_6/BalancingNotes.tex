\documentclass[11pt, oneside]{article}   	% use "amsart" instead of "article" for AMSLaTeX format
\usepackage{geometry}                		% See geometry.pdf to learn the layout options. There are lots.
\geometry{letterpaper}                   		% ... or a4paper or a5paper or ... 

\usepackage[parfill]{parskip}    		% Activate to begin paragraphs with an empty line rather than an indent
\usepackage{graphicx}				% Use pdf, png, jpg, or eps§ with pdflatex; use eps in DVI mode
								% TeX will automatically convert eps --> pdf in pdflatex		
\usepackage{amssymb}
\usepackage{cite}

% numbered examples
\usepackage{gb4e}
\usepackage{enumitem}
\usepackage{cancel}
\usepackage{amsmath}
%\usepackage{mathtools}

\usepackage{mhchem}

% Scientific Laws
\newtheorem{definition}{Definition}
\newtheorem{law}{Law}
\newtheorem{theory}{Theory}
\newtheorem{hint}{Hint}

\title{Module 6: Stoichiometry (Reprise)\\ Fun with Balancing}
\author{Mark Peever\\ \texttt{mpeever@gmail.com}}
\date{November 15, 2025}

\begin{document}
\maketitle

\begin{center}

\end{center}

\section{Overview}
This is a short side-quest: a look at three different ways we could think about balancing Chemical Equations.

In every case we're going to consider the complete combustion of hectane, because we know it works out to be a bit ugly:
\footnote{No, I didn't realize how ugly it would be when I picked it as an example in class. I like to work examples ``cold," and they sometimes surprise me.}\\
\begin{center}
\ce{2C6H14 + 19O2 -> 12CO2 + 14H2O}
\end{center}


\section{Balancing Chemical Equations as a System of Equations}
\begin{itemize}
\item our unbalanced chemical equation looks like:\\
\begin{center}
 \ce{C6H14 + O2 -> CO2 + H2O}
 \end{center}

\item we \emph{know} this one isn't balanced: 
\begin{itemize}
\item there are 6 \ce{C} atoms on the left, and only 1 on the right
\item there are 14 \ce{H} atoms on the left, and only 2 on the right
\item there are 2 \ce{O} atoms on the left, and 3 on the right
\end{itemize}

\item we'll begin by balancing this one with some algebra:\footnote{We'll use Greek letters as variables here, in order to avoid confusion with chemical symbols, but I skipped $\delta$ because that one is generally given specific meaning in Calculus.}\\
\begin{center}
 \ce{\alpha C6H14 + \beta O2 -> \gamma CO2 + \epsilon H2O}
 \end{center}
 
\item so we're going to say that there are values of $\alpha, \beta, \gamma, \epsilon$ such that our chemical equation is balanced
\footnote{Interestingly, these values aren't unique.}
\footnote{Traditionally in linear algebra people use variables like $x_{0}, x_{1}, x_{2}, x_{3}$, etc.}

\item \emph{if that's true}, then we can come up with specific equations for each of our elements:
\begin{enumerate}%[label=Equation \arabic*]
\item there must be the same number of \ce{C} atoms on each side, so: $ 6 \alpha = \gamma$
\item there must be the same number of \ce{H} atoms on each side, so: $14 \alpha = 2 \epsilon$ 
\item there must be the same number of \ce{O} atoms on each side, so: $2 \beta = 2 \gamma + \epsilon$ 
\end{enumerate}
\item now we can assemble our equations into a system:
\begin{equation} 
\boxed{
\begin{split}
    6 \alpha &=  \gamma \\
    14 \alpha  &= 2 \epsilon \\
    2 \beta &= 2 \gamma +  \epsilon \\
\end{split}
 }
 \end{equation}

\item now we get everything onto the same side of the ``$=$" sign:\footnote{Study this step carefully until you understand where I got this.}
\begin{equation} 
\boxed{
\begin{split}
    6 \alpha   &                 & -1 \gamma  &                   &= 0 \\
    14 \alpha &                 &                    & -2 \epsilon  &= 0 \\
                    &  2 \beta    & -2 \gamma  & -1 \epsilon  &= 0 \\
\end{split}
 }
 \end{equation}
 \item note there are only three equations, \emph{even though there are four unknowns}
 \footnote{I expect this is always true, but have no interest in proving it.}
  this means there is not a \emph{unique} solution
 
 \item but we already know there's not a unique solution, there is a \emph{set of solutions}
 \footnote{We could, for example, multiply everything in a balanced equation by 17 and it would still be balanced.}
 
 \item so we're going to identify that set, and then choose the simplest solution in the set
 \footnote{This will make sense when we get there.}

\item from here there are a lot of options, so we'll address those in different sections
\end{itemize} 

\section{Solving by Elimination}
\begin{itemize}
\item one way to solve a system of linear equations is to add, subtract, and multiply equations to simplify our system\ldots
\emph{on y va!}\footnote{French for ``Let's go!"}

\item we'll start by subtracting twice the first equation from the second:
\begin{equation} 
\boxed{
\begin{split}
    6 \alpha   &                 & -1 \gamma  &                   &= 0 \\
    2 \alpha  &                  & +2\gamma  & -2 \epsilon  &= 0 \\
                   &  2 \beta    & -2 \gamma  & -1 \epsilon  &= 0 \\
\end{split}
 }
\end{equation}

\item now we'll divide the second equation by $2$:
\begin{equation} 
\boxed{
\begin{split}
    6 \alpha   &                 & -1 \gamma  &                   &= 0 \\
       \alpha   &                 & +1 \gamma  & -1 \epsilon  &= 0 \\
                   &  2 \beta    & -2 \gamma  & -1 \epsilon  &= 0 \\
\end{split}
 }
\end{equation}

\item now we'll subtract $6$ times the second equation from the first: 
\begin{equation} 
\boxed{
\begin{split}
                    &                 & -7 \gamma  & +6 \epsilon &= 0 \\
       \alpha   &                 & +1 \gamma & -1 \epsilon  &= 0 \\
                   &  2 \beta    & -2 \gamma  & -1 \epsilon  &= 0 \\
\end{split}
 }
\end{equation}

\item now we'll rearrange our equations into 2, 3, 1: 
\begin{equation} 
\boxed{
\begin{split}
       \alpha   &                 & +1 \gamma & -1 \epsilon  &= 0 \\
                   &  2 \beta    & -2 \gamma  & -1 \epsilon  &= 0 \\
                   &                 & -7 \gamma  & +6 \epsilon &= 0 \\
\end{split}
 }
\end{equation}

\item now we'll multiply equation 2 by $\frac{7}{2}$ and subtract equation 3 from it:
\footnote{I bet you were wondering where 19 would show up, weren't you? I'll be honest: I was happy to see it show up\ldots I was starting to worry.} 
\begin{equation} 
\boxed{
\begin{split}
       \alpha   &                 & +1 \gamma & -1 \epsilon  &= 0 \\
                   &  7 \beta    &                     & -\frac{19}{2} \epsilon  &= 0 \\
                   &                 & -7 \gamma  & +6 \epsilon &= 0 \\
\end{split}
 }
\end{equation}

\item now we'll multiply equation 1 by $7$ and add equation 3:
\begin{equation} 
\boxed{
\begin{split}
       7 \alpha &                &                    & -1 \epsilon  &= 0 \\
                   &  7 \beta    &                     & -\frac{19}{2} \epsilon  &= 0 \\
                   &                 & -7 \gamma   & +6 \epsilon &= 0 \\
\end{split}
 }
\end{equation}

\item now we'll multiply equations 1 and 2 by $\frac{1}{7}$ and equation 3 by $\frac{-1}{7}$:
\begin{equation} 
\boxed{
\begin{split}
        \alpha &                &                   &- \frac{1}{7} \epsilon  &= 0 \\
                   &  \beta      &                   & -\frac{19}{14} \epsilon  &= 0 \\
                   &                & \gamma     &- \frac{6}{7} \epsilon &= 0 \\
\end{split}
 }
\end{equation}

\item which means:
\begin{equation} 
\boxed{
\begin{split}
        \alpha &= \frac{1}{7} \epsilon\\
        \beta   &= \frac{19}{14} \epsilon \\
        \gamma  &= \frac{6}{7} \epsilon\\
\end{split}
 }
\end{equation}

\item so now we just have to pick a ``good" value for $\epsilon$, let's try the LCD, $14$:
\begin{equation} 
\boxed{
\begin{split}
        \alpha &= 2 \\
        \beta   &= 19 \\
        \gamma  &= 12 \\
         \epsilon &= 14\\
\end{split}
 }
\end{equation}
\item isn't that amazing? we found the right answer!
\end{itemize}


\clearpage
\section{Solving by Elimination in a Matrix}
\begin{itemize}
\item this is \emph{exactly} the same as solving equations by elimination (above), but with a better notation
\item so first we assemble our system of equations into a matrix, which looks like:\\
 $$
\begin{bmatrix}
6   &  0 & -1 &  0 & 0 \\
14 &  0 &  0 & -2 & 0 \\ 
0   &  2 & -2 & -1 & 0 \\ 
\end{bmatrix}
$$

\item we're going to add, subtract, and simplify our system just like we did before, but now we think in terms of Rows.
\footnote{Remember, each row in the matrix is an equation in our original system.}

\item we'll start by subtracting $2 \times$ the first equation from the second, which we abbreviate $ R2 - 2 R1:$\footnote{``Row 2 minus 2 times Row 1"}\\
$$
\begin{bmatrix}
6   &  0 & -1 &  0 & 0 \\
2   &  0 &  2 & -2 & 0 \\ 
0   &  2 & -2 & -1 & 0 \\ 
\end{bmatrix}
$$

\item now we'll divide the second equation by $2$, which we abbreviate $\frac{1}{2}R2$: \\
$$
\begin{bmatrix}
6   &  0 & -1 &  0 & 0 \\
1   &  0 &  1 & -1 & 0 \\ 
0   &  2 & -2 & -1 & 0 \\ 
\end{bmatrix}
$$

\item now we'll subtract $6$ times the second equation from the first ($R1 - 6R2$):\\ 
$$
\begin{bmatrix}
0   &  0 & -7 &  6 & 0 \\
1   &  0 &  1 & -1 & 0 \\ 
0   &  2 & -2 & -1 & 0 \\ 
\end{bmatrix}
$$


\item now we'll rearrange our equations, so $R2 \leftrightarrow R1, R2 \leftrightarrow R3 $ :\\ 
$$
\begin{bmatrix}
1   &  0 &  1 & -1 & 0 \\ 
0   &  2 & -2 & -1 & 0 \\ 
0   &  0 & -7 &  6 & 0 \\
\end{bmatrix}
$$


\item now $2R2 - \frac{7}{2} R3$ \\
$$
\begin{bmatrix}
1   &  0 &  1 & -1                 & 0 \\ 
0   &  7 &  0 & -\frac{19}{2} & 0 \\ 
0   &  0 & -7 &  6                 & 0 \\
\end{bmatrix}
$$

\item  $7R1 + R3$:\\
$$
\begin{bmatrix}
7   &  0 &  0 & -1                 & 0 \\ 
0   &  7 &  0 & -\frac{19}{2} & 0 \\ 
0   &  0 & -7 &  6                 & 0 \\
\end{bmatrix}
$$

\item $\frac{1}{7} R1$, $\frac{1}{7} R2$, $\frac{-1}{7} R3$:\\
$$
\begin{bmatrix}
1   &  0 &  0 &  \frac{-1}{7}    & 0 \\ 
0   &  1 &  0 & -\frac{19}{14} & 0 \\ 
0   &  0 &  1 &  - \frac{6}{7}   & 0 \\
\end{bmatrix}
$$
\item this is as simplified as our matrix will get, as we have fewer equations than unknowns, so we're done
\item we can rewrite our matrix as:
\begin{equation} 
\boxed{
\begin{split}
        1 \alpha &+ 0 \beta  &+ 0 \gamma   & - \frac{1}{7} \epsilon       &= 0\\
        0 \alpha &+ 1\beta   &+ 0 \gamma   & - \frac{19}{14} \epsilon       &= 0\\
        0 \alpha &+ 0 \beta  &+ 1 \gamma   & - \frac{6}{7} \epsilon       &= 0\\
\end{split}
 }
\end{equation}

\item which brings us right back to the same solution set:
\begin{equation} 
\boxed{
\begin{split}
        \alpha &             &                &=  \frac{1}{7} \epsilon     \\
                   & \beta   &                 &=  \frac{19}{14} \epsilon \\
                   &            & \gamma   &=  \frac{6}{7} \epsilon     \\
\end{split}
 }
\end{equation}

\item so we can see it's possible to use matrices as shorthand for elimination in systems of linear equations
\footnote{Not everyone will find this technique easier to read, but I certainly do. It tends to be laborious for smaller problems, but scales very well.}
\end{itemize}



\section{Solving by Substitution}
\begin{itemize}
\item the idea here is that we can figure out one variable in terms of another, then we can eliminate variables
\item this can be a \emph{lot} more work that elimination (above), but it might be easier to understand each step
\footnote{And in this case, substitution is actually the shorter approach, but that's not always true.}
\item in our equations, we know $6 \alpha = \gamma$, so we can rewrite equation 3 like:
\begin{equation} 
\boxed{
\begin{split}
                    2 \beta    & -2 (6 \alpha)  & -1 \epsilon  &= 0 \\
\end{split}
 }
 \end{equation}
 \item and we know that $14 \alpha = 2 \epsilon$, so we can replace $\epsilon$ with $7 \alpha$:
 \begin{equation} 
\boxed{
\begin{split}
                    &  2 \beta    & -2 (6 \alpha)  & -1 (7 \alpha)  &= 0 \\
\end{split}
 }
 \end{equation}

\item so now we have $\beta$ in terms of $\alpha$:
 \begin{equation} 
\boxed{
\begin{split}
                    2 \beta    & -2 (6 \alpha)  & -1 (7 \alpha)  &=  0 \\
                    2 \beta    &                       & -19 \alpha     &= 0 \\
                    2 \beta    &                       &                      &= 19 \alpha \\
                    \beta       &                       &                      &= \frac{19}{2} \alpha \\
\end{split}
 }
 \end{equation}

\item so now we know:
\begin{equation} 
\boxed{
\begin{split}
   \alpha &= \frac{1}{7} \epsilon \\
   \beta  &= \frac{19}{2} \alpha \\
   \gamma &= 6 \alpha \\
\end{split}
 }
 \end{equation}
 
 \item we can then express them all in terms of $\epsilon$ by substituting $ \frac{1}{7} \epsilon$ everywhere we have $\alpha$:
 \begin{equation} 
\boxed{
\begin{split}
   \alpha &= \frac{1}{7} \epsilon \\
   \beta  &= \frac{19}{2} \alpha \\
             &= \frac{19}{2} (\frac{1}{7} \epsilon)\\
             &= \frac{19}{14} \epsilon \\
   \gamma &= 6 \alpha \\
                 &= 6 (\frac{1}{7} \epsilon)\\
                 &= \frac{6}{7} \epsilon \\
\end{split}
 }
 \end{equation}

\item so now we have a ``solved" system, but we still need to pick a good value for $\epsilon$:
  \begin{equation} 
\boxed{
\begin{split}
   \alpha &= \frac{1}{7} \epsilon \\
   \beta  &= \frac{19}{14} \epsilon \\
   \gamma &= \frac{6}{7} \epsilon \\
\end{split}
 }
 \end{equation}
 
 \item given our fractions, it makes the most sense to choose $\epsilon = 14$, so we get:
   \begin{equation} 
\boxed{
\begin{split}
   \alpha &= (\frac{1}{7})  (14) \\
              &= 2 \\
   \beta  &= (\frac{19}{14}) (14) \\
             &=  19 \\
   \gamma &= (\frac{6}{7}) (14) \\
                 &= 12 \\
   \epsilon &= 14 \\
\end{split}
 }
 \end{equation}
 
 \item notice this is \emph{exactly} the set of coefficients we found balancing our reaction ``the old way"
 \item notice, too, this is the set of coefficients we found using elimination (above)
 \footnote{It's almost like Algebra actually works or something.}
 \end{itemize}
 
 \section{Conclusion}
 We can balance chemical equations by keeping tabs on the atoms on each side. This way works, and it's probably the fastest approach for most of the chemical equations we'll see in our class.
 
 We can choose a more elegant approach by treating the \textbf{stoichiometric coefficients} as variables and constructing a system of linear equations to solve for those variables.
 \footnote{I found a short paper published in \emph{Applied Mathematics} explaining this approach\cite{hamid:balancing:linear}. I was delighted to see the author used a similar combustion reaction to ours as an example.
 My mother used to say, ``Great minds think alike, and fools seldom differ."} 

We can solve our system of equations by elimination (using matrices as a shorthand notation, if we like), or by substitution. Regardless, we note that we won't have a \emph{unique} solution to our system of equations, because we have fewer equations than unknowns. This example demonstrates that's not a real problem, as long as we choose one variable to be our ``baseline," and then select a usable value to assign it.
\footnote{This example also demonstrates that I was willing to burn a whole lot of time on my weekend to come up with this document. Perhaps I have too much time on my hands. The ever-growing list of projects around the homestead suggests it's an issue of priority more than time.}
 

 
\nocite{hamid:balancing:linear}
\bibliography{../Chemistry}{}
\bibliographystyle{apalike}
\end{document}  