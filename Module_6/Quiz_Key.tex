\documentclass[11pt,addpoints]{exam}   	% use "amsart" instead of "article" for AMSLaTeX format
\usepackage{geometry}                		% See geometry.pdf to learn the layout options. There are lots.
\geometry{letterpaper}                   		% ... or a4paper or a5paper or ... 
%\geometry{landscape}                		% Activate for rotated page geometry
%\usepackage[parfill]{parskip}    		% Activate to begin paragraphs with an empty line rather than an indent
\usepackage{graphicx}				% Use pdf, png, jpg, or eps§ with pdflatex; use eps in DVI mode
								% TeX will automatically convert eps --> pdf in pdflatex	
\usepackage{cancel}									
\usepackage{amssymb}
\usepackage{mhchem}

\title{Module 6 Quiz}
\author{Mark Peever}
\date{November 14, 2025}							% Activate to display a given date or no date

\begin{document}
\maketitle

\pointsinrightmargin
%\marginpointname{ \points}
\printanswers

\begin{center}
\fbox{\fbox{\parbox{5.5in}{\centering
$m = n \times M$ or $m = n \times m_{molar}$\\
$1 amu = 1 \frac{g}{mol}$ \\
$N_A = 6.02214076 \times 10^{23} $ \\
}}}
\end{center}
\vspace{0.1in}
\makebox[\textwidth]{Name:\enspace\hrulefill}
\vspace{0.2in}

\begin{questions}
\question[5] Ella has $500.0 g$ of propane gas (\ce{C3H8}). How many \emph{moles} of propane does Ella have?\\
(Show your work)

\begin{solution}
\begin{equation} 
\begin{split}
%    M_{C} &= 12.0107 \frac{g}{mol} \\
%    M_{H} &= 1.00794 \frac{g}{mol} \\
    M_{propane} &= 3 M_{C} + 8 M_{H} \\
                          &= 3 (12.0107 \frac{g}{mol}) + 8 (1.00794 \frac{g}{mol}) \\
                          &= 36.0321 \frac{g}{mol} + 8.06352 \frac{g}{mol} \\
                          &= 44.09562  \frac{g}{mol} \\
                          &= 44.0956  \frac{g}{mol} \\
    m &= n \times M \\
    n &= \frac{m}{M} \\
       &= \frac{500.0 \xcancel{g}}{44.0956  \frac{\xcancel{g}}{mol}} \\
       &= 11.3389998095 mol \\
       &= 11.34 mol
 \end{split}
 \end{equation}
 \end{solution}

\question[5] Ella wants to burn her propane completely, so she puts it into a balloon and Nate touches it with a cattle prod (\emph{BOOM!}).
Ella and Nate know that the propane has reacted with the oxygen in the air to produce water and carbon dioxide.\\
Write the chemical reaction that occurred. Be sure your chemical equation is balanced!
\begin{solution}
\ce{C3H8 + 5O2 -> 3CO2 + 4H2O}
\end{solution}

\question[5] Vianne is curious how much \ce{O2} was consumed in the reaction, so she does a very quick calculation to figure it out.
How many moles of \ce{O2} were consumed?

\begin{solution}
There are $5$ moles of \ce{O2} consumed for every mole of \ce{C3H8} in our reaction: \ce{C3H8 + 5O2 -> 3CO2 + 4H2O}

\begin{equation} 
\begin{split}
    \frac{n_{\ce{O2}}}{5} &= \frac{n_{\ce{C3H8}}}{1} \\
    n_{\ce{O2}} &= 5 n_{\ce{C3H8}} \\
                       &= 5 (11.34 mol ) \\
                       &= 56.70 mol 
 \end{split}
 \end{equation}

\end{solution}


\question[5] No\'emi, like Vianne, wants to know how much \ce{O2} was consumed, but she's more interested in the oxygen's \emph{mass}. How many \emph{grams} of \ce{O2} were consumed?

\begin{solution}
\begin{equation} 
\begin{split}
    M_{\ce{O2}} &= 2 (15.9994 \frac{g}{mol}) \\
                        &= 31.9988 \frac{g}{mol} \\                        
    m &= n \times M \\
        &= (56.70 mol ) (31.9988 \frac{g}{mol}) \\
        &= (56.70 \xcancel{mol} ) (31.9988 \frac{g}{\xcancel{mol}}) \\
        &= 1814.33196 g \\
        &= 1814 g
 \end{split}
 \end{equation}
 \end{solution}


\question[3] Ezra isn't that interested in mass or in moles, but he's curious how many balloons worth of \ce{CO2} gas was produced in the reaction.
If Ella managed to get all her propane into a single balloon, how many \emph{balloons} would it take to contain all the \ce{CO2}? 

\begin{solution}
By the Gay-Lussac Law, the volume ratio of \ce{C3H8} to \ce{CO2} is $1:3$.\\
Therefore, there are $3$ balloons worth of \ce{CO2} produced.
\end{solution}


\question[2] Thomas is pretty sure there was an \emph{excess} of oxygen in that \emph{BOOM!} reaction, and that the propane was the \emph{limiting} reactant.
Is Thomas right? How do you know?

\begin{solution}
Thomas is correct. An outdoor combustion has virtually limitless \ce{O2} to consume, so \ce{C3H8} is the limiting reactant.
\end{solution}


\end{questions}

\begin{center}
This exam has \numquestions\ questions for a total of \numpoints\ points.
\end{center}

\end{document}  