\documentclass[10pt, oneside]{article}   	% use "amsart" instead of "article" for AMSLaTeX format
\usepackage{geometry}                		% See geometry.pdf to learn the layout options. There are lots.
\geometry{letterpaper}                   		% ... or a4paper or a5paper or ... 
%\geometry{landscape}                		% Activate for rotated page geometry
%\usepackage[parfill]{parskip}    		% Activate to begin paragraphs with an empty line rather than an indent
\usepackage{graphicx}				% Use pdf, png, jpg, or eps§ with pdflatex; use eps in DVI mode
								% TeX will automatically convert eps --> pdf in pdflatex		
\usepackage{amssymb}

\usepackage{hyperref}

\title{Module 3, Part 2}
\author{Mark Peever\\ \texttt{mpeever@gmail.com}}
\date{September 26, 2025}

\begin{document}
\maketitle

\section*{Objectives}
\marginpar{0 minutes}
Refer to \href{https://drive.google.com/file/d/1a59ZMchwsvZDRRW7-2GuTrvZKAS8EWCE/view?usp=sharing}{Module 3 Notes.pdf}.\\


By the end of this class, the students should be able to\ldots
\begin{itemize}
\item state and explain the Law of Conservation of Matter (Mass Conservation)
\item give a succinct definition of ``Element"
\item give a succinct definition of ``Compound"

\end{itemize}

\section*{Welcome \& Devotion}
\marginpar{5 minutes}
\begin{itemize}
\item have one student read \href{https://tinyurl.com/26k8csmf}{Acts 17:22–32}
\end{itemize}

\section*{Law of Mass Conservation}
\marginpar{5 minutes}
\begin{itemize}
\item the Law of Conservation of Matter is a fundamental scientific law!
\item Discuss:
\begin{itemize}
\item what happens when a candle burns?
\item where does the gasoline in your tank go?
\item why does your engine oil need topping up?
\end{itemize} 
\end{itemize}

\section*{Elements}
\marginpar{20 minutes}
\begin{itemize}
\item (Definition) Decomposition is breaking down a substance into two or more other substances.
\item (Definition) An Element is a substance that cannot be decomposed into different substances
\item Elements are listed in the Periodic Table of Elements (p. 75):
\begin{itemize}
\item go over element name
\item go over element symbol
\item go over atomic mass
\item go over an element's location on the table
\end{itemize}
\end{itemize}

\section*{Compounds}
\marginpar{40 minutes}

\begin{itemize}
\item (Definition) A compound is a substance that can be decomposed into elements by chemical means.
\item (The Law of Definite Proportions) The proportion of elements in any compound is always the same.
\item (The Law of Multiple Proportions) If two elements combine to form different compounds, 
the ratio of masses of the second element that react with a fixed mass of the first element will be a simple, whole-number ratio.
\item compounds are made up of molecules, just like elements are made up of atoms
\item (Definition) A compound made of at least one metal and at least one non-metal is an Ionic Compound
\item (Definition) A compound made solely of non-metal atoms is a Covalent Compound
\end{itemize}


\section*{Questions}
\marginpar{5 minutes}

\section*{Assignment}
\begin{itemize}
\item Review Problems: p. 97 \# 1--10 (not to be turned in)
\item Practice Problems: p. 98 \# 1--10 (due 2025-10-03)
\end{itemize}



\end{document}  