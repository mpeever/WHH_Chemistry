\documentclass[11pt, oneside]{article}   	% use "amsart" instead of "article" for AMSLaTeX format
\usepackage{geometry}                		% See geometry.pdf to learn the layout options. There are lots.
\geometry{letterpaper}                   		% ... or a4paper or a5paper or ... 

\usepackage[parfill]{parskip}    		% Activate to begin paragraphs with an empty line rather than an indent
\usepackage{graphicx}				% Use pdf, png, jpg, or eps§ with pdflatex; use eps in DVI mode
								% TeX will automatically convert eps --> pdf in pdflatex		
\usepackage{amssymb}
\usepackage{cite}

% numbered examples
\usepackage{gb4e}
\usepackage{enumitem}
\usepackage{cancel}
\usepackage{amsmath}

\usepackage{mhchem}

% Scientific Laws
\newtheorem{definition}{Definition}
\newtheorem{law}{Law}
\newtheorem{theory}{Theory}
\newtheorem{hint}{Hint}

\title{Module 5: Counting Molecules and Atoms in Chemical Equations }
\author{Mark Peever \texttt{mpeever@gmail.com}}
\date{October 24 -- 31, 2025}

\begin{document}
\maketitle

\begin{center}

\end{center}

\section{Overview}
\begin{enumerate}
\item \textbf{Decomposition Reactions} 
\item \textbf{Formation Reactions}
\item \textbf{Complete Combustion Reactions}
\item \textbf{Incomplete Combustion Reactions}
\item \textbf{Molecular Mass}
\item A \textbf{Mole} is a fixed number of objects (usually molecules or atoms).
\end{enumerate}

\section{Classifying Chemical Reactions}

\subsection{Decomposition Reactions}

\begin{definition}[Decomposition Reaction]\label{defn:reaction:decomposition}
A reaction that changes a compound into its constituent elements
\end{definition}

\begin{itemize}
\item these are reasonably easy to predict
\item they often (not always) involve some sort of energy input\footnote{A stable compound represents a \emph{localized} low-energy state, so getting it to decompose can require some energy input.}
\item we can think of our old friend: \ce{2H2O (l) -> 2H2 (g) + O2 (g) }
\end{itemize} 


\subsection{Formation Reactions}

\begin{definition}[Formation Reaction]\label{defn:reaction:formation}
A reaction that starts with two or more elements and produces one compound
\end{definition}
\begin{itemize}
\item these are really the opposite of decomposition reactions
\item we can even write them by writing decomposition reactions backwards: \ce{2H2 (g) + O2 (g) ->  2H2O (l)} 
\end{itemize}


\subsection{Complete Combustion Reactions}

\begin{definition}[Complete Combustion Reaction]\label{defn:reaction:complete-combustion}
A reaction in which \ce{O2} is added to a compound containing carbon (\ce{C}) and hydrogen (\ce{H}), producing \ce{CO2} and \ce{H2O}
\end{definition}
\begin{itemize}
\item this is a narrow definition of \emph{burning}\footnote{This is really a glimpse into the exciting world of \emph{organic chemistry}.}
\item this definition excludes some really exciting combustion reactions, like \ce{2Mg (s) + O2 (g) -> 2MgO (s) }
\item technically, our \emph{formation} example is a combustion, but we're classifying that differently for now
\end{itemize}


\subsection{Incomplete Combustion Reactions}

%\begin{definition}[Incomplete Combustion Reaction]\label{defn:reaction:incomplete-combustion}
%A reaction in which \ce{O2} is added to a compound containing carbon (\ce{C}) and hydrogen (\ce{H}), producing \ce{CO2} and \ce{H2O}
%\end{definition}
\begin{itemize}
\item these are combustion reactions where insufficient \ce{O2} means the combustion produces \ce{CO} or even just \ce{C} instead of \ce{CO2}
\item these can be pretty harmful --- even dangerous (breathing \ce{CO} can actually kill you) --- but they can also be beneficial\footnote{This is, after all, how we make charcoal.}
\item the main take-away here is just that a combustion reaction can go one of three ways, depending on the \ce{O2} levels in the environment
\end{itemize}


\section{Atomic Mass}

We've already talked about The Law of Definite Proportions, and how that led Dalton to formulate theories about atoms. Now we're going to dig more into that\ldots

\begin{itemize}
\item every element on the Periodic Table contains two important numbers: the \emph{atomic number} (above the element symbol), and the \emph{atomic mass} (below the symbol)
\item remember that an element is made up of only one type of atom, so we'll use the terms ``atom" and ``element" interchangeably here
\item the atomic number tells us the number of protons (and hence the number of electrons) in the atom\footnote{More on this later}
\item the atomic mass tells us how much mass the atom has, in atomic mass units (\textbf{amu})
\item the relationship between amu and grams is: $ 1 g = 6.02214076 \times 10^{23} amu$
\item so one \ce{H} atom has the mass of $ 1.00794 amu$, the mass of $ 6.02214076 \times 10^{23} $ \ce{H} atoms is $ 1.00794 g$
\item that means our Periodic Table can help us calculate how many atoms of any given element we have, if we know the mass of the sample:
\begin{enumerate}[label=Example \arabic*]
\item How many atoms are in $1.00 kg$ of gold (\ce{Au})? 
\begin{equation} 
\boxed{
\begin{split}
    m_{unit} &= \frac{196.966569 amu}{1 atom} \\
    m_{sample} &= 1.00 kg \\
                        &= ( 1.00 kg ) (\frac{1000 g}{1 kg}) ( \frac{6.02214076 \times 10^{23} amu}{1 g}) \\
                        &= ( 1.00 \xcancel{kg} ) (\frac{1000 \xcancel{g}}{1 \xcancel{kg}}) ( \frac{6.02214076 \times 10^{23} amu}{1 \xcancel{g}}) \\
                        &= 6.02214076 \times 10^{26} amu \\
   count &= \frac{m}{m_{unit}} \\
             &=  (6.02214076 \times 10^{26} amu) (\frac{1 atom}{196.966569 amu}) \\
             &=  (6.02214076 \times 10^{26} \xcancel{amu}) (\frac{1 atom}{196.966569 \xcancel{amu}}) \\
             &=  3.0574430933 \times 10^{24} atoms \\
             &=  3.05744309 \times 10^{24} atoms \\                   
 \end{split}
 }
 \end{equation}
 
\end{enumerate}
\end{itemize}

\section{Molecular Mass}


\section{Mole}

\begin{definition}[Mole]\label{defn:mole}
A mole is $6.02214076 \times 10^{23}$  objects.
\end{definition}

\begin{itemize}
\item a \emph{mole} is a number of objects
\item a mole is like a pair, or a dozen, or a gross (see Table \ref{table:collections:names})
\item an element's mass in \emph{amu} is the same as a mole of that element's mass in \emph{g}\footnote{This is why there aren't mass units in the Periodic Table: a single \ce{H} atom has a mass of $1.01 amu$, a mole of \ce{H} atoms has a mass of $1.01 g$.}
\item because atoms and molecules are tiny\footnote{Like really, really small.}, we find it easier to measure out \emph{moles} than \emph{atoms}
\item let's consider our old friend: \ce{2H2 + O2 -> 2H2O}
\begin{itemize}
\item so it takes two \ce{H2} molecules and one \ce{O2} molecule to make two \ce{H2O} molecules
\item or, we could say it takes two dozen \ce{H2} molecules and one dozen \ce{O2} molecule to make two dozen \ce{H2O} molecules
\item or, we could say it takes two gross of \ce{H2} molecules and one gross of \ce{O2} molecule to make two gross of \ce{H2O} molecules
\item or, we could say it takes two moles of \ce{H2} molecules and one mole of \ce{O2} molecule to make two mole of \ce{H2O} molecules
\end{itemize}
\item technically, chemical reactions occur ``per each," but we find it much easier to measure them ``per mole"
\item so all our chemistry is going to be done ``per mole" 
\end{itemize}

\begin{table}[p]
\centering
\begin{tabular}[b]{l|l}
\hline
Name & Number \\
\hline
%each & $1$ \\
pair  & $2$ \\
%duo  & $2$ \\
trio  & $3$ \\ 
half-dozen & $6$ \\
dozen & $12$ \\
baker's dozen & $13$ \\
score & $20$ \\
gross & $144$ \\
\textbf{mole}   & \textbf{$6.022 \times 10^{23}$}\\
\end{tabular}
\caption{Names of collections of objects}
\label{table:collections:names}
\end{table}


\section{Homework}
Review Problems: p. 161 \# 1--10 (not to be turned in)\\
Practice Problems: p. 162 \# 1--10 (due 2025-11-07)\\
Experiment 5.1, p. 149 (due 2025-11-07)\\


\nocite{wile-chem-2}
\bibliography{../Chemistry}{}
\bibliographystyle{apalike}
\end{document}  