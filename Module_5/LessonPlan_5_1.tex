\documentclass[10pt, oneside]{article}   	% use "amsart" instead of "article" for AMSLaTeX format
\usepackage{geometry}                		% See geometry.pdf to learn the layout options. There are lots.
\geometry{letterpaper}                   		% ... or a4paper or a5paper or ... 
%\geometry{landscape}                		% Activate for rotated page geometry
%\usepackage[parfill]{parskip}    		% Activate to begin paragraphs with an empty line rather than an indent
\usepackage{graphicx}				% Use pdf, png, jpg, or eps§ with pdflatex; use eps in DVI mode
								% TeX will automatically convert eps --> pdf in pdflatex		
\usepackage{amssymb}
\usepackage{mhchem}
\usepackage{hyperref}

\title{Module 5, Part 1}
\author{Mark Peever\\ \texttt{mpeever@gmail.com}}
\date{October 24, 2025}

\begin{document}
\maketitle

\section*{Objectives}
\marginpar{0 minutes}
Refer to \href{https://drive.google.com/file/d/1CKOE3BX37Ix59TutlOzFNpghvfOVk0r7/view?usp=drive_link}{Module 5 Notes.pdf}.\\

By the end of this class, the students should be able to\ldots
\begin{itemize}
\item differentiate between several different types of chemical reaction:
\begin{itemize}
\item  decomposition reactions
\item  formation reactions
\item  complete combustion reactions
\item  incomplete combustion reactions
\end{itemize}
\item describe molecular mass
\item give a succinct description of a \emph{mole}
\end{itemize}

\section*{Welcome \& Devotion}
\marginpar{5 minutes}
\begin{itemize}
\item have one student read \href{https://www.biblegateway.com/passage/?search=psalm\%20107\&version=LSB}{Psalm 107:33--42}
\end{itemize}

\section*{Types of Chemical Reactions}
\marginpar{20 minutes}
\begin{itemize}
\item work through definitions of:
\begin{itemize}
\item  decomposition reactions
\item  formation reactions
\item  complete combustion reactions
\item  incomplete combustion reactions
\end{itemize}
\item classify the following reactions after balancing them:
\begin{itemize}
\item how do we classify \ce{H2 + O2 -> H2O}?
\item how do we classify \ce{C3H8 + O2 -> CO2 + H2O}?
\item how do we classify \ce{C8H18 + O2 -> CO + H2O}?
\end{itemize}
\end{itemize}

\section*{Molecular Mass}
\marginpar{20 minutes}
\begin{itemize}
\item work out on the board:
\begin{itemize}
\item  molecular mass of \ce{H2O}
\item  molecular mass of \ce{NaOH}
\item  molecular mass of \ce{H2SO4}
\item  molecular mass of \ce{CH4} (methane)
\item  molecular mass of \ce{C4H10} (butane)
\item  molecular mass of \ce{C3H8} (propane)
\item  molecular mass of \ce{C6H14} (hectane)
\item  molecular mass of \ce{C8H18} (octane)
\item  molecular mass of \ce{C12H22O11} (sucrose)
\end{itemize}
\end{itemize}

\section*{The marvelous Mole}
\marginpar{20 minutes}
\begin{itemize}
\item what is a mole? $1 mol = 6.02214076 \times 10^{23} objects$
\item Avagadro's number: $ N_A = 6.02214076 \times 10^{23} objects$
\end{itemize}

\begin{table}[h]
\centering
\begin{tabular}[b]{l|l}
\hline
Name & Number \\
\hline
pair  & $2$ \\
trio  & $3$ \\ 
half-dozen & $6$ \\
dozen & $12$ \\
baker's dozen & $13$ \\
score & $20$ \\
gross & $144$ \\
\textbf{mole}   & \textbf{$6.022 \times 10^{23}$}\\
\end{tabular}
\end{table}

\section*{Questions for me}
\marginpar{10 minutes}

\section*{Assignment}
\marginpar{5 minutes}
\begin{itemize}
\item Review Problems: p. 161 \# 1--10 (not to be turned in)
\item Practice Problems: p. 162 \# 1--10 (due 2025-11-07)
\item Experiment 5.1, p. 149 (due 2025-11-07)
\end{itemize}



\end{document}  