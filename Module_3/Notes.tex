\documentclass[11pt, oneside]{article}   	% use "amsart" instead of "article" for AMSLaTeX format
\usepackage{geometry}                		% See geometry.pdf to learn the layout options. There are lots.
\geometry{letterpaper}                   		% ... or a4paper or a5paper or ... 

\usepackage[parfill]{parskip}    		% Activate to begin paragraphs with an empty line rather than an indent
\usepackage{graphicx}				% Use pdf, png, jpg, or eps§ with pdflatex; use eps in DVI mode
								% TeX will automatically convert eps --> pdf in pdflatex		
\usepackage{amssymb}
\usepackage{cite}

% numbered examples
\usepackage{gb4e}
\usepackage{enumitem}
\usepackage{cancel}
\usepackage{amsmath}

\usepackage{mhchem}

% Scientific Laws
\newtheorem{definition}{Definition}
\newtheorem{law}{Law}

\title{Module 3: Atoms and Molecules }
\author{Mark Peever \texttt{mpeever@gmail.com}}
\date{September 19 -- 26, 2025}

\begin{document}
\maketitle

\begin{center}

\end{center}

\section{Overview}
\begin{enumerate}
\item the \textbf{Law of Mass Conservation} says matter can't be created or destroyed
\item \textbf{Elements} are the color palette for matter
\item \textbf{Compounds} are made of elements that are chemically combined
\item the \textbf{Periodic Table} is a thing of rare beauty that will haunt your dreams
\end{enumerate}

\section{Mass Conservation}

\begin{law}[The Law of Conservation of Mass]\label{law-conservation-mass}
Matter cannot be created or destroyed, it can only change forms.
\end{law}

\begin{itemize}
\item this is another conservation law, like the energy conservation law in Module 2
\item we can't actually create matter, and we can't actually destroy it, all we can do is convert it from one form to another
\footnote{OK, so this isn't \emph{exactly} true: Lavoisier (1743 -- 1794) came along when people were still trying to digest Newton (1642 -- 1727), they weren't ready for Einstein (1879 -- 1955) yet. Let's leave this one here for now, but you should be aware that nuclear reactions really do convert matter into energy.}
\item we can convert matter with chemical processes (\emph{e.g.} we can burn wood or wax)
\item we can change the state of matter with physical processes (\emph{e.g.} we can freeze water to make ice or evaporate water to make water vapor)
\item but we generally can't make more of it, and we can't make less of it
\end{itemize}

\section{Elements}
\begin{definition}[Decomposition]
Decomposition is breaking down a substance into two or more other substances.
\end{definition}

\begin{definition}[Element]
An element is a substance that cannot be decomposed into a less massive substance.
\end{definition}

\begin{itemize}
\item \textbf{decomposition} is breaking down a substance into other substances
\item many (most) substances can be decomposed, but some cannot
\item substances that cannot be decomposed are called \textbf{elements} \cite[p. 74]{wile-chem-2}
\item every physical thing is made up of one or more elements:
\begin{itemize}
\item water is made up of two gasses: Oxygen and Hydrogen
\item sulfuric acid is made up of three gasses: Hydrogen, Sulfur, and Oxygen
\item steel is made up of two solids: Iron and Carbon
\item Iron is made of Iron --- it's an element \footnote{It's iron all the way down!}
\end{itemize}
\end{itemize}

\subsection{The Periodic Table}
\begin{itemize}
\item each entry in the Periodic Table contains four pieces of information:
\begin{enumerate}
\item the element's \textbf{symbol} (\emph{e.g.} H, He, Li, Be, B, C, \ldots)
\item the element's \textbf{atomic number}
\item the element's \textbf{atomic mass}
\item the element's location on the chart
\end{enumerate}
\item in general terms, the element's symbol acts as a mnemonic, although they don't always work the way we might think (why is lead called ``Pb"?)
\item the atomic number is unique: it's the defining feature of an element\footnote{On some Periodic Tables, you'll see Hydrogen (H) listed twice, because Hydrogen is weird.}
\item we'll get to the atomic mass later
\item be aware that not all the elements occur naturally \cite[p. 76]{wile-chem-2}, there are more elements now than when I was in school(?!)
\item elements on the \emph{left} of the Periodic Table are metals; elements on the \emph{right} are non-metals, \emph{except Hydrogen (H)}\footnote{I told you Hydrogen is weird.}
\end{itemize}

\section{Compounds}
\begin{definition}[Compound]
A compound is a substance that can be decomposed into elements by chemical means.
\end{definition}

\begin{law}[The Law of Definite Proportions]\label{law-definite-proportions}
The proportion of elements in any compound is always the same.
\end{law}

\begin{itemize}
\item there are basically two types of matter: elements and compounds
\item a compound is a substance made up of elements (\emph{e.g.} water, steel, sulfuric acid)
\item note compounds are made of elements combined \emph{chemically}: bolting a piece of Iron to a piece of Tin doesn't make a compound
\end{itemize}

\section{The Law of Multiple Proportions}

\begin{law}[The Law of Multiple Proportions]\label{law-multiple-proportions}
If two elements combine to form different compounds, the ratio of masses of the \emph{second} element that react with a fixed mass of the \emph{first} element 
will be a simple, whole-number ratio.
\end{law}

\begin{itemize}
\item this one is easier to understand than it seems:
\begin{itemize}
\item if we have two compounds made up of two elements
\item and if they use the same amount (mass) of one of the elements, then:
\item the amount (mass) of the \emph{other} element in one compound will be a simple ratio of the mass of that same element in the \emph{other} compound
\end{itemize}
\item so if you compare water (\ce{H2O}) and hydrogen peroxide (\ce{H2O2}), if you have the same amount of \ce{H} in each compound (by \emph{mass}), then you'll have twice as much \ce{O} in the second compound as in the first
\item or if you compare water and hydronium (\ce{H3O^+}), if you have the same amount of \ce{O} in each compound, the ratio of \ce{H} in the two compounds will be $2:3$, a simple, whole-number ratio 
\item or if you compare hydrogen peroxide and hydroxyl (\ce{OH^-}), if you have the same amount of \ce{O} in each compound, the ratio of \ce{H} in the two compounds will be $1:1$, a simple, whole-number ratio 
\end{itemize}

\section{Dalton's Atomic Theory}
\begin{itemize}
\item John Dalton (1766 -- 1844) came up with an atomic theory\cite[p. 83--85]{wile-chem-2}
\begin{enumerate}
\item all elements are composed of small, indivisible particles called ``atoms"
\item all atoms of the same element have exactly the same properties
\item atoms of different elements have different properties
\item compounds are formed when atoms are joined together --- since atoms are indivisible, they can only join in simple, whole-number ratios
\end{enumerate}
\item as far as it goes, Dalton's Atomic Theory is [mostly] correct
\item chemical reactions are just atoms rearranging: they don't create new atoms, nor do they destroy existing atoms
\item the total number of atoms stays the same, so the total amount of mass stays the same (see Law \ref{law-conservation-mass})
\item since atoms are indivisible, they can only combine in consistent ratios (see Law \ref{law-definite-proportions})
\item since atoms are indivisible, they can only combine in whole-number ratios (see Law \ref{law-multiple-proportions})
\item it turns out atoms aren't actually indivisible: they, too, are made of smaller particles\cite[p. 85]{wile-chem-2}
\item it turns out atoms of a single element aren't actually identical: there are slight variations in mass between \emph{isotopes}\cite[p. 85]{wile-chem-2}
\end{itemize}


\section{Molecules}
\begin{itemize}
\item compounds are made of \emph{molecules}, which are groups of connected atoms
\item just like elements are made of identical atoms, compounds are made of identical molecules
\end{itemize}


\section{Abbreviating and Classifying Compounds}
\begin{itemize}
\item we write compounds based on the elements that make them up
\item the written compounds are called \emph{chemical formulae}
\item \emph{e.g.} water is made up of two Hydrogen atoms and one Oxygen atom, so we write it as \ce{H2O}
\item \emph{e.g.} table salt is made up of one Sodium atom and one Chlorine atom, so we write it as \ce{NaCl}\footnote{Never write it as \ce{TS}, always write it as \ce{NaCl}.}
\item \emph{e.g.} sulfuric acid is written as \ce{H2SO4}, so each molecule contains two Hydrogen atoms, one Sulfur atom, and four Oxygen atoms\footnote{This one will make more sense in Module \#9.}
\end{itemize}


\section{Ionic and Covalent Compounds}
\begin{definition}[Iconic Compound]
A compound made of at least one metal and at least one non-metal is an \emph{Iconic Compound}
\end{definition}

\begin{definition}[Covalent Compound]
A compound made solely of non-metal atoms is a \emph{Covalent Compound}
\end{definition}

\subsection{Examples}
\begin{enumerate}[label=Example \arabic*]
\item Is \ce{H2O} an ionic or covalent compound?
\item Is \ce{NaCl} an ionic or covalent compound?
\item Is \ce{PbSO4} an ionic or covalent compound?
 \end{enumerate}


\section{Naming Compounds}
\begin{itemize}
\item to name ionic compounds:
\begin{enumerate}
\item start with the name of the first atom in the compound
\item take the next atom and replace its ending with the ``-ide" suffix
\item combine those to get a name
\end{enumerate}
\item so \ce{NaCl} is ``sodium chloride"
\item and \ce{PbO} is ``lead oxide"
\item and \ce{PbCl2} is ``lead chloride"
\item naming covalent compounds is a little more complicated, because they can form in more variations\ldots
\item to name covalent compounds:
\begin{enumerate}
\item start with the name of the first atom in the molecule
\item take the next atom and replace its ending with the ``-ide" suffix
\item add the appropriate prefix to each atom in order to indicate count (see Table \ref{table:covalent-prefix})
\item you can drop the ``mono-" prefix off the first atom
\item combine those to get a name
\end{enumerate}
\item so \ce{H2O} is ``dihydrogen monoxide"
\item so \ce{CO} is ``carbon monoxide"
\item so \ce{CO2} is ``carbon dixide"
\end{itemize}

\begin{table}
\centering
\begin{tabular}[b]{| l | l |}
\textbf{Prefix} & \textbf{Count} \\
\hline
mono & one \\
di       & two \\
tri       & three \\
tetra       & four \\
penta       & five \\
hecta       & six \\
hepta       & seven \\
octa       & eight \\
nona       & nine \\
deca       & ten \\
\end{tabular}
\caption{Prefixes for Covalent Compound Names}
\label{table:covalent-prefix}
\end{table}


Beware that there are many compounds with other names that don't seem to fit our scheme:
\begin{itemize}
\item \ce{H2O} (dihydrogen monoxide) is also called ``water"
\item \ce{H2O2} (dihydrogen dioxide) is also called ``hydrogen peroxide"
\item \ce{OH^-} is ``hydroxyl''\footnote{We'll spend a lot more time talking about this one later.}
\end{itemize}

\nocite{wile-chem-2}
\bibliography{../Chemistry}{}
\bibliographystyle{apalike}
\end{document}  