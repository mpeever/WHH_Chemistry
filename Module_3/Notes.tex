\documentclass[11pt, oneside]{article}   	% use "amsart" instead of "article" for AMSLaTeX format
\usepackage{geometry}                		% See geometry.pdf to learn the layout options. There are lots.
\geometry{letterpaper}                   		% ... or a4paper or a5paper or ... 

\usepackage[parfill]{parskip}    		% Activate to begin paragraphs with an empty line rather than an indent
\usepackage{graphicx}				% Use pdf, png, jpg, or eps§ with pdflatex; use eps in DVI mode
								% TeX will automatically convert eps --> pdf in pdflatex		
\usepackage{amssymb}
\usepackage{cite}

% numbered examples
\usepackage{gb4e}
\usepackage{enumitem}
\usepackage{cancel}
\usepackage{amsmath}

% Scientific Laws
\newtheorem{definition}{Definition}
\newtheorem{law}{Law}

\title{Module 3: Atoms and Molecules }
\author{Mark Peever \texttt{mpeever@gmail.com}}

\begin{document}
\maketitle

\begin{center}

\end{center}

\section{Overview}
\begin{enumerate}
\item the \textbf{Law of Mass Conservation} says matter can't be created or destroyed
\item \textbf{Elements} are the color palette for matter
\item \textbf{Compounds} are made of elements that are chemically combined
\item the \textbf{Periodic Table} is a thing of rare beauty that will haunt your dreams
\end{enumerate}

\section{Mass Conservation}

\begin{law}[The Law of Conservation of Mass]
Matter cannot be created or destroyed, it can only change forms.
\end{law}

\begin{itemize}
\item this is another conservation law, like the energy conservation law in Module 2
\item we can't actually create matter, and we can't actually destroy it, all we can do is convert it from one form to another
\footnote{OK, so this isn't \emph{exactly} true: Lavoisier (1743 -- 1794) came along when people were still trying to digest Newton (1642 -- 1727), they weren't ready for Einstein (1879 -- 1955) yet. Let's leave this one here for now, but you should be aware that nuclear reactions really do convert matter into energy.}
\item we can convert matter with chemical processes (\emph{e.g.} we can burn wood or wax)
\item we can change the state of matter with physical processes (\emph{e.g.} we can freeze water to make ice or evaporate water to make water vapor)
\item but we generally can't make more of it, and we can't make less of it
\end{itemize}

\section{Elements}
\begin{definition}[Decomposition]
Decomposition is breaking down a substance into two or more other substances.
\end{definition}

\begin{definition}[Element]
An element is a substance that cannot be decomposed into a less massive substance.
\end{definition}

\begin{itemize}
\item \textbf{decomposition} is breaking down a substance into other substances
\item many (most) substances can be decomposed, but some cannot
\item substances that cannot be decomposed are called \textbf{elements} \cite[p. 74]{wile-chem-2}
\item every physical thing is made up of one or more elements:
\begin{itemize}
\item water is made up of two gasses: Oxygen and Hydrogen
\item sulfuric acid is made up of three gasses: Hydrogen, Sulfur, and Oxygen
\item steel is made up of two solids: Iron and Carbon
\item Iron is made of Iron --- it's an element \footnote{It's iron all the way down!}
\end{itemize}
\end{itemize}

\subsection{The Periodic Table}
\begin{itemize}
\item each entry in the Periodic Table contains four pieces of information:
\begin{enumerate}
\item the element's \textbf{symbol} (\emph{e.g.} H, He, Li, Be, B, C, \ldots)
\item the element's \textbf{atomic number}
\item the element's \textbf{atomic mass}
\item the element's location on the chart
\end{enumerate}
\item in general terms, the element's symbol acts as a mnemonic, although they don't always work the way we might think (why is lead called ``Pb"?)
\item the atomic number is unique: it's the defining feature of an element\footnote{On some Periodic Tables, you'll see Hydrogen (H) listed twice, because Hydrogen is weird.}
\item we'll get to the atomic mass later
\item be aware that not all the elements occur naturally \cite[p. 76]{wile-chem-2}, there are more elements now than when I was in school(?!)
\item elements on the \emph{left} of the Periodic Table are metals; elements on the \emph{right} are non-metals, \emph{except Hydrogen (H)}\footnote{I told you Hydrogen is weird.}
\end{itemize}

\section{Compounds}
\begin{definition}[Compound]
A compound is a substance that can be decomposed into elements by chemical means.
\end{definition}

\begin{law}[The Law of Definite Proportions]
The proportion of elements in any compound is always the same.
\end{law}

\begin{itemize}
\item there are basically two types of matter: elements and compounds
\item a compound is a substance made up of elements (\emph{e.g.} water, steel, sulfuric acid)
\item note compounds are made of elements combined \emph{chemically}: bolting a piece of Iron to a piece of Tin doesn't make a compound
\item 

\end{itemize}





\nocite{wile-chem-2}
\bibliography{../Chemistry}{}
\bibliographystyle{apalike}
\end{document}  