\documentclass[11pt,addpoints]{exam}   	% use "amsart" instead of "article" for AMSLaTeX format
\usepackage{geometry}                		% See geometry.pdf to learn the layout options. There are lots.
\geometry{letterpaper}                   		% ... or a4paper or a5paper or ... 
%\geometry{landscape}                		% Activate for rotated page geometry
%\usepackage[parfill]{parskip}    		% Activate to begin paragraphs with an empty line rather than an indent
\usepackage{graphicx}				% Use pdf, png, jpg, or eps§ with pdflatex; use eps in DVI mode
								% TeX will automatically convert eps --> pdf in pdflatex		
\usepackage{amssymb}
\usepackage{mhchem}

\title{Quarter 1 Exam Review}
\author{Mark Peever}
\date{September 25, 2025}							% Activate to display a given date or no date

\begin{document}
\maketitle

\pointsinrightmargin
%\marginpointname{ \points}
\printanswers

\begin{center}
\fbox{\fbox{\parbox{5.5in}{\centering
This is for your benefit! Don't forget to use the correct sig figs!
}}}
\end{center}
\vspace{0.2in}

\begin{questions}
\question What is $25.0^{\circ}C$ in Kelvin?  
\begin{solution}
$298.15 K$
\end{solution}

\question What is the mass of $1.00 L$ of  water?
\begin{solution}
$1.00 kg$
\end{solution} 

\question The density of copper is $8.96 \frac{g}{cm^{3}}$. What is the mass of a $5.00 cm^3$ piece of \ce{Cu} (copper)?
\begin{solution}
 $44.8 g$
\end{solution} 

\question How much thermal energy does it take to raise the temperature of $1.00 L$ of water by $5.0^{\circ}C$? Answer in Joules.
\begin{solution}
 $2.1 \times 10^4 J$
\end{solution} 

\pagebreak
\question How much thermal energy does it take to raise the temperature of $1.00 L$ of water by $5.0^{\circ}C$? Answer in calories.
\begin{solution}
 $5.0 \times 10^3 cal$
\end{solution} 

\question A $1.000 kg$ object has a specific heat of  $0.5990 \frac{J}{g \cdot ^{\circ}C}$.
How much thermal energy does it take to raise its temperature by $5.0^{\circ}C$? 
\begin{solution}
 $3.0 \times 10^3 J$
\end{solution} 

\question How many calories are in 28.49 J?
\begin{solution}
$6.809 cal$
\end{solution}

\question What is $50.00 ^{\circ}C$ in Kelvin?
\begin{solution}
$323.15 K$
\end{solution}

\end{questions}

\pagebreak
\section*{Fun Facts}
\begin{itemize}
\item
$q = m c \Delta T$
\vspace{0.2in}

\item
$\rho = \frac{m}{V} $
\vspace{0.2in}

\item
$T_K = T_C + 273.15 K$
\vspace{0.2in}

\item
$T_F = \frac{9}{5}T_C + 32^{\circ}F$
\vspace{0.2in}

\item
$T_C = \frac{5}{9}(T_F - 32^{\circ}F)$
\vspace{0.2in}

\item
$c_{water} = 1.000 \frac{calorie}{g \cdot ^{\circ}C}$
\vspace{0.2in}

\item
$c_{water} = 4.184 \frac{J}{g \cdot ^{\circ}C}$
\vspace{0.2in}

\item
$c_{copper} = 0.3851 \frac{J}{g \cdot ^{\circ}C}$
\vspace{0.2in}

\item
$\rho_{water} = 1.000 \frac{g}{cm^3}$
\vspace{0.2in}

\item
$\rho_{copper} = 8.96 \frac{g}{cm^3}$
\vspace{0.2in}

\item
$ 1.0000 cm^3 = 1.0000 mL $
\vspace{0.2in}

\item
$ 1.000 cal = 4.184 J $
\end{itemize}
\vspace{0.2in}

\end{document}  