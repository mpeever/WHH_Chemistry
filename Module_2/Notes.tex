\documentclass[11pt, oneside]{article}   	% use "amsart" instead of "article" for AMSLaTeX format
\usepackage{geometry}                		% See geometry.pdf to learn the layout options. There are lots.
\geometry{letterpaper}                   		% ... or a4paper or a5paper or ... 

\usepackage[parfill]{parskip}    		% Activate to begin paragraphs with an empty line rather than an indent
\usepackage{graphicx}				% Use pdf, png, jpg, or eps§ with pdflatex; use eps in DVI mode
								% TeX will automatically convert eps --> pdf in pdflatex		
\usepackage{amssymb}
\usepackage{cite}

% numbered examples
\usepackage{gb4e}
\usepackage{enumitem}
\usepackage{cancel}
\usepackage{amsmath}

% Scientific Laws
\newtheorem{law-thermo-d-1}{The First Law of Thermodynamics}[section]

\title{Module 2: Energy, Heat, and Temperature }
\author{Mark Peever \texttt{mpeever@gmail.com}}

\begin{document}
\maketitle

\begin{center}
2 Peter 3:10--12
\end{center}

\section{Overview}
\begin{enumerate}
\item \textbf{Energy} is the ability to do work
% TODO
\end{enumerate}

\section{Heat, Energy, and Temperature}
\begin{itemize}
\item energy is the ability to do work
\item work done on a system can move it, speed it up, slow it down, or transform it
\begin{enumerate}
\item we generally think of work as the force applied to an object times the distance the object is moved by that force ($W = F \cdot d$)
\item ok, ok, \ldots it's something more like: $W = \vec{F} \cdot \vec{d}$
\item fine! it's  $W = \int \vec{F} \cdot d\vec{x}$, but we're still in high school here \ldots
\end{enumerate}
\item \textbf{Heat} is energy transferred as a result of temperature differences
\item \textbf{Temperature} is the average thermal energy of a system
\end{itemize}


\section{The Nature of Scientific Laws}
\begin{itemize}
\item science is an \emph{inductive} process: it involves observation, measurement, and experimentation
\item in general terms, the scientific process moves through several stages:
\begin{enumerate}
\item observation
\item hypothesis
\item experimentation
\item theory
\item repeated confirmation
\item scientific law
\end{enumerate}
\item in principle, even a scientific law is open to refutation with careful observation, experimentation, and theory
\item it's surprisingly easy to ``walk off the map" in scientific endeavors
\end{itemize}

\section{The First Law of Thermodynamics}

\begin{law-thermo-d-1} \label{law:law-thermo-d-1}
Energy cannot be created or destroyed. It can only change form.
\end{law-thermo-d-1}

\begin{itemize}
\item this is ``The Law of Conservation of Energy"
\item we can move energy from place to place, or from form to form, but we can't actually make more
\item the two most basic forms of energy are:
\begin{enumerate}
\item Potential Energy: energy that is stored
\item Kinetic Energy: energy that is in motion
\end{enumerate}
\item we can think of chemical energy as a form of potential energy: it's energy stored in chemical bonds
\item we can think of energy stored in a battery as potential energy
\item we can think of energy stored in a spring as potential energy
\item heat is a form of kinetic energy: it's energy moving into (or out of) a system
\end{itemize}

\section{Measuring Heat and Energy}
\begin{itemize}
\item we measure mechanical energy in \emph{Joules (J)} in the metric system
\item the explanation of a Joule here is pretty convoluted, let's use: $1J = \frac{1kg \times 1 m^{2}}{1 s^{2}}$
\end{itemize}

\subsection{Temperature Scales}
\begin{itemize}
\item there are four main temperature scales we use:
\begin{enumerate}
\item the Fahrenheit scale ($^{\circ} F$)
\item the Rankine scale
\item the Celsius scale ($^{\circ} C$)
\item the Kelvin scale
\end{enumerate}
\item the last two are what we'll use in the metric system
\item at \emph{standard pressure}, water freezes at $32^{\circ} F$ or $0^{\circ} C$
\item at \emph{standard pressure}, water boils at $212^{\circ} F$ or $100^{\circ} C$ 
\item (at what temperature does water boil in Moscow?)
\end{itemize}




\nocite{wile-chem-2}
\bibliography{../Chemistry}{}
\bibliographystyle{apalike}
\end{document}  