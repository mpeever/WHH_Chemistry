\documentclass[11pt, oneside]{article}   	% use "amsart" instead of "article" for AMSLaTeX format
\usepackage{geometry}                		% See geometry.pdf to learn the layout options. There are lots.
\geometry{letterpaper}                   		% ... or a4paper or a5paper or ... 

\usepackage[parfill]{parskip}    		% Activate to begin paragraphs with an empty line rather than an indent
\usepackage{graphicx}				% Use pdf, png, jpg, or eps§ with pdflatex; use eps in DVI mode
								% TeX will automatically convert eps --> pdf in pdflatex		
\usepackage{amssymb}
\usepackage{cite}

% numbered examples
\usepackage{gb4e}
\usepackage{enumitem}
\usepackage{cancel}
\usepackage{amsmath}

% Scientific Laws
\newtheorem{law}{Law}

\title{Module 2: Energy, Heat, and Temperature }
\author{Mark Peever \texttt{mpeever@gmail.com}}
\date{September 5 -- 12, 2025}

\begin{document}
\maketitle

\section{Overview}
\begin{enumerate}
\item \textbf{Energy} is the ability to do work
\item \textbf{Heat} is thermal energy flowing into a system
\item \textbf{Temperature} is a measure of the average thermal energy of a system
% TODO
\end{enumerate}

\section{Heat, Energy, and Temperature}
\begin{itemize}
\item energy is the ability to do work
\item work done on a system can move it, speed it up, slow it down, or transform it
\begin{enumerate}
\item we generally think of work as the force applied to an object times the distance the object is moved by that force ($W = F \cdot d$)
\footnote{OK, ok, \ldots it's something more like: $W = \vec{F} \cdot \vec{d}$. It would be even more accurate to say something like  $W = \int_{c} \vec{F} \cdot d\vec{x}$, but we're still in high school here.}
\end{enumerate}
\item \textbf{Heat} is energy transferred as a result of temperature differences
\item \textbf{Temperature} is the average thermal energy of a system
\end{itemize}


\section{The Nature of Scientific Laws}
\begin{itemize}
\item science is an \emph{inductive} process: it involves observation, measurement, and experimentation
\item in general terms, the scientific process moves through several stages:
\begin{enumerate}
\item observation
\item hypothesis
\item experimentation
\item theory
\item repeated confirmation
\item scientific law
\end{enumerate}
\item in principle, even a scientific law is open to refutation with careful observation, experimentation, and theory
\item it's surprisingly easy to ``walk off the map" in scientific endeavors
\end{itemize}

\section{The First Law of Thermodynamics}

\begin{law}[The First Law of Thermodynamics] \label{law:law-thermo-d-1}
Energy cannot be created or destroyed. It can only change form.
\end{law}

\begin{itemize}
\item this is ``The Law of Conservation of Energy"
\item we can move energy from place to place, or from form to form, but we can't actually make more
\item the two most basic forms of energy are:
\begin{enumerate}
\item Potential Energy: energy that is stored
\item Kinetic Energy: energy that is in motion
\end{enumerate}
\item we can think of chemical energy as a form of potential energy: it's energy stored in chemical bonds
\item we can think of energy stored in a battery as potential energy
\item we can think of energy stored in a spring as potential energy
\item heat is a form of kinetic energy: it's energy moving into (or out of) a system
\end{itemize}

\section{Measuring Heat and Energy}
\begin{itemize}
\item we measure mechanical energy in \emph{Joules (J)} in the metric system
\item the explanation of a Joule here is pretty convoluted, let's use: $1J = \frac{1kg \times 1 m^{2}}{1 s^{2}}$
\end{itemize}

\subsection{Temperature Scales}
\begin{itemize}
\item there are four main temperature scales we use:
\begin{enumerate}
\item the Fahrenheit scale ($^{\circ} F$)
\item the Rankine scale
\item the Celsius scale ($^{\circ} C$)
\item the Kelvin scale
\end{enumerate}
\item the last two are what we'll use in the metric system
\item at \emph{standard pressure}, water freezes at $32^{\circ} F$ or $0^{\circ} C$
\item at \emph{standard pressure}, water boils at $212^{\circ} F$ or $100^{\circ} C$ 
\item (at what temperature does water boil in Moscow?)
\end{itemize}

\subsubsection{Converting between Fahrenheit and Celsius}
\begin{itemize}
\item we can convert between temperatures in Fahrenheit and temperatures in Celsius with some Algebra I:
\begin{enumerate}
\item hint: this is a \emph{linear} relationship
\item we already have two points, so we can define a line: $(32^{\circ} F, 0^{\circ}C)$ and $(212^{\circ} F, 100^{\circ} C)$
\item we know that $y = mx + b$
\item we know that $m = \frac{y_1 - y_0}{x_1 - x_0}$
\item so let's apply that to our current problem:
\begin{equation} 
\boxed{
\begin{split}
    m      &= \frac{T_{C1} - T_{C0}}{T_{F1} - T_{F0}}  \\
             &= \frac{100^{\circ}C - 0^{\circ}C}{212^{\circ} F - 32^{\circ} F} \\
             &= \frac{100^{\circ}C}{180^{\circ} F} \\
             &= \frac{5^{\circ}C}{9^{\circ} F} \\
 \end{split}
 }
\end{equation}

\item and:
\begin{equation} 
\boxed{
\begin{split}
   T_C   &= m T_F + b \\ 
   T_C - m T_F &= b \\ 
   b &= T_C - m T_F \\
 \end{split}
 }
\end{equation}

\item so:
\begin{equation} 
\boxed{
\begin{split}
   b &= T_{C0} - ( m ) ( T_{F0}) \\
      &= ( 0^{\circ}C ) - (\frac{5^{\circ}C}{9^{\circ} F} ) ( 32^{\circ} F ) \\
      &= - \frac{(32^{\circ} F)(5^{\circ}C)}{9^{\circ} F} \\
      &= - \frac{160}{9} ^{\circ} C  \\
 \end{split}
 }
\end{equation}
\item which gives us: $T_C = (\frac{5^{\circ}C}{9^{\circ} F}) T_F -  \frac{160}{9} ^{\circ} C $
\end{enumerate}
\item and that's super ugly\ldots we can just use: $T_C = (\frac{5}{9})(T_F - 32^{\circ} F)$ \cite[p. 45]{wile-chem-2}
\item and please, \emph{please} don't conflate units of measure with variables, like they do in the book!\footnote{In a just society, the penalty of this would make our ears tingle to hear of it. 
I cannot be made to believe it won't come up in the Final Judgment on the Last Day.}
\item conversely, we can convert Celsius measurements to Fahrenheit with: $T_F = (\frac{9}{5}) T_C + 32^{\circ} F$  \cite[p. 46]{wile-chem-2}
\end{itemize}

\subsubsection{Kelvin and Rankine Scales}
\begin{itemize}
\item Kelvin and Rankine scales are \emph{absolute} temperature scales: there are no negative temperatures
\item to convert between Celsius and Kelvin scales, just add 273.15 K: $T_K = T_C + 273.15 K $ \cite[p. 46]{wile-chem-2}
\item similarly, to convert between Fahrenheit and Rankine scales, just add $459.67 ^{\circ} R$: $T_R = T_F + 459.67 ^{\circ} R$
\item Rankine is \emph{not} in our book, but I want you to be aware there is an analog to Kelvin for the English system
\end{itemize}

\subsection{Examples}
\begin{enumerate}[label=Example \arabic*]
\item The highest temperature Mr. Peever can remember from when he was living in St. Louis is  $116^{\circ} F$. What is that temperature in the Celsius scale?
\begin{equation} 
\boxed{
\begin{split}
    T_C &= (\frac{5^{\circ} C}{9^{\circ} F}) (T_F - 32^{\circ} F) \\
            &= (\frac{5^{\circ} C}{9^{\circ} F}) (116^{\circ} F - 32^{\circ} F) \\
            &= (\frac{5^{\circ} C}{9^{\circ} F}) (84^{\circ} F) \\
            &= (\frac{5^{\circ} C}{9\xcancel{^{\circ} F}}) (84\xcancel{^{\circ} F}) \\
            &= \frac{420^{\circ} C}{9} \\
            &= 46.666667^{\circ} C \\
            &= 46.7^{\circ} C \\
 \end{split}
 }
 \end{equation}

 \item The lowest temperature Mr. Peever can remember from when he was living in Canada is  $-45^{\circ} C$. What is that temperature in the Fahrenheit scale?
\begin{equation} 
\boxed{
\begin{split}
    T_F  &= (\frac{9^{\circ} F}{5^{\circ} C}) T_C + 32^{\circ} F \\
            &=  (\frac{9^{\circ} F}{5^{\circ} C}) (-45 ^{\circ} C) + 32^{\circ} F \\
            &= (\frac{9^{\circ} F}{5\xcancel{^{\circ} C}}) (-45\xcancel{^{\circ} C})  + 32^{\circ} F \\
            &= (\frac{9^{\circ} F}{1}) (-9)  + 32^{\circ} F \\
            &= -81^{\circ} F + 32^{\circ} F \\
            &= -49^{\circ} F \\
 \end{split}
 }
 \end{equation}
 
  \item What is Mr. Peever's lowest temperature in the Kelvin scale?
\begin{equation} 
\boxed{
\begin{split}
    T_K  &= T_C + 273.15 K \\
             &= -45 ^{\circ} C +  273.15 K \\
             &= (-45 ^{\circ} C)(\frac{1 K}{1 ^{\circ}C}) +  273.15 K \\
             &= (-45 \xcancel{^{\circ} C})(\frac{1 K}{1 \xcancel{^{\circ}C}}) +  273.15 K \\
             &= -45 K +  273.15 K \\
             &= 228.15 K \\
 \end{split}
 }
 \end{equation}
 
 \end{enumerate}

\section{Calorie, calories, and kilocalories}
\begin{itemize}
\item Heat is energy, but\ldots we generally measure heat in \emph{calories}\footnote{Because of course we do.}
\item 1 calorie is the amount of heat necessary to raise the temperature of $1g$ of water by $1^{\circ} C$
\item since both calories and Joules measure energy, we ought to be able to convert between them: $1 calorie = 4.184 Joules$ \cite[p. 48]{wile-chem-2}
\item in the world of nutrition, we measure energy in \emph{kilocalories} ($1 kcal = 1000 cal$)
\item oddly, we generally write kilocalories as ``Calories": $1 Cal = 1 kcal = 1000 cal$)
\end{itemize}

\section{Measuring Heat}
\begin{itemize}
\item remember that heat is thermal energy moving in (or out) of a system
\item we can calculate heat as: $q = m c \Delta T$
\item generally, the Delta symbol($\Delta$) indicates a change, so $\Delta T$ is a change in temperature: $\Delta T = T_{final} - T_{initial}$
\item (note: we could also write that as $\Delta T = T_2 - T_1$, or even as $\Delta T = T_1 - T_0$)\footnote{This last is the Physics way to Chemistry.}
\item so back to our heat equation\ldots
\begin{enumerate}
\item $q$ is heat
\item $m$ is the \emph{mass} of our object (generally in grams)
\item $c$ is the \emph{specific heat} of our object (generally in $\frac{J}{g \cdot ^{\circ}C}$)
\item $\Delta T$ is the change in temperature of our object (generally in $^{\circ}C$, but $K$ would work too)
\end{enumerate}
\end{itemize}

\subsection{Examples}
\begin{enumerate}[label=Example \arabic*]
\item What is the specific heat of water?
\begin{equation} 
\boxed{
\begin{split}
    q &= m c \Delta T  \\
    \frac{q}{c} &= m \Delta T \\
    \frac{1}{c} &= \frac{m \Delta T}{q} \\
    (\frac{1}{c}) ^{-1} &= (\frac{m \Delta T}{q})^{-1} \\
    c &= \frac{q}{m \Delta T} \\    
       &= \frac{1 cal} {1g \cdot 1^{\circ} C} \\
       &= \frac{1 cal} {1g \cdot 1^{\circ} C} \\
       &= 1 \frac{cal} {g \cdot ^{\circ} C} \\
 \end{split}
 }
 \end{equation}
 \item What is the specific heat of water in Joules?
\begin{equation} 
\boxed{
\begin{split}
      c &= 1 \frac{cal} {g \cdot ^{\circ} C} \\
         &= 1 (\frac{cal} {g \cdot ^{\circ} C})(\frac{4.184 J}{1 cal})\\
         &= 1 (\frac{\xcancel{cal}} {g \cdot ^{\circ} C})(\frac{4.184 J}{1 \xcancel{cal}})\\
         &= 1 (\frac{4.184 J} {g \cdot ^{\circ} C})\\ 
         &= 4.184 \frac{ J} {g \cdot ^{\circ} C}\\     
 \end{split}
 }
 \end{equation}

 \end{enumerate}

\pagebreak

\section{Homework}
Review Problems: p. 67 \# 1--10 (not to be turned in)\\
Practice Problems: p. 68 \# 1--10 (due 2025-09-19)\\
Experiment 2.1, pp. 43--45 (due 2025-09-12)\\

\pagebreak
\subsection{Answers to Review Problems}
Answers to problems \#1--10, page 67

\begin{enumerate}
\item
\begin{enumerate}
\item The man is \emph{not} doing work the entire 2 minutes, because there is no motion for the first 1.5 minutes.
\item Yes, the man eventually succeeds in doing work.
\item The man does work in the final 0.5 minutes, when the car is moving.
\end{enumerate}

\item
\begin{enumerate}
\item a lump of coal has potential energy
\item a flash of lightning has kinetic energy 
\item a candle flame has kinetic energy
\item a tornado has kinetic energy
\end{enumerate}

\item The chemist can conclude that the liquid in the vat was hotter (\emph{i.e.} had higher thermal energy) than its surroundings: the liquid's temperature dropped as it lost heat to its surroundings.

\item No, a totally isolated object could not transfer thermal energy as heat to anything else, so its temperature would remain constant.

\item Science is not the ultimate source of truth, because scientific laws are subject to change and/or refinement as we make more and better observations (see p. 39).

\item If the two objects are identical, and they start at the same temperature, then the one that absorbs more heat will end up hotter. $ 1 cal = 4.184 J $ (see p. 48), so $ 100.0 cal = 418.4 J $, since $ 418.4 J > 100.0 J $, the first object has a higher final temperature.  

\item Drinking ice-cold water causes your body to produce more thermal energy to maintain its internal temperature. This energy is produced by speeding up your metabolism, consuming more stored energy.

\item The substance boils where the graph of Temperature vs. Time flattens at the upper end. On this graph, that's at 15 minutes. The temperature at the flattened part of the graph is $55^{\circ}C$, so this substance boils at a temperature of  $55^{\circ}C$.

\item The specific heat of iron ($c_{Fe}$) is roughly three times that of gold ($c_{Au}$). So we can calculate:
\begin{equation} 
\boxed{
\begin{split}
      q &= m \cdot c \cdot \Delta T \\
      c &= \frac{q}{m \cdot \Delta T} 
 \end{split}
 }
 \end{equation}
 
 \begin{equation} 
\boxed{
\begin{split}
      c_{Fe} &= 3 \cdot c_{Au} \\
      \frac{q_{Fe}}{m_{Fe} \cdot \Delta T_{Fe}}  &= 3 \cdot \frac{q_{Au}}{m_{Au} \cdot \Delta T_{Au}} \\
 \end{split}
 }
 \end{equation}
 But we know $q_{Fe} = q_{Au}$ and $m_{Fe} = m_{Au}$, so:\\
 
  \begin{equation} 
\boxed{
\begin{split}
      \frac{\cancel{q_{Fe}}}{\cancel{m_{Fe}} \cdot \Delta T_{Fe}}  &= 3 \cdot \frac{\cancel{q_{Au}}}{\cancel{m_{Au}} \cdot \Delta T_{Au}} \\
      \frac{1}{\Delta T_{Fe}} &= 3 \cdot \frac{1}{\Delta T_{Au}} \\
      \Delta T_{Fe} &= \frac{\Delta T_{Au}}{3} \\
                            &= \frac{900^{\circ}C}{3} \\
                            &= 300^{\circ}C
 \end{split}
 }
 \end{equation}
 
 \item A calorimeter measures thermal energy by transferring thermal energy from an object into water and measuring the temperature change of the water (p. 54). It needs to be made of an insulated material so that the thermal energy transfer from the object into the water is the only heat in the system. If the calorimeter were not made of an insulated material, the heat loss from the water into its surroundings would be enough to cause uncertainty about the total heat exchange.
 
\end{enumerate}

\pagebreak
\subsection{Answers to Practice Problems}
Answers to problems \#1--10, page 68

\begin{enumerate}
\item $15.0^{\circ}C = 59.0^{\circ}F$
\item 
\begin{enumerate}
\item $3.5 K = -270^{\circ}C$
\item $3.5 K = -450 ^{\circ}F$
\end{enumerate}
\item $115^{\circ}F = 46.1^{\circ}C$
\item $10460000 J$ (or $1.046 \times 10^{7} J$)
\item $38000 J$
\item $11.1 \frac{J}{g \cdot ^{\circ}C}$
\item $-3.0 \times 10^{\circ}C$
\item $12600 cal $ or $12.6 kcal$
\item $0.96 \frac{J}{g \cdot ^{\circ}C}$ or $ 0.23 \frac{cal}{g \cdot ^{\circ}C}$
\item $32000 \frac{J}{g \cdot ^{\circ}C}$ or $7600 \frac{cal}{g \cdot ^{\circ}C}$
\end{enumerate}


\nocite{wile-chem-2}
\bibliography{../Chemistry}{}
\bibliographystyle{apalike}
\end{document}  