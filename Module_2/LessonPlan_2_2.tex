\documentclass[10pt, oneside]{article}   	% use "amsart" instead of "article" for AMSLaTeX format
\usepackage{geometry}                		% See geometry.pdf to learn the layout options. There are lots.
\geometry{letterpaper}                   		% ... or a4paper or a5paper or ... 
%\geometry{landscape}                		% Activate for rotated page geometry
%\usepackage[parfill]{parskip}    		% Activate to begin paragraphs with an empty line rather than an indent
\usepackage{graphicx}				% Use pdf, png, jpg, or eps§ with pdflatex; use eps in DVI mode
								% TeX will automatically convert eps --> pdf in pdflatex		
\usepackage{amssymb}

\usepackage{hyperref}

\title{Module 2, Part 2}
\author{Mark Peever\\ \texttt{mpeever@gmail.com}}
\date{September 12, 2025}

\begin{document}
\maketitle

\section*{Objectives}
\marginpar{0 minutes}
Refer to \href{https://drive.google.com/file/d/1lXFrIbob_UusJJQ4y8gLQMyNPjURZ2zt/view?usp=drive_link}{Module 2 Notes.pdf}.\\

By the end of this class, the students should be able to\ldots
\begin{itemize}
\item state and explain the First Law of Thermodynamics
\item differentiate between Energy, Heat, and Temperature
\item convert temperatures between Fahrenheit, Celsius, and Kelvin scales
\end{itemize}

\section*{Welcome \& Devotion}
\marginpar{5 minutes}
\begin{itemize}
\item have one student read \href{https://tinyurl.com/26k8csmf}{Acts 17:22–32}
\end{itemize}

\section*{Problem Problems from Last Week's Homework}
\marginpar{20 minutes}
\begin{itemize}
\item work out \#5, 9, and 10 from last week's homework on the whiteboard
\end{itemize}

\section*{Energy, Heat, and Temperature}
\marginpar{30 minutes}
\begin{itemize}
\item review the First Law of Thermodynamics from last week
\item \emph{Energy} is the ability to do work
\item \emph{Heat} is thermal energy in motion
\item \emph{Temperature} is the average thermal energy in a substance
\end{itemize}

\section{Temperature Scales and Conversions}
\marginpar{20 minutes}
\begin{itemize}
\item go over Fahrenheit scale
\item go over Celsius scale
\item go over Kelvin scale
\item go through simple temperature conversions
\end{itemize}

\section*{Questions}

\section*{Assignment}
\marginpar{5 minutes}
\begin{itemize}
\item Problems p. 67 \# 1--10
\item Problems p. 68 \# 1--10 (turn in next week: 2025-09-19)
\end{itemize}



\end{document}  