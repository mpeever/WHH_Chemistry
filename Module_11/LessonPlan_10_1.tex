\documentclass[10pt, oneside]{article}   	% use "amsart" instead of "article" for AMSLaTeX format
\usepackage{geometry}                		% See geometry.pdf to learn the layout options. There are lots.
\geometry{letterpaper}                   		% ... or a4paper or a5paper or ... 
%\geometry{landscape}                		% Activate for rotated page geometry
%\usepackage[parfill]{parskip}    		% Activate to begin paragraphs with an empty line rather than an indent
\usepackage{graphicx}				% Use pdf, png, jpg, or eps§ with pdflatex; use eps in DVI mode
								% TeX will automatically convert eps --> pdf in pdflatex		
\usepackage{amssymb}
\usepackage{mhchem}
\usepackage{hyperref}

\title{Module 10, Part 1}
\author{Mark Peever\\ \texttt{mpeever@gmail.com}}
\date{February 06, 2026}

\begin{document}
\maketitle

\section*{Objectives}
\marginpar{0 minutes}
Refer to \href{https://drive.google.com/file/d/18y4OeHpdaHSJHFDW2GUUf8dNE7SdLXWM/view?usp=drive_link}{Module 10 Notes}.\\

By the end of this class, the students should be able to\ldots
\begin{itemize}
\item describe electrolytes
\item describe acids, bases, and salts
\end{itemize}

\section*{Welcome \& Devotion}
\marginpar{5 minutes}
\begin{itemize}
\item have one student read \href{https://www.biblegateway.com/passage/?search=Proverbs\%2025\%3A20\&version=LSB}{Proverbs 25:20}
\end{itemize}

\section*{Electrolytes}
\marginpar{10 minutes}
\begin{itemize}
\item \textbf{Electrolytes} give ions in aqueous solutions
\item \textbf{acids}, \textbf{bases}, and \textbf{salts} are all electrolytes
\end{itemize}

\section*{Acids}
\marginpar{15 minutes}
\begin{itemize}
\item \textbf{acids} are covalent electrolytes that:
\begin{enumerate}
\item taste sour
\item turn litmus paper red
\item donate a proton (\ce{H^{+}})
\end{enumerate}
\item show \ce{H2SO4 (aq) + 2H2O (l) -> 2H3O^{+} (aq) + SO4^{2-} (aq)}
\end{itemize}

\section*{Bases}
\marginpar{15 minutes}
\begin{itemize}
\item \textbf{bases} are electrolytes that:
\begin{enumerate}
\item taste bitter
\item feel slippery
\item turn litmus paper blue
\item accept a proton (\ce{H^{+}})
\end{enumerate}
\item show \ce{NH3 (aq) + H2O (aq) -> NH4^{+} (aq) + OH^{-} (aq)}
\end{itemize}

\section*{Salts}
\marginpar{15 minutes}
\begin{itemize}
\item \textbf{bases} are electrolytes that are electrically balanced
\item they're almost always (maybe always) ionic compounds
\item \emph{e.g.} \ce{MgSO4}
\item \emph{e.g.} \ce{NaCl}
\end{itemize}

\section*{Acid/Base Reactions}
\marginpar{20 minutes}
\begin{itemize}
\item show ionization with \ce{H3O^{+}}
\item show ionization with \ce{OH^{-}}
\item show covalent base reaction
\item \ce{Acid + Base -> Salt + Water}
\item \emph{e.g.} \ce{MgSO4}
\item \emph{e.g.} \ce{NaCl}
\end{itemize}

\section*{Questions for me}
\marginpar{5 minutes}

\section*{Announcements}
\begin{itemize}
\item go over plans for lab next week
\item remember quiz next week
\end{itemize}

%\section*{Assignment}
%\marginpar{5 minutes}
%\begin{itemize}
%\item Review Problems: p. 317 \# 1--10 (not to be turned in)
%\item Practice Problems: p. 318 \# 1--10 (due 2026-02-06)
%\item Experiment 9.1 (due 2026-02-06)
%\end{itemize}


\end{document}  