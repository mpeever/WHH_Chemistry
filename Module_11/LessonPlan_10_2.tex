\documentclass[10pt, oneside]{article}   	% use "amsart" instead of "article" for AMSLaTeX format
\usepackage{geometry}                		% See geometry.pdf to learn the layout options. There are lots.
\geometry{letterpaper}                   		% ... or a4paper or a5paper or ... 
%\geometry{landscape}                		% Activate for rotated page geometry
%\usepackage[parfill]{parskip}    		% Activate to begin paragraphs with an empty line rather than an indent
\usepackage{graphicx}				% Use pdf, png, jpg, or eps§ with pdflatex; use eps in DVI mode
								% TeX will automatically convert eps --> pdf in pdflatex		
\usepackage{amssymb}
\usepackage{mhchem}
\usepackage{hyperref}

\title{Module 9}
\author{Mark Peever\\ \texttt{mpeever@gmail.com}}
\date{January 30, 2026}

\begin{document}
\maketitle

\section*{Objectives}
\marginpar{0 minutes}
Refer to \href{https://drive.google.com/file/d/12srN0NMLpdzRnF2pxvVdx8_-AGgPTe_f/view?usp=sharing}{Module 9 Notes}.\\

By the end of this class, the students should be able to\ldots
\begin{itemize}
\item describe polyatomic ions
\item enumerate several ``favorite" polyatomic ions
\item describe VSEPR Theory
\item describe molecule shape and explain how it affects molecular parity
\end{itemize}

\section*{Welcome \& Devotion}
\marginpar{5 minutes}
\begin{itemize}
\item have one student read \href{https://www.biblegateway.com/passage/?search=Colossians\%201\&version=LSB}{Colossians 1:15--20}
\end{itemize}

\section*{Module 8 Quiz}
\marginpar{20 minutes}
\begin{itemize}
\item have students complete the Module 8 Quiz
\item go over the Quiz correct answers and take in the papers
\end{itemize}




\section*{Polyatomic Ions}
\marginpar{20 minutes}
\begin{itemize}
\item \textbf{Polyatomic Ions} are ionized molecules
\item they can form ionic bonds just like ionized atoms can
\item example: \ce{Na2SO4}
\item example: \ce{CaCO3}
\item example: \ce{Mg(NO3)2}
\item example: \ce{NH4Cl}
\item example: \ce{(NH4)2CO3}
\item it's worth learning a few ``favorites" from Table 9.1 on p. 288:
\begin{enumerate}
\item \ce{NH4^{+}}
\item \ce{NO3^{-}}
\item \ce{SO4^{2-}}
\item \ce{OH^{-}}
\end{enumerate}
\end{itemize}

\section*{Molecular Shape}
\marginpar{15 minutes}
\begin{itemize}
\item summarize VSEPR Theory: electron bond pairs tend to repel each other
\item show shapes:
\begin{enumerate}
\item tetrahedral (\emph{e.g.} \ce{CH4})
\item pyramidal (\emph{e.g.} \ce{NH3})
\item bent (\emph{e.g.} \ce{H2O})
\item trigonal (\emph{e.g.} \ce{CH2O} --- the dual bond on the \ce{CO} pair repels the \ce{CH} bonds ) 
\item linear (\emph{e.g.} \ce{CO2} --- the dual bonds repel and keep straight)
\end{enumerate}
\end{itemize}

\section*{Polar Covalent Bonds}
\marginpar{5 minutes}
\begin{itemize}
\item show how \ce{HCl} forms a dipole
\item how how \ce{H2} is a purely covalent bond
\end{itemize}

\section*{Polar Covalent Molecules}
\marginpar{5 minutes}
\begin{itemize}
\item show how \ce{CH4} is a purely covalent molecule
\item show how \ce{H2O} is a polar covalent molecule
\end{itemize}

\section*{Questions for me}
\marginpar{5 minutes}

\section*{Assignment}
\marginpar{5 minutes}
\begin{itemize}
\item Review Problems: p. 317 \# 1--10 (not to be turned in)
\item Practice Problems: p. 318 \# 1--10 (due 2026-02-06)
\item Experiment 9.1 (due 2026-02-06)
\end{itemize}



\end{document}  