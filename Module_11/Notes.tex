\documentclass[11pt, oneside]{article}   	% use "amsart" instead of "article" for AMSLaTeX format
\usepackage{geometry}                		% See geometry.pdf to learn the layout options. There are lots.
\geometry{letterpaper}                   		% ... or a4paper or a5paper or ... 

\usepackage[parfill]{parskip}    		% Activate to begin paragraphs with an empty line rather than an indent
\usepackage{graphicx}				% Use pdf, png, jpg, or eps§ with pdflatex; use eps in DVI mode
								% TeX will automatically convert eps --> pdf in pdflatex		
\usepackage{amssymb}
\usepackage{cite}

% numbered examples
\usepackage{gb4e}
\usepackage{enumitem}
\usepackage{cancel}
\usepackage{amsmath}

\usepackage{mhchem}
\usepackage{lewis}
%\usepackage{urlbst}
\usepackage{url}
\usepackage{hyperref}


% Scientific Laws
\newtheorem{definition}{Definition}
\newtheorem{law}{Law}
\newtheorem{theory}{Theory}
\newtheorem{hint}{Hint}

\title{Module 11: The Chemistry of Solutions}
\author{Mark Peever\\ \texttt{mpeever@gmail.com}}
\date{February 20 -- 27, 2026}

\begin{document}
\maketitle

\begin{center}

\end{center}

\section{Overview}
\begin{enumerate}
\item A \textbf{solution} is a homogenous mixture.
\item A solution consists of one or more \textbf{solutes} dissolved into a \textbf{solvent}
\item We have talked mainly about \emph{aqueous} solutions so far, but there are many solvents other than water.
\end{enumerate}

\section{How Solutes Dissolve in Solvents}
\subsection{How Solids Dissolve}
\begin{itemize}
\item remember\footnote{We've discussed this at length in class, but we'll discuss some more if you don't remember.} 
how \emph{aqueous} solutions work with ionic compounds:
\footnote{The book is careful to point out that not all ionic compounds are soluble in water \cite[p. 355]{wile-chem-2}.}
\begin{itemize}
\item in aqueous solutions, the solvent is water (\ce{H2O})
\item ionic compounds break into positive ions (cations) and negative ions (anions) in water
\item \emph{e.g.} \ce{NaCl (aq) -> Na^{+} (aq) + Cl^{-} (aq)}
\item \ce{H2O} is polar covalent: the $-2 \delta$ charge at the \ce{O} and the $+1 \delta$ charge at each \ce{H} attract oppositely-charged ions
\item  so\ldots 
\item the \ce{Na^{+}} ions tend to gather at the \ce{O} side of the water molecules
\item the \ce{Cl^{-}} ions tend to gather around the \ce{H} side of the water molecules
\item as a result, the \ce{NaCl} seems to disapper into the water
\footnote{My initial wording was, ``the \ce{NaCl} appears to disappear\ldots" which is sort of awkward.}
\end{itemize}

\item a polar covalent solute does something similar (not quite the same):
\footnote{Remember polar covalent molecules \emph{in general} don't break into ions in aqueous solution. Electrolytes are an exception.}
\begin{itemize}
\item if we put \ce{PH3} in \ce{H2O}, each \ce{PH3} molecule will tend to attract opposite ends
\item like in water, there is a $-3 \delta$ charge around the \ce{P} end of the molecule
\item and there is a $+1 \delta$ charge at each \ce{H} atom
\item so\ldots 
\item the \ce{P} end of the \ce{PH3} molecules will tend to attract the \ce{H} end of the \ce{H2O} molecules
\item the \ce{H} end of the \ce{PH3} molecules will tend to attract the \ce{O} end of the \ce{H2O} molecules
\end{itemize}

\item a substance that cannot be dissolved into another substance is said to be \textbf{insoluble} in it
\footnote{Notice solubility --- hence all \emph{insolubility} --- is a result of a \emph{pair} of substances. Think about it, it'll make sense.}
\end{itemize}

\subsection{How Liquids and Gases Dissolve}
\begin{itemize}
\item in principle, there mechanism is the same for dissolving gases and liquids as solids, \emph{but}\ldots
\item liquids take less attraction to dissolve than solids do: the solute is less ``held together'' when it's a liquid
\item gases take less attraction to dissolve even than liquids do: there's very little structure at all holding them together
\footnote{Take note of this reaction: \ce{CO2 (g) + H2O (l) <=> H2CO3 (aq)} (see \cite[p. 353]{wile-chem-2}). 
Notice dissolving \ce{CO2} in \ce{H2O} involves a \emph{chemical} change as well as a \emph{physical} one.} 
\end{itemize}

\section{Solubility}

\begin{definition}[Solubility]\label{definition:solubility}
The maximum amount of a solute that can dissolve in a given amount of solvent.
\end{definition}

\begin{hint}\label{hint:solubility}
The solubility of any solute depends on both the identity of the solute and the identity of the solvent.
\end{hint}

\begin{definition}[Saturated Solution]\label{definition:saturated-solution}
A solution in which the maximum amount of solute has been dissolved.
\end{definition}

\begin{itemize}
\item solubility measures how much (and, indirectly) how easily a solute dissolves into a given solvent
\item notice solubility is a function of both the solute and the solvent: something highly soluble in water might be insoluble in oil
\item solubility is generally measured in ``grams per hundred," or how many grams of solute can dissolve into $100.0 g$ of solvent
\item when you dissolve the maximum amount of solute into a solvent, you have a \textbf{saturated solution} (see Definition \ref{definition:saturated-solution}, p. \pageref{definition:saturated-solution})
\footnote{There is such a thing as a \emph{super-saturated solution}, where it's possible to get more solute into a solvent than it can hold.}
\item the solubility of a solute can also be a function of things like temperature and pressure (see Hint \ref{hint:solubility-rules-temperature} , p. \pageref{hint:solubility-rules-temperature} and Hint \ref{hint:solubility-rules-pressure}, p. \pageref{hint:solubility-rules-pressure}.)
\footnote{This is why soda goes flat if you leave the lid off, and it goes flat faster if it's warm.}
\end{itemize}

\begin{hint}\label{hint:solubility-rules-temperature}
For solid solutes, solubility usually increases with increasing temperatures.\\
The solubility of liquid solutes is not affected by temperature.\\
The solubility of gases decreases with increasing temperatures.
\end{hint}

\begin{hint}\label{hint:solubility-rules-pressure}
Increasing pressure increases the solubility of gases.\\
Pressure does not affect solubility of either liquids or solids.
\end{hint}

\section{Energy Changes that Occur when Making a Solution}

\begin{definition}[Exothermic Process]\label{definition:process-exothermic}
A process that releases heat.
\end{definition}

\begin{definition}[Endothermic Process]\label{definition:process-endothermic}
A process that absorbs heat.
\end{definition}

\begin{itemize}
\item we've classified dissolving (\emph{dissolution}) as a physical process, but it gets a bit more complicated than that
\item some solutes trigger chemical reactions when they dissolve (\emph{e.g.} \ce{HCl + H2O (l) -> H3O^{+} (aq) + Cl^{-} (aq)})
\item these chemical reactions can be \emph{exothermic} (see Definition \ref{definition:process-exothermic}, p. \pageref{definition:process-exothermic})
\item some dissolutions are exothermic on their own, like dissolving strong bases (\emph{e.g.} \ce{NaOH  -> Na^{+} (aq) + OH^{-} (aq)}
\item most dissolutions are \emph{endothermic} (see Definition \ref{definition:process-endothermic}, p. \pageref{definition:process-endothermic}) 
\end{itemize}

\section{Applying Stoichiometry to Solutions}

\begin{hint}\label{hint:stoichiometry-in-solutions}
Stoichiometry in solutions is no different than stoichiometry in general, except that mole calculations are often based on concentration.
\end{hint}

\begin{definition}[Concentration]\label{definition:concentration}
The concentration of a substance in solution is the amount of that substance in a given volume:
$$ [ \hspace{0.1 in} ] = \frac{n}{V} $$	
\end{definition}

\begin{definition}[Molarity]\label{definition:concentration-molarity}
\textbf{Molarity} is the number of moles of a substance per liter of solution:
$$ 1 M = \frac{1 mole}{1 liter} $$
\end{definition}


\section{Molality}

\begin{definition}[Molality]\label{definition:concentration-molality}
\textbf{Molality} is the number of moles of a substance per kilogram of solution:
$$ 1 m = \frac{1 mole}{1 kg} $$
\end{definition}

\begin{itemize}
\item in Module 10, we discussed concentration in terms of molarity (See Definitions 7 \& 8 in Module 10 notes)
\item we can also measure concentration in \textbf{molality}, which is the number of moles of a solute divided by the mass of the solvent.
\item so we don't overload the equations, we'll write the formula for molality as:\\
$$ b = \frac{n}{m}$$
\item we'll abbreviate ``Molar" as $M$ and ``Molal" as $m$
\end{itemize}

\section{Freezing Point Depression}
\begin{itemize}
\item in general, a solution has a lower freezing point than the solvent alone
\item this is why we salt roads in winter 
\item a substance freezes when it loses enough energy that the molecules (or atoms) start to stay in place relative to each other
\item a solution has more than one kind of molecule (or atom), so it needs to lose enough energy for both the solute(s) and the solvent to solidify
\item attraction between solute and solvent particles add energy to the whole
\item we can calculate freezing point depression as:\\
$$ \Delta T = -i \cdot K_{f} \cdot m$$
\item in this equation\ldots
\begin{itemize}
\item $i$ is how many ions or molecules the solute splits into when dissolved
\item $K_{f}$ is a constant: the ``freezing point depression constant"
\item $m$ is the molality of the solution 
\end{itemize}
\end{itemize}

\section{Boiling Point Elevation}
\begin{itemize}
\item solutions tend to have higher boiling points than their solvents
\item in the case of boiling, the addition of solute molecule attraction means it takes more energy to pry solvent molecules (or atoms) into a gas
\item boiling point elevation can be calculated as:\\
$$\Delta T = i \cdot K_{b} \cdot m$$
\item in this equation\ldots
\begin{itemize}
\item $i$ is how many ions or molecules the solute splits into when dissolved
\item $K_{b}$ is a constant: the ``boiling point elevation constant"
\item $m$ is the molality of the solution 
\end{itemize}
\end{itemize}


\section{Homework}
Review Problems: p. 381 \# 1--10 (not to be turned in)\\
Practice Problems: p. 382 \# 1--10 (due 2026-03-06)\\

\nocite{wile-chem-2}
\bibliography{../Chemistry}
\bibliographystyle{apalike}

%\clearpage
\subsection{Answers to Practice Problems}
Answers to Practice Problems p. 382 \# 1--10:
\begin{enumerate}
\item The water in the hot tub gets cloudy if the temperature falls, because as the temperature of the hot tub decreases, the solubility of the solutes decreases, and they come out of solution.
\item Gas is less soluble in water when it's warm than when it's cold.
\item $73.8 g$
\item $0.21 L$
\item $5.6 \times 10^{-2} M$
\item$15 m$
\item $0.266 m$
\item $4.0 m$
\item $-11.0 ^{\circ}C$
\item $101.5 ^{\circ}C$
\end{enumerate}
\end{document}  