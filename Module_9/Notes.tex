\documentclass[11pt, oneside]{article}   	% use "amsart" instead of "article" for AMSLaTeX format
\usepackage{geometry}                		% See geometry.pdf to learn the layout options. There are lots.
\geometry{letterpaper}                   		% ... or a4paper or a5paper or ... 

\usepackage[parfill]{parskip}    		% Activate to begin paragraphs with an empty line rather than an indent
\usepackage{graphicx}				% Use pdf, png, jpg, or eps§ with pdflatex; use eps in DVI mode
								% TeX will automatically convert eps --> pdf in pdflatex		
\usepackage{amssymb}
\usepackage{cite}

% numbered examples
\usepackage{gb4e}
\usepackage{enumitem}
\usepackage{cancel}
\usepackage{amsmath}

\usepackage{mhchem}
\usepackage{lewis}
%\usepackage{urlbst}
\usepackage{url}
\usepackage{hyperref}


% Scientific Laws
\newtheorem{definition}{Definition}
\newtheorem{law}{Law}
\newtheorem{theory}{Theory}
\newtheorem{hint}{Hint}

\title{Module 9: Polyatomic Ions and Geometry }
\author{Mark Peever\\ \texttt{mpeever@gmail.com}}
\date{January 30, 2025}

\begin{document}
\maketitle

\begin{center}

\end{center}

\section{Overview}
\begin{enumerate}
\item \textbf{Polyatomic Ions} are ions the contain a covalent bond
\item \textbf{VSEPR Theory} predicts molecular shape: molecules attain whatever shape maximizes the distances between valence electrons on the central atom
\footnote{I have a theory about motor scooters that sounds a whole lot like ``VSEPR Theory," but it's not the same thing.}
\item \textbf{Molecular Shape} affects chemical properties of covalent compounds
\end{enumerate}

\section{Polyatomic Ions}
\begin{definition}[Polyatomic Ions]\label{definition:polyatomic-ions}
Ions which are formed when a group of atoms gains or loses electrons.
\end{definition}

\begin{itemize}
\item it is possible for a group of atoms to form a molecule with a charge
\item that molecule is then both a molecule \emph{and} an ion
\item \emph{e.g.} \ce{OH^{-}} (hydroxide) looks like:  \lewis{O}{.}{.}{.}{.}{}{}{.}{.}\hspace{-0.5em} -- \hspace{-0.75em}\lewis{H}{}{}{}{}{}{}{}{}
\item the hydroxide molecule can then form an ionic compound with a positive ion, \emph{e.g.} \ce{NaOH} or \ce{Mg(OH)2} 
\footnote{This is the general form of a \emph{base}: bases are ionic compounds formed with hydroxide.}
\item \emph{e.g.} \ce{H3O^{+}} (hydronium) looks like:  
\lewis{H}{}{}{}{}{}{}{}{}\hspace{-0.5em}--\hspace{-0.75em} \lewis{O}{}{}{}{
\hspace{-1.0em} \lewis{H}{}{}{}{}{}{}{|}{}
}{.}{.}{
\hspace{-1.0em} \lewis{H}{}{}{|}{}{}{}{}{}
}{}

\item there is a table of common polyatomic ions (Table \ref{table:polyatomic-ions}, p. \pageref{table:polyatomic-ions})  reproduced from the text \cite[p. 288]{wile-chem-2}
\end{itemize}

\begin{table}
\centering
\begin{tabular}[p]{| l | l |}
\textbf{Ion Name} & \textbf{Formula} \\
\hline
ammonium  & \ce{NH4+} \\
hydroxide    & \ce{OH-} \\
chlorate       & \ce{ClO3-} \\
chlorite        & \ce{ClO2-} \\
nitrate          & \ce{NO3-} \\
nitrite           & \ce{NO2-} \\
acetate        & \ce{C2H3O2-} \\
cyanide       & \ce{CN-} \\
carbonate   & \ce{CO3^{2-}} \\
chromate    & \ce{CrO4^{2-}} \\
dichromate & \ce{Cr2O7^{2-}}\\
sulfate        & \ce{SO4^{2-}} \\
sulfite         & \ce{SO3^{2-}} \\
phosphate  & \ce{PO4^{3-}} 
\end{tabular}
\caption{Polyatomic Ions}
\label{table:polyatomic-ions}
\end{table}

\begin{hint}\label{hint:table:polyatomic-ions}
It's worth learning  (Table \ref{table:polyatomic-ions}, p. \pageref{table:polyatomic-ions}).
We'll be seeing these ions a lot, and we need to recognize them like the old friends they will become.
\end{hint}	


\section{Molecular Shape and VSEPR}
\begin{itemize}
\item \textbf{V}alence \textbf{S}hell \textbf{E}lectron \textbf{P}air \textbf{R}epulsion (VSEPR) Theory says that each pair of bonded electrons will repel other bonded electron pairs
\item this means that molecules generally form in such a way that the distance between molecular bonds is maximized
\item so, for example, we draw a \ce{CH4} molecule like this: 
\lewis{H}{}{}{}{}{}{}{}{}\hspace{-0.5em}--\hspace{-0.75em}\lewis{C}{}{}{}{
\lewis{H}{}{}{}{}{}{}{|}{}
}
{}{}{
\lewis{H}{}{}{|}{}{}{}{}{}
}{}\hspace{-0.5em}--\hspace{-0.75em}\lewis{H}{}{}{}{}{}{}{}{}

\item but it actually looks more like Figure \ref{figure:tetrahedral-ch4}, p. \pageref{figure:tetrahedral-ch4} 
\footnote{My artistic talent is such that my drawing might require some explanation: there is one \ce{H}  atom directly above the \ce{C} atom, while the other three are spaced in a triangle beneath the \ce{C} atom. There is one coming out of the page toward you, one behind the page, and one off more-or-less to the right.}
(see \cite[p. 290]{wile-chem-2})
\begin{figure}[hp]
\centering
\includegraphics[scale=0.25]{ch4}
\caption{Tetrahedral \ce{CH4} Molecule}
 \label{figure:tetrahedral-ch4}
 \end{figure}  
\item the idea is that each bond is attempting to get away from the others, and they all find the optimal distance from ``all other bonds" by forming a tetrahedron
\item unbonded electrons can also affect where bonds end up, as in the case of \ce{NH3}, see Figure 9.2 in your text \cite[p. 292]{wile-chem-2}
\item notice there the unbonded pair on the \ce{N} atom has a less pronounced effect on the bonds than another bond would
\item this even affects water, see Figure 9.3 in your text \cite[p. 293]{wile-chem-2}
\item notice, too, in the case of double bonds, they still attempt to repel each other, ``bowing out" from each other \cite[pp. 294--295]{wile-chem-2} 
\end{itemize}	

\section{Purely Covalent and Polar Covalent Bonds}

\begin{definition}{Purely Covalent Bond}\label{definition:purely-covalent-bond}
A covalent bond where electrons are shared equally between the atoms involved.
\end{definition}

\begin{definition}{Polar Covalent Bond}\label{definition:polar-covalent-bond}
A covalent bond where electrons are shared unequally between the atoms involved.
\end{definition}

\begin{itemize}
\item we're not going to spend a lot of time on molecular shape, but the idea of a ``polar covalent bond" is interesting
\item the idea here is that some covalent bonds form a ``polar" shape, with a negative side and a positive side
\item notice this isn't the same thing as an ionic bond: the atoms aren't ionizing
\footnote{You might catch a mistake I made in class on p. 299 \cite{wile-chem-2}.}
\item but it's possible for an electrically neutral molecule to more electrons on one side, and more protons on the other
\item this leads to a ``polar" molecule which has a net neutral charge, but still has a ``more positive" and a ``more negative" side
\end{itemize}

\begin{hint}[Determining Whether a Covalent Bond is Polar]\label{hint:polar-covalent-bonds}
If a covalent bond exists between two identical atoms, it is purely covalent.
Otherwise, it is polar.
\end{hint}	

\section{Purely Covalent and Polar Covalent Molecules}
\begin{itemize}
\item a molecule that contains only purely covalent bonds is a purely covalent molecule (see Definition \ref{definition:purely-covalent-bond})
\item but a molecule might also be purely covalent if it contains multiple polar covalent bonds that cancel each other out (see Definition \ref{definition:polar-covalent-bond})
\item however, if a molecule contains polar covalent bonds \emph{and} if the bonds pull unequally, the molecule itself will be polar
\item for example, \ce{CH4} contains four polar bonds, but they all cancel each other out, so \ce{CH4} is a purely covalent molecule
\item for example, \ce{H2O} contains two polar bonds, but they \emph{don't} cancel each other out, so \ce{H2O} is a polar covalent molecule
\end{itemize}

\begin{hint}[Polar Covalent Bonds and Solubility]\label{hint:polar-covalent-bonds-and-solubility}
Polar covalent compounds can dissolve other polar covalent compounds \emph{and} ionic compounds.
Purely covalent compounds can dissolve \emph{only} purely covalent compounds.
\end{hint}	

\section{Homework}
Review Problems: p. 317 \# 1--10 (not to be turned in)\\
Practice Problems: p. 318 \# 1--10 (due 2026-02-06)\\

\nocite{wile-chem-2}
\bibliography{../Chemistry}
\bibliographystyle{apalike}

%\clearpage
\subsection{Answers to Practice Problems}
Answers to Practice Problems p. 218 \# 1--10:
\begin{enumerate}
\item 
\begin{enumerate}
\item \ce{K2SO4}
\item \ce{Ca(NO3)2}
\item \ce{MgCO3}
\item \ce{Al2(CrO4)3}
\end{enumerate}
\item 
\begin{enumerate}
\item ammonium oxide
\item potassium nitrite
\item calcium phosphate
\item aluminum phosphate
\end{enumerate}
\item \ce{Ca(NO3)2 (aq) + Na2CO3 (aq) -> CaCO3 (s) + 2NaNO3 (aq)} 
\item \ce{PCl3} is pyramidal
\item \ce{H2} is linear
\item \ce{SiCl4} is tetrahedral
\item \ce{H2S} is bent
\item \ce{CS2} is linear
\item 
\begin{enumerate}
\item \ce{MgCl2} is ionic
\item \ce{CF3Cl} is polar covalent
\item \ce{CS2} is purely covalent
\item \ce{H2} is purely covalent
\item \ce{SiCl4} is purely covalent
\item \ce{PCl3} is polar covalent
\end{enumerate}
\item 
\begin{enumerate}
\item \ce{MgCl2} is soluble in water
\item \ce{CF3Cl} is soluble in water
\item \ce{CS2} is not soluble in water
\item \ce{H2} is not soluble in water
\item \ce{SiCl4} is not soluble in water
\item \ce{PCl3} is soluble in water
\end{enumerate}
\end{enumerate}

\end{document}  