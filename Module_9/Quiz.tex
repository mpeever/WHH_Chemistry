\documentclass[11pt,addpoints]{exam}   	% use "amsart" instead of "article" for AMSLaTeX format
\usepackage{geometry}                		% See geometry.pdf to learn the layout options. There are lots.
\geometry{letterpaper}                   		% ... or a4paper or a5paper or ... 
%\geometry{landscape}                		% Activate for rotated page geometry
%\usepackage[parfill]{parskip}    		% Activate to begin paragraphs with an empty line rather than an indent
\usepackage{graphicx}				% Use pdf, png, jpg, or eps§ with pdflatex; use eps in DVI mode
								% TeX will automatically convert eps --> pdf in pdflatex		
\usepackage{amssymb}
\usepackage{mhchem}
\usepackage{lewis}


\title{Module 8 Quiz}
\author{Mark Peever}
\date{January 30, 2026}							% Activate to display a given date or no date

\begin{document}
\maketitle

\pointsinrightmargin
%\marginpointname{ \points}

%\begin{center}
%\fbox{\fbox{\parbox{5.5in}{\centering
%$q = m c \Delta T$ \\
%$\rho = \frac{m}{V} $
%}}}
%\end{center}
\vspace{0.1in}
\makebox[\textwidth]{Name:\enspace\hrulefill}
\vspace{0.2in}

\begin{questions}
\question[5] The Lewis structure for \ce{Mg} looks like this: \lewis{Mg}{}{}{}{}{.}{.}{}{}\\
Based on the Lewis structure, how many valence electrons does \ce{Mg} have?
 \begin{choices}
\choice 1
\choice 2
\choice 3
\choice 4
\end{choices}

\question[5] In its base state, the Lewis structure for \ce{H} looks like this: \lewis{H}{}{}{}{}{.}{}{}{}\\
Which of the following is the correct Lewis structure for \ce{H2}?
 \begin{choices}
\choice \lewis{H}{}{}{}{}{}{}{}{}\hspace{-0.5em}--\hspace{-0.75em}\lewis{H}{}{}{}{}{}{}{}{}
\choice \lewis{H}{.}{.}{}{}{}{}{}{}\hspace{-0.5em}--\hspace{-0.75em}\lewis{H}{}{}{}{}{.}{.}{}{}
\choice \lewis{\ce{H^{2+}}}{.}{.}{}{}{}{}{}{} \hspace{-0.5em}--\hspace{-0.75em} \lewis{\ce{H^{2+}}}{}{}{}{}{.}{.}{}{}
\choice \lewis{H}{}{}{}{}{.}{}{}{} \lewis{H}{.}{}{}{}{}{}{}{}
\end{choices}

\question[5] The Lewis structure for methane looks like:\\
\lewis{H}{}{}{}{}{}{}{}{}\hspace{-0.5em}--\hspace{-0.75em}\lewis{C}{}{}{}{
\lewis{H}{}{}{}{}{}{}{|}{}
}
{}{}{
\lewis{H}{}{}{|}{}{}{}{}{}
}{}\hspace{-0.5em}--\hspace{-0.75em}\lewis{H}{}{}{}{}{}{}{}{}

What is the correct molecular formula for methane, based on its Lewis structure?
 \begin{choices}
\choice \ce{C4H} 
\choice \ce{CH4} 
\choice \ce{C4H4} 
\choice \ce{CH4^{2+}} 
\end{choices}

\question[5] The Lewis structure for Lithium ion like this: \lewis{\ce{Li^{+}}}{}{}{}{}{}{}{}{}\\
Based on the Lewis structure, how many valence electrons does a Liithium atom (\emph{i.e.} a \emph{non}-ionized Lithium atom) have?

 \begin{choices}
\choice 1
\choice 2
\choice 3
\choice 4
\end{choices}

\clearpage
\question[5] In its base state, the Lewis structure for \ce{O} is \lewis{O}{.}{}{.}{}{.}{.}{.}{.}\\
Which of the following is the correct Lewis structure for \ce{O2}?

 \begin{choices}
\choice \lewis{O}{}{}{}{}{}{}{}{}\hspace{-0.5em}--\hspace{-0.75em}\lewis{O}{}{}{}{}{}{}{}{}
\choice \lewis{O}{.}{.}{.}{.}{}{}{.}{.}\hspace{-0.5em} -- \hspace{-0.75em}\lewis{O}{}{}{.}{.}{.}{.}{.}{.}
\choice \lewis{O}{}{}{.}{.}{}{}{.}{.}\hspace{-0.5em} = \hspace{-0.75em}\lewis{O}{}{}{.}{.}{}{}{.}{.}
\choice \lewis{O}{.}{.}{.}{.}{}{}{.}{.}\hspace{-0.5em} = \hspace{-0.75em}\lewis{O}{}{}{.}{.}{.}{.}{.}{.}
\end{choices}


\question[5] In its base state, the Lewis structure for \ce{Cl} is \lewis{Cl}{.}{.}{.}{}{.}{.}{.}{.}\\
Which of the following is the correct Lewis structure for \ce{Cl2}?

 \begin{choices}
\choice \lewis{Cl}{}{}{}{}{}{}{}{}\hspace{-0.5em}--\hspace{-0.75em}\lewis{Cl}{}{}{}{}{}{}{}{}
\choice \lewis{Cl}{.}{.}{.}{.}{}{}{.}{.}\hspace{-0.5em} -- \hspace{-0.75em}\lewis{Cl}{}{}{.}{.}{.}{.}{.}{.}
\choice \lewis{Cl}{}{}{.}{.}{}{}{.}{.}\hspace{-0.5em} = \hspace{-0.75em}\lewis{Cl}{}{}{.}{.}{}{}{.}{.}
\choice \lewis{Cl}{.}{.}{.}{.}{}{}{.}{.}\hspace{-0.5em} = \hspace{-0.75em}\lewis{Cl}{}{}{.}{.}{.}{.}{.}{.}
\end{choices}

\end{questions}

\begin{center}
This exam has \numquestions\ questions for a total of \numpoints\ points.
\end{center}

\end{document}  