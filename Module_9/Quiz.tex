\documentclass[11pt]{article}   	% use "amsart" instead of "article" for AMSLaTeX format
\usepackage{geometry}                		% See geometry.pdf to learn the layout options. There are lots.
\geometry{letterpaper}                   		% ... or a4paper or a5paper or ... 
%\geometry{landscape}                		% Activate for rotated page geometry
%\usepackage[parfill]{parskip}    		% Activate to begin paragraphs with an empty line rather than an indent
\usepackage{graphicx}				% Use pdf, png, jpg, or eps§ with pdflatex; use eps in DVI mode
								% TeX will automatically convert eps --> pdf in pdflatex		
\usepackage{amssymb}
\usepackage{mhchem}
\usepackage{lewis}


\title{Module 9 Quiz}
\author{Mark Peever}
\date{February 13, 2026}

\begin{document}
\maketitle

\begin{table}[h]
\centering
\begin{tabular}[p]{| l | l | l |}
\hline
 & \textbf{Ion Name} & \textbf{Formula} \\

\hline

    & \hspace {2in} & \hspace {2in} \\
1. & \hspace {2in} & \hspace {2in} \\
\hline

    & \hspace {2in} & \hspace {2in} \\
2. & \hspace {2in} & \hspace {2in} \\
\hline

    & \hspace {2in} & \hspace {2in} \\
3. & \hspace {2in} & \hspace {2in} \\
\hline

    & \hspace {2in} & \hspace {2in} \\
4. & \hspace {2in} & \hspace {2in} \\
\hline

    & \hspace {2in} & \hspace {2in} \\
5. & \hspace {2in} & \hspace {2in} \\
\hline

    & \hspace {2in} & \hspace {2in} \\
6. & \hspace {2in} & \hspace {2in} \\
\hline

    & \hspace {2in} & \hspace {2in} \\
7. & \hspace {2in} & \hspace {2in} \\
\hline

    & \hspace {2in} & \hspace {2in} \\
8. & \hspace {2in} & \hspace {2in} \\
\hline

    & \hspace {2in} & \hspace {2in} \\
9. & \hspace {2in} & \hspace {2in} \\
\hline

    & \hspace {2in} & \hspace {2in} \\
10. & \hspace {2in} & \hspace {2in} \\
\hline

    & \hspace {2in} & \hspace {2in} \\
11. & \hspace {2in} & \hspace {2in} \\
\hline

    & \hspace {2in} & \hspace {2in} \\
12. & \hspace {2in} & \hspace {2in} \\
\hline

    & \hspace {2in} & \hspace {2in} \\
13. & \hspace {2in} & \hspace {2in} \\
\hline

    & \hspace {2in} & \hspace {2in} \\
14. & \hspace {2in} & \hspace {2in} \\
\hline

%    & \hspace {2in} & \hspace {2in} \\
%15. & \hspace {2in} & \hspace {2in} \\
%\hline

\end{tabular}
\end{table}



\begin{table}[h]
\centering
\begin{tabular}[p]{| l | l |}
\textbf{Ion Name} & \textbf{Formula} \\
\hline
ammonium  & \ce{NH4+} \\
hydronium  & \ce{H3O+} \\
hydroxide    & \ce{OH-} \\
chlorate       & \ce{ClO3-} \\
chlorite        & \ce{ClO2-} \\
nitrate          & \ce{NO3-} \\
nitrite           & \ce{NO2-} \\
acetate        & \ce{C2H3O2-} \\
cyanide       & \ce{CN-} \\
carbonate   & \ce{CO3^{2-}} \\
chromate    & \ce{CrO4^{2-}} \\
dichromate & \ce{Cr2O7^{2-}}\\
sulfate        & \ce{SO4^{2-}} \\
sulfite         & \ce{SO3^{2-}} \\
phosphate  & \ce{PO4^{3-}} 
\end{tabular}
\caption{Polyatomic Ions}
\label{table:polyatomic-ions}
\end{table}







\end{document}  