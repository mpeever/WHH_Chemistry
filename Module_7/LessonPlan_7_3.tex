\documentclass[10pt, oneside]{article}   	% use "amsart" instead of "article" for AMSLaTeX format
\usepackage{geometry}                		% See geometry.pdf to learn the layout options. There are lots.
\geometry{letterpaper}                   		% ... or a4paper or a5paper or ... 
%\geometry{landscape}                		% Activate for rotated page geometry
%\usepackage[parfill]{parskip}    		% Activate to begin paragraphs with an empty line rather than an indent
\usepackage{graphicx}				% Use pdf, png, jpg, or eps§ with pdflatex; use eps in DVI mode
								% TeX will automatically convert eps --> pdf in pdflatex		
\usepackage{amssymb}
\usepackage{mhchem}
\usepackage{hyperref}

\title{Module 7, Part 3}
\author{Mark Peever\\ \texttt{mpeever@gmail.com}}
\date{December 12, 2025}

\begin{document}
\maketitle

\section*{Objectives}
\marginpar{0 minutes}
Refer to \href{https://drive.google.com/file/d/16133VKM35EouvPPKMvsYA3UIiagBbDy7/view?usp=sharing}{Module 7 Notes}.\\

By the end of this class, the students should be able to\ldots
\begin{itemize}
\item give a brief description of quantum physics
\item describe Bohr's model of the atom in simple terms
\item describe the Quantum model of the atom in simple terms
\item write electron configurations for elements in the Periodic Table
\item identify electron configurations for elements based on their location in the Periodic Table
\end{itemize}

\section*{Welcome \& Devotion}
\marginpar{5 minutes}
\begin{itemize}
\item have one student read \href{https://www.biblegateway.com/passage/?search=hebrews\%201\%3A1-4\&version=LSB}{Hebrews 1:1--4}
\end{itemize}

%\section*{Review Module 6 Quiz}
%\marginpar{30 minutes}
%Refer to the Module 6 Quiz key.
%
%\section*{Electromagnetic Radiation and Light (Reprise)}
%\marginpar{10 minutes}
%\begin{itemize}
%\item $ v = \lambda \times \nu $
%\item touch on the speed of light
%\item introduce Planck's Hypothesis: $ E = h \times \nu $
%\item briefly introduce wave-particle duality
%\end{itemize}
%
%\section*{Subatomic Particles}
%\marginpar{15 minutes}
%\begin{itemize}
%\item introduce electrons
%\item introduce protons
%\item introduce neutrons
%\end{itemize}
%
%\section*{Atomic Models}
%\marginpar{20 minutes}
%\begin{itemize}
%\item introduce Thompson's plum pudding
%\item introduce Rutherford's planetary model
%\end{itemize}

\section*{Quantum Physics Overview}
\marginpar{15 minutes}
\begin{itemize}
\item energy is \emph{quantized} into discreet packets
\item  \emph{everything} has a touch of wave/particle duality
\item briefly introduce uncertainty
\end{itemize}

\section*{The Bohr Model of the Atom}
\marginpar{20 minutes}
\begin{itemize}
\item introduce fixed orbits
\item describe the electron excitement theory
\end{itemize}

\section*{The Quantum Model of the Atom}
\marginpar{40 minutes}
\begin{itemize}
\item introduce orbitals: s, p, d, f
\item introduce electron configuration notation
\item cover the Periodic Table in terms of orbitals
\end{itemize}

\section*{Questions for me}
\marginpar{10 minutes}

\section*{Assignment}
\marginpar{5 minutes}
\begin{itemize}
\item Review Problems: p. 245 \# 1--10 (not to be turned in)
\item Practice Problems: p. 246 \# 1--10 (due 2025-12-19)
\item Experiment 7.1
\end{itemize}


\end{document}  