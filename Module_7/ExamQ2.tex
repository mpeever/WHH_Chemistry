\documentclass[11pt,addpoints]{exam}   	% use "amsart" instead of "article" for AMSLaTeX format
\usepackage{geometry}                		% See geometry.pdf to learn the layout options. There are lots.
\geometry{letterpaper}                   		% ... or a4paper or a5paper or ... 
%\geometry{landscape}                		% Activate for rotated page geometry
%\usepackage[parfill]{parskip}    		% Activate to begin paragraphs with an empty line rather than an indent
\usepackage{graphicx}				% Use pdf, png, jpg, or eps§ with pdflatex; use eps in DVI mode
								% TeX will automatically convert eps --> pdf in pdflatex		
\usepackage{amsmath}
\usepackage{cancel}
\usepackage{amssymb}
\usepackage{multicol}
\usepackage{mhchem}

\title{Quarter 2 Exam}
\author{Mark Peever}
\date{December 12, 2025}							% Activate to display a given date or no date

\begin{document}
\maketitle

\pointsinrightmargin
%\marginpointname{ \points}
%\printanswers

\begin{center}
\fbox{\fbox{\parbox{5.5in}{\centering
Answer all questions on a separate sheet of paper. Show all your work.
}}}
\end{center}
\vspace{0.1in}
\makebox[\textwidth]{Name:\enspace\hrulefill}
\vspace{0.2in}

\begin{questions}

\question[5] Jack has a tank of methane gas (\ce{CH4}). He wants to measure out $0.500$ ($5.00 \times 10^{-1}$) moles of methane by mass, so he needs to know the molar mass of methane.\\
What is the molar mass of methane (\ce{CH4})?\\

\begin{solution}
\begin{equation} 
\begin{split}
    M_{C} &= 12.0107 \frac{g}{mol} \\
    M_{H} &= 1.00794 \frac{g}{mol} \\
    M_{\ce{CH4}} &= M_{C} + 4 M_{H} \\
                          &= (12.0107 \frac{g}{mol}) + 4 (1.00794 \frac{g}{mol}) \\
                          &= 12.0107 \frac{g}{mol} + 4.03176 \frac{g}{mol} \\
                          &= 16.04246  \frac{g}{mol} \\
                          &= 16.0425  \frac{g}{mol}
 \end{split}
 \end{equation}
 \end{solution}

\question[2] Am\'elie knows that methane is a \emph{covalent compound}, because it's made up of two non-metals.
Am\'elie doesn't like the name ``methane," though. (She doesn't even like to say ``water," preferring to call it ``dihydrogen monoxide.")\\
What other name could she call \ce{CH4}?\\

\begin{solution}
Am\'elie  could call it ``monocarbon tetrahydride'' or just ``carbon tetrahydride''. 
 \end{solution}


\question[5] Jack fills a balloon with $0.500$ moles of methane gas (\ce{CH4})  and gives the balloon to Ella. How many \emph{grams} of methane gas does Ella have in her balloon?\\
(Remember to use the correct significant figures!)\\

\begin{solution}
\begin{equation} 
\begin{split}
M_{\ce{CH4}} &= 16.0425  \frac{g}{mol}\\
    m &= n \times M \\
	&= (0.500 mol) ( 16.0425  \frac{g}{mol} ) \\
	&= (0.500 \xcancel{mol}) ( 16.0425  \frac{g}{\xcancel{mol}} ) \\
	&= 8.02125 g \\
	&= 8.02 g \\
 \end{split}
 \end{equation}
 \end{solution}


\question[5] Ella wants to burn her methane gas. She remembers that it's possible for methane to burn incompletely, and Avery reminds her that the
unbalanced chemical reaction for incomplete combustion of methane requires methane and oxygen and produces carbon monoxide and water. \\ 
Avery wouldn't ever do something so \emph{gauche} as using an unbalanced chemical equation, so she balances it first. 
What does the balanced chemical equation look like?\\

\begin{solution}
\ce{2CH4 + 3O2 -> 2CO + 4H2O}
\end{solution}


\question[5] Ella wants to burn her methane \emph{completely}, so she decides to burn it outside.
She hangs it on a tree limb and Nate touches it with a cattle prod (\emph{BOOM!}). 
Ella and Nate know that the methane has reacted with the oxygen in the air to produce water and carbon dioxide.\\
Write the chemical reaction that occurred. Be sure your chemical equation is balanced!
\begin{solution}
\ce{CH4 + 2O2 -> CO2 + 2H2O}
\end{solution}

\question[5] Vianne is curious how much \ce{O2} was consumed in the reaction, so she does a very quick calculation to figure it out.
How many moles of \ce{O2} were consumed?

\begin{solution}
There are $2$ moles of \ce{O2} consumed for every mole of \ce{CH4} in our reaction: \ce{CH4 + 2O2 -> CO2 + 2H2O}\\

\begin{equation} 
\begin{split}
    \frac{n_{\ce{O2}}}{2} &= \frac{n_{\ce{CH4}}}{1} \\
    n_{\ce{O2}} &= 2 n_{\ce{CH4}} \\
                       &= 2 (0.500 mol ) \\
                       &= 1.000 mol \\
                       &= 1.000 mol
 \end{split}
 \end{equation}
\end{solution}


\question[5] No\'emi, like Vianne, wants to know how much \ce{O2} was consumed, but she's more interested in the oxygen's \emph{mass}. How many \emph{grams} of \ce{O2} were consumed?

\begin{solution}
\begin{equation} 
\begin{split}
    M_{\ce{O2}} &= 2 (15.9994 \frac{g}{mol}) \\
                        &= 31.9988 \frac{g}{mol} \\                        
    m &= n \times M \\
        &= (1.000 mol ) (31.9988 \frac{g}{mol}) \\
        &= (1.000 \xcancel{mol} ) (31.9988 \frac{g}{\xcancel{mol}}) \\
        &= 31.9988 g \\
        &= 32.00 g
 \end{split}
 \end{equation}
 \end{solution}


\question[5] Ezra isn't that interested in mass or in moles, but he's curious how many balloons worth of \ce{CO2} gas was produced in the reaction.
If Ella had all her methane gas in a single balloon, how many \emph{balloons} would it take to contain all the \ce{CO2}? 

\begin{solution}
By the Gay-Lussac Law, the volume ratio of \ce{CH4} to \ce{CO2} is $1:1$.\\
Therefore, there is $1$ balloon worth of \ce{CO2} produced.
\end{solution}


\question[5] Caroline is curious how much water vapor is produced in the \emph{BOOM!} reaction. How many moles of water vapor (\ce{H2O}) were produced?
\begin{solution}
There are $2$ moles of \ce{O2} produced for every mole of \ce{CH4} in our reaction: \ce{CH4 + 2O2 -> CO2 + 2H2O}\\

\begin{equation} 
\begin{split}
    \frac{n_{\ce{O2}}}{2} &= \frac{n_{\ce{CH4}}}{1} \\
    n_{\ce{O2}} &= 2 n_{\ce{CH4}} \\
                       &= 2 (0.500 mol ) \\
                       &= 1.000 mol \\
                       &= 1.000 mol
 \end{split}
 \end{equation}
\end{solution}



\question[5] Gideon asks Caroline to calculate the number of \emph{grams} of water vapor produced when Ella and Nate blew up the balloon. How many grams of \ce{H2O} were produced?\\

\begin{solution}
\begin{equation} 
\begin{split}
    M_{\ce{H2O}} &= 2 (1.00794 \frac{g}{mol}) + (15.9994 \frac{g}{mol}) \\
                           &= (2.01588 \frac{g}{mol}) + (15.9994 \frac{g}{mol}) \\
                           &= 18.01528 \frac{g}{mol} \\      
                           &= 18.0153 \frac{g}{mol} \\              
    m &= n \times M \\
        &= (1.000 mol ) (18.0153 \frac{g}{mol}) \\
        &= (1.000 \xcancel{mol} ) (18.0153 \frac{g}{\xcancel{mol}}) \\
        &= 18.0153 g \\
        &= 18.02 g
 \end{split}
 \end{equation}
 \end{solution}


\question[3] Thomas is pretty sure there was an \emph{excess} of oxygen in that \emph{BOOM!} reaction, and that the methane was the \emph{limiting} reactant.
Is Thomas correct? How do you know?

\begin{solution}
Thomas generally is right. An outdoor combustion has virtually limitless \ce{O2} to consume, so \ce{CH4} is the limiting reactant.
\end{solution}



\bonusquestion \emph{Bonus Question} 
The book of Genesis closes with, ``he was put in a coffin in Egypt" (Genesis 50:26).
Of course, Joseph isn't the first person to be buried in the book of Genesis.
Who is the first dead person buried in Scripture?
\begin{solution}
\begin{itemize}
\item Sarah
\end{itemize}
\end{solution}

\end{questions}

\begin{center}
This exam has \numquestions\ questions for a total of \numpoints\ points and 1 bonus points.
\end{center}

\pagebreak
\section*{Fun Facts}
\begin{multicols}{2}
\begin{itemize}
\item
$q = m c \Delta T$
\vspace{0.2in}

\item
$\Delta T = T_{final} - T_{initial}$
\vspace{0.2in}

\item
$\rho = \frac{m}{V} $
\vspace{0.2in}

\item
$T_K = T_C + 273.15 K$
\vspace{0.2in}

\item
$T_F = \frac{9}{5}T_C + 32^{\circ}F$
\vspace{0.2in}

\item
$T_C = \frac{5}{9}(T_F - 32^{\circ}F)$
\vspace{0.2in}

\item
$c_{water} = 1.000 \frac{calorie}{g \cdot ^{\circ}C}$
\vspace{0.2in}

\item
$c_{water} = 4.184 \frac{J}{g \cdot ^{\circ}C}$
\vspace{0.2in}

\item
$c_{copper} = 0.3851 \frac{J}{g \cdot ^{\circ}C}$
\vspace{0.2in}

\item
$\rho_{water} = 1.000 \frac{g}{cm^3}$
\vspace{0.2in}

\item
$\rho_{copper} = 8.96 \frac{g}{cm^3}$
\vspace{0.2in}

\item
$ 1.0000 cm^3 = 1.0000 mL $
\vspace{0.2in}

\item
$ 1.000 cal = 4.184 J $
\vspace{0.2in}

\item
$m = n \times M$
\vspace{0.2in}

\item
$m = n \times m_{molar}$
\vspace{0.2in}

\item
$1 amu = 1 \frac{g}{mol}$ 
\vspace{0.2in}

\item
$N_A = 6.02214076 \times 10^{23} $ 
\vspace{0.2in}

\end{itemize}
\vspace{0.2in}

\end{multicols}
\end{document}  