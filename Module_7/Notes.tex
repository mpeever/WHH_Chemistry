\documentclass[11pt, oneside]{article}   	% use "amsart" instead of "article" for AMSLaTeX format
\usepackage{geometry}                		% See geometry.pdf to learn the layout options. There are lots.
\geometry{letterpaper}                   		% ... or a4paper or a5paper or ... 

\usepackage[parfill]{parskip}    		% Activate to begin paragraphs with an empty line rather than an indent
\usepackage{graphicx}				% Use pdf, png, jpg, or eps§ with pdflatex; use eps in DVI mode
								% TeX will automatically convert eps --> pdf in pdflatex		
\usepackage{amssymb}
\usepackage{cite}

% numbered examples
\usepackage{gb4e}
\usepackage{enumitem}
\usepackage{cancel}
\usepackage{amsmath}

\usepackage{mhchem}

\usepackage{hyperref}

% Scientific Laws
\newtheorem{definition}{Definition}
\newtheorem{law}{Law}
\newtheorem{theory}{Theory}
\newtheorem{hint}{Hint}

\title{Module 7: Atomic Structure }
\author{Mark Peever\\ \texttt{mpeever@gmail.com}}
\date{November 21 -- December 5, 2025}

\begin{document}
\maketitle

\begin{center}

\end{center}

\section{Overview}
\begin{enumerate}
\item \textbf{Electrical Charge} is controlled by an atom's  \textbf{Protons}  and  \textbf{Electrons}  
\item an atom's \textbf{Nucleus} is made up of \textbf{Protons} and \textbf{Neutrons}
\item \textbf{Light} is a form of electromagnetic energy, and it proves to be much more important than we realize
\item it turns out the \textbf{Frequency} of electromagnetic radiation determines how much energy it carries
\item the \textbf{Bohr Model} of the atom gives us a good mental model for how atoms work, but it has limitations
\item the \textbf{Quantum Model} of the atom is based somewhat on the Bohr Model, but is much more complex
\item with the Quantum Model of the atom, we can build \textbf{electron configurations} that reflect the atom's position in the Periodic Table
\end{enumerate}

\section{Subatomic Particles}
\begin{itemize}
\item it turns out Dalton wasn't quite right about atoms: \emph{atom} means ``indivisible," but it turns out atoms are actually divisible
\item William Crookes (1832--1919) demonstrated that  a battery connected to a tube filled with gas would cause a glow in at the positively-charged end
\item Crookes experimented, and concluded that the glow at the end of the tube was a result of a stream of particles
\item J. J. Thompson (1856--1940) demonstrated that the behavior in the ``Crookes' Tube''\footnote{We now call this a ``cathode ray tube." This is the basic technology inside old-school televisions and computer monitors, with a whole lot of improvements and refinements.} didn't depend on the type of gas in the tube
\item Thompson concluded that the stream of particles in the cathode ray tube was actually a small bit of the atoms in the gas that broke away from the atoms
\item because they seemed to be that same particles regardless of the type of gas used, he concluded they were basic building blocks of the atoms
\item Thompson called the particles \textbf{electrons}, and demonstrated they have a \emph{negative electrical charge}
\item Thompson came up with the so-called plum-pudding model of the atom, where the atom is a mass of positive charge, with electrons embedded in it\cite[p. 209]{wile-chem-2}
\end{itemize}

\section{Side Quest: Electrical Charge}
\begin{itemize}
\item electrical charge is intimately tied into our understanding of the atom, but we need some background on what electrical charge is
\item electrical charges are classified as \emph{positive} or \emph{negative}\footnote{These aren't value judgments, we just use them to categorize charge, and it turns out the math works out nicely. Don't get too hung up on them: negative charges aren't worse than positive charges, they're just charged oppositely.}
\item opposite charges attract each other: a negatively-charged object will be attracted towards a positively-charged object
\item like charges repel each other: two positively-charged objects will tend to push each other apart, two negatively-charged objects will do the same
\item if an object has no charge, or if it has an equal number of positive and negative charges, then it is \emph{neutral}, or has a $0$ charge
\item it's important to note that we really don't understand what electrical charge \emph{is}, but we have a decent idea how it behaves 
\end{itemize}

\section{Atomic Structure and Electrical Charge}
\begin{itemize}
\item it turns out atoms contain three types of particles, each with their own charge:
\begin{itemize}
\item \emph{electrons} have a negative charge
\item \emph{protons} have a positive charge
\item \emph{neutrons} have a neutral charge
\end{itemize}
\item electrons are the smallest of the three in terms of both mass and volume by a \emph{huge} amount (see Table \ref{table:particles:mass-and-charge}, p. \pageref{table:particles:mass-and-charge})
\item let's revisit the Periodic Table with our new knowledge:
\begin{itemize}
\item the Atomic Number in our Periodic Table is actually the count of protons in an atom
\item in theory, the number of electrons and protons is the same, but that's not generally the case\footnote{This is actually where chemical bonds come from, but we'll get more into that later.}
\item the Atomic Mass of an element in the Periodic Table is the sum of the masses of protons and neutrons (which are pretty close in mass)
\item electrons are small enough they don't really affect the atomic mass\footnote{They're not even a rounding error on the mass of a proton!}
\item you'd \emph{think} the atomic masses would all be whole numbers, but they're not because\ldots
\begin{enumerate}
\item protons and neutrons don't have \emph{exactly} the same mass (see Table \ref{table:particles:mass-and-charge}, p. \pageref{table:particles:mass-and-charge})
\item atomic masses are \emph{averages}: there are variations (``isotopes") of elements   
\end{enumerate}
\end{itemize}
\item it's important to note that an element's identity is determined by its Atomic Number, which is the number of protons it has
\item if three atoms have the same number of protons, but differing numbers of neutrons, then they're the same element, even if they have different masses
\item variants of mass within a single element are called ``isotopes"
\item isotopes behave the same as far as we're concerned (at the chemical reaction level), although there can be huge differences on the nuclear level\footnote{There are less stable and more stable isotopes for many elements: \ce{^{14}C}, for example ``decays" over time, where \ce{^{12}C} is stable.}
\end{itemize}


\begin{table}[hp]
\centering
\begin{tabular}[b]{l | l | l | l }
\hline
Name & Charge & Charge (in Coulombs) & Mass \\
\hline
electron  & $-1 e$   & $-1.602176634 \times 10^{-19} C $  &  $9.1093837139 \times 10^{-28} g$ \\ 
proton    & $1 e$     & $1.602176634 \times  10^{-19} C $  &  $1.67262192595 \times 10^{-24} g$ ($1 amu$) \\
neutron  & $0 e$     & $0 C $                                               &  $1.67492750056 \times 10^{-24} g$ \\
\end{tabular}
\caption{Mass and Charge of Subatomic Particles}
\label{table:particles:mass-and-charge}
\end{table}

\section{Rutherford and the Planetary Model of the Atom}
\begin{itemize}
\item Ernest Rutherford (1871--1937) experimented with $\alpha$ (alpha) particles and gold (\ce{Au}) foil to investigate the Thompson ``plum-pudding" model
\item he found the $\alpha$ particles were much more uniformly thrown ``off course" than he expected
\item this led him to believe that the positively-charged parts of an atom were more localized than the Thompson model would indicate\footnote{\emph{I.e.} the positive part of the atom was ``all in one place," rather than being a sort of an amorphous mass.} 
\item Rutherford's model of the atom is pretty close to the iconic atom you see on movies and T.V.: protons all gathered in the center, with electrons circling them at a distance
\item note carefully the critique of Rutherford's model\cite[p. 213]{wile-chem-2}: even in Rutherford's time, it was known that a charged particle moving in a circle would lose energy in the form of emitted light, so that the electron(s) would quickly fall into the atom's center
\end{itemize}

\section{The Nature of Light}
\begin{itemize}
\item light is super complicated, we don't really understand it, and what we do understand is confusing\footnote{You have been warned}
\end{itemize}

\subsection{Wave/Particle Duality}
\begin{definition}[Particle/Wave Duality Theory]\label{definition:wave-particle-duality}
The theory that light sometimes behaves as a particle, and sometimes behaves as a wave.
\end{definition}

\begin{itemize}
\item light is an electromagnetic wave: it is a wave traveling through the electromagnetic fields of the universe
\item light is a particle: it's a stream of small particles called ``photons"
\item our current understanding of light is that both of these statements are true: light is a wave \emph{and} light is a particle
\end{itemize}

\subsection{All about Waves}
\begin{definition}[Wavelength]\label{definition:wavelength}
The distance between two crests of a wave
\end{definition}

\begin{definition}[Amplitude]\label{definition:amplitude}
A measure of the height of the crests or the depths of the troughs on a wave.
\end{definition}

\begin{definition}[Frequency]\label{definition:frequency}
The number of wave crests (or troughs) that pass a given point each second.
\end{definition}

\begin{itemize}
\item waves are a disturbance in a medium that carries energy while the medium does not\footnote{This can be confusing to students, because the waves we see in real life aren't always the best example of waves. Surf, for example, is actually a \emph{decaying} waveform.}
\item so the idea is that a medium might be ``sitting still" as a whole, but a wave can still move through it, carrying energy
\item waves have a:
\begin{itemize}
\item wavelength ($\lambda$) (measured in meters, centimeters, etc.)
\item frequency ($f$ or $\nu$) (measured in Hertz (Hz), $1 Hz = 1 s^{-1}$)
\item amplitude ($A$)
\item a speed ($v$) (measured in ``meters per second" ($\frac{m}{s}$))\footnote{In the case of electromagnetic waves, we use the symbol $c$ to indicate the speed of light ($3.00 \times 10^{8} \frac{m}{s}$).}
\end{itemize}
\item generally we take the speed of a wave to be: $v = \lambda  f$
\item so waves with higher frequency have lower wavelength
\item the energy of a wave is related to its amplitude: waves with higher amplitude have higher energy
\item since we're talking mainly about light here, we'll mention the speed of light:\\ $$c = 3.00 \times 10^{8} \frac{m}{s}$$
\end{itemize}


\subsection{The Electromagnetic Spectrum}
\begin{itemize}
\item it turns out what we perceive as light is really just a slice of all the electromagnetic waves that exist\footnote{If you really want to blow your mind, look up Maxwell's Equations. Whoa!}
\item we already know about that: radio waves are electromagnetic waves with \emph{lower frequency} than visible light, ultraviolet rays are electromagnetic waves with \emph{higher frequency} than visible light (see \cite[p. 219]{wile-chem-2})\footnote{Fun fact: ``infrared" means ``lower than red," meaning it has lower frequency than the color red. ``Ultraviolet" means ``higher than violet," meaning it has higher frequency than the color violet.}
\item while traditionally energy is related to the \emph{amplitude} of a wave, when we talk about electromagnetic waves, it turns out that energy is related to the \emph{frequency} of the wave\footnote{Our text just sort of mentions this like it's no big deal, but it's one of the most shocking discoveries of the last 200 years.}
\item we can say that $E = h f$ where $E$ is energy, $f$ is frequency, and $h$ is \textbf{Planck's Constant}:\\
$$h = 6.63 \times 10^{-34} \frac{J}{Hz}$$ 
\item $\frac{J}{Hz}$ is ugly, so I generally write it as: $h = 6.63 \times 10^{-34} J \cdot s$. I always find Hertz an awkward unit of measurement.
\footnote{$1 Hz = \frac{1}{s}$, one Hertz is one ``per second."}
\end{itemize}

\section{The Bohr Model of the Atom}
\begin{itemize}
\item this is not ``the boring model of the atom"!
\item it's not really possible to overstate how much Max Planck's work shook up the world of Physics: \emph{everyone} knew that the energy of a wave depends on its \emph{Amplitude}, not its \emph{frequency}
\item in 1908, Albert Einstein won the Nobel Prize for his work on the photoelectric effect,\footnote{Not, surprisingly, for his work on Relativity.} which turned out to be a confirmation of Planck's work
\item so Max Planck's hypothesis was experimentally confirmed, but it still re-wrote what we ``knew" to be true
\item this gave rise to the idea of \emph{Quantum Physics}, which Bohr incorporated into his model of the atom
\item the most fundamental concept of Quantum Physics is that energy doesn't \emph{flow}, rather it moves in [very, very tiny] packets, and you can't ever have $\frac{1}{2}$ a packet
\item so you can have $0n$ packets, $5n$ packets, or even $1.23456789 \times 10^{100} n$ packets, but you can't ever have a \emph{partial} packet
\item Bohr started with Rutherford's model, but added some features:
\begin{itemize}
\item he included neutrons in the nucleus of the atom\footnote{Rutherford's model of the atom didn't include neutrons, as they hadn't yet been discovered.}
\item he proposed electrons only live in certain \emph{allowed} locations, or ``orbits"
\item lower-energy orbits are closer to the nucleus, higher-energy orbits are further out
\item when an electron gains energy, it can ``jump" to a higher-energy orbit, when it loses energy it ``jumps" to a lower-energy orbit, but it can't ever be between those two orbits
\end{itemize}
\item Bohr's model didn't make sense at the time, but it explained a lot:
\begin{itemize}
\item Rutherford's model couldn't explain why electrons don't [eventually] crash into the nucleus, Bohr said they're not ``allowed" to lose that much energy
\item scientists had already recognized that each element has its own ``signature" wavelengths (and thus frequencies) of light it emits when it is heated
\end{itemize}
\item so in Bohr's model, if a \ce{H} atom is given a shot of energy (say from intense heat), then the electron\footnote{\ce{H} has only one electron, right?} gains energy and ``jumps" to an outer orbit; when the electron ``jumps" back down, it releases a photon whose frequency corresponds to the energy level difference between the outer and inner orbits
\item Bohr's model really only works perfectly for \ce{H} atoms, but it's the forerunner\footnote{Not to sound all messianic here.} of what we now consider the ``current" model
\end{itemize}

\section{The Quantum Model of the Atom}
\begin{itemize}
\item we still believe a lot of the ideas behind the Bohr model
\item we now think the orbits are more complicated: not merely circles at fixed distances, but odd, cloud-like shapes we call \emph{orbitals}
\footnote{Orbitals do share the idea of fixes energy levels, so we still think Bohr was right about that, but they're not circular paths at fixed radii.}
\item there are different shapes of orbitals, detailed in the text book
\footnote{I don't want to get too deep into the bunny trail, but one feature of quantum mechanics is \emph{uncertainty}, where an electron in an orbital isn't really like a small ball zooming around the nucleus.
Instead, an electron in an orbital is very like a cloud where the more dense parts of the cloud indicate a place where the electron is ``more probable," and the less dense parts are where the electron is ``less probable."
If we think in those terms, we might think of the spaces between orbitals are places where the electron has ``zero probability."
This gets complicated quickly, and it can be easy to get lost in it.} 
\item the orbitals aren't unique: there are many $s$ orbitals (each one with a different energy level), many $p$ orbitals, and many $d$ orbitals
\item like in the Bohr Model, the Quantum Model puts these different orbitals at different distances from the atom's nucleus that correspond to energy levels
\end{itemize}

\subsection{$s$ Orbitals}
\begin{itemize}
\item the $s$ orbitals have a \emph{spherical} shape
\item there is one $s$ orbital per energy level: $1s$, $2s$, $3s$, etc. 
\item so $1s$ means ``energy level 1, $s$ orbital"
\item just like in the Bohr Model, the electrons in the more outer orbitals (\emph{e.g.} the $3s$ orbital) have more energy than electrons in the more inner orbitals (\emph{e.g.} $1s$)
\end{itemize}

\subsection{$p$ Orbitals}
\begin{itemize}
\item the $p$ orbitals have a \emph{dumbbell} shape
\item there is no $p$ orbital in the innermost energy level, so there's not a $1p$ orbital
\item there are three $p$ orbitals per energy level, oriented along three perpendicular axes (we can think of $x$, $y$, and $z$ axes here)
\item so there are three $2p$ orbitals, three $3p$ orbitals, three $4p$ orbitals, etc.
\item just like in the Bohr Model, the electrons in the more outer orbitals (\emph{e.g.} the $3p$ orbital) have more energy than electrons in the more inner orbitals (\emph{e.g.} $2p$)
\end{itemize}

\subsection{$d$ Orbitals}
\begin{itemize}
\item the $d$ orbitals have a ``complicated" shape\cite[p. 230--231]{wile-chem-2}
\item there are no $d$ orbitals below the third energy level, so there is no $1d$ or $2d$ orbital
\item there are five $d$ orbitals per energy level
\item just like in the Bohr Model, the electrons in the more outer orbitals (\emph{e.g.} the $4d$ orbital) have more energy than electrons in the more inner orbitals (\emph{e.g.} $3d$)
\end{itemize}

\subsection{Electron Configurations}
\begin{itemize}
\item it turns out for any given orbital, there is room for two electrons in that orbital:
\begin{itemize}
\item so there can be two electrons in $1s$, two in $2s$, etc.
\item two in \emph{each} of the $p$ orbitals (so six $p$ electrons per energy level)
\item two in \emph{each} of the $d$ orbitals (so ten $d$ electrons per energy level)
\item two in \emph{each} of the $f$ orbitals (so fourteen $f$ electrons per energy level), but our text doesn't discuss $f$ orbitals
\end{itemize}
\item and we didn't discuss the $f$ orbitals here
\item I found the \href{https://chem.libretexts.org/@go/page/1650}{Electron Orbitals page at LibreTexts Chemistry} to be helpful and approachable\cite{chem:libretexts:orbitals}
\end{itemize}

\begin{definition}[Ground State]\label{defn:ground-state}
The lowest possible energy state for a given substance.
\end{definition}

\begin{itemize}
\item we've already discussed how energy always ``wants" to be kinetic\footnote{Remember we specified it's a ``local minimum" for potential energy that drives almost everything.}
\item \textbf{All forms of matter try to stay in their lowest possible energy state.}\cite[pp. 231--232]{wile-chem-2}
\item so every element will try to stay at --- or return to --- its ground state (Definition \ref{defn:ground-state}, p. \pageref{defn:ground-state})
\item so when an atom is excited, one or more of its electrons ``jumps" to a different energy level
\item but the atom wants to get back to its ground state, so the electron ``jumps" down to the lowest level it can, and releases electromagnetic energy (in the form of a photon) so that it can get back to its ground state
\item in the Quantum Model of the atom, we ``build" atoms by filling in electrons from lowest-energy to highest-energy locations
\item there are some examples in the text book\cite[pp. 232--233]{wile-chem-2}, but we'll work through some here
\end{itemize}

\subsection{Examples}
\begin{enumerate}[label=Example \arabic*]
\item Hydrogen (\ce{H})
\begin{itemize}
\item \ce{H} has one proton (atomic number 1) and no neutrons (atomic mass is 1)
\item \ce{H} has one electron
\item the lowest energy level is $1$, and the lowest energy orbital is $1s$ (level 1 only has one orbital)
\item so the \ce{H} atom in its ground state has an electron configuration of $1s^{1}$ (one electron in orbital $1s$)
\end{itemize}

\item Sodium (\ce{Na})
\begin{itemize}
\item \ce{Na} has eleven protons (atomic number 11) and 12 neutrons (atomic mass is 23, $23 - 11 = 12$)
\item \ce{Na} has 11 electrons:
\begin{itemize}
\item the lowest orbital, $1s$, can hold 2 electrons
\item the next lowest orbital, $2s$ can hold 2 electrons
\item the next lowest orbital, $2p$ can hold 6 electrons
\item that leaves one more electron ($11- (2 + 2 + 6) = 1$), so it goes into the next lowest orbital
\item the next lowest energy orbital is $3s$, so one electron goes into it
\end{itemize}
 \item so the \ce{Na} atom in its ground state has an electron configuration of $1s^{2} 2s^{2} 2p^{6} 3s^{1}$ 
\end{itemize}
\end{enumerate}


\section{Electron Configurations and the Periodic Table}
\begin{itemize}
\item our electron configuration is reflected in an atom's position in the Periodic Table
\item elements in the two leftmost columns (the Alkaline and Alkaline Earth metals) all have configurations ending in an $s$ orbital
\begin{itemize}
\item the column indicates the count the configuration ends in (\emph{e.g.} \ce{H} is in column 1A, so it ends in $s^{1}$)
\item the row indicates which layer the configuration ends in (\emph{e.g.} \ce{Cs} is in row 6, so it ends in $6s^{1}$)
\footnote{Actually, $1s^{2} 2s^{2} 2p^{6} 3s^{2} 3p^{6} 4s^{2} 3d^{10} 4p^{6} 5s^{2} 4d^{10} 5p^{6} 6s^{1}$}
\item so we can just assume \ce{Mg} ends in $3s^{2}$
\end{itemize}
\item elements in the six rightmost columns all have configurations ending in a $p$ orbital
\begin{itemize}
\item the column indicates the count the configuration ends in (\emph{e.g.} \ce{Si} is in column 4A, so it ends in $p^{4}$)
\item the row indicates which layer the configuration ends in (\emph{e.g.} \ce{Br} is in row 3, so it ends in $3p^{5}$)
\end{itemize}
\item the transition metals (the middle section) all end in $d$ orbitals, but because a $d$ orbital has lower energy than the $p$ orbital one layer down, we have to shift our rows\ldots
\item so \ce{Cr} is in column 6B, row 4, so its electron configuration ends in $3d^{4}$
\footnote{Actually, $1s^{2} 2s^{2} 2p^{6} 3s^{2} 3p^{6} 4s^{2} 3d^{4} $}
\item note we can write an \emph{abbreviated} electron configuration starting with the \emph{last} 8A element and just work up from there:
\begin{itemize}
 \item \ce{Na} is row 3, row 1A: the element in row 2, column 8A is \ce{Ne}, so we can write \ce{Na} as $[Ne] 3s^{1}$ 
 \item \ce{Ag} is row 5, column 1B: the element in row 4 column 8A is \ce{Kr}, so we can write \ce{Ag} as $[Kr] 5s^{2} 4d^{9}$ 
 \item \ce{Cs} is in row 6, column 1A: the element in row 5, column 8A is \ce{Xe}, so we can write \ce{Cs} as $[Xe] 6s^{1}$
\end{itemize}

\end{itemize}

\section{Homework}
Review Problems: p. 245 \# 1--10 (not to be turned in)\\
Practice Problems: p. 246 \# 1--10 (due 2025-12-12)\\
Experiment 7.1, p. 203 (due 2025-12-05)\\

\nocite{wile-chem-2}
\bibliography{../Chemistry}{}
\bibliographystyle{apalike}

\clearpage
\subsection{Answers to Practice Problems}
Answers to Practice Problems p. 246 \# 1--10:
\begin{enumerate}
\item
\begin{enumerate}[label=(\alph*)]
\item 40, 40, 50
\item 80, 80, 122
\item 28, 28, 30
\item 86, 86, 136
\end{enumerate}

\item \ce{^{22}Na}, \ce{^{23}Na}, and \ce{^{24}Na}  

\item \ce{^{84}Y}

\item $2.5 \times 10^{-6} m$

\item $3.5 \times 10^{-13} J$

\item $5.6 \times 10^{-19} J$

\item the $3s$ orbital has higher energy than the $2p$ electron, the $s$ orbital is spherical, the $p$ orbital is dumbbell-shaped

\item 
\begin{enumerate}[label=(\alph*)]
\item $1s^{2} 2s^{2} 2p^{6} 3s^{2} 3p^{6} 4s^{2} 3d^{2} $
\item $1s^{2} 2s^{2} 2p^{6} 3s^{2} 3p^{4} $
\item $1s^{2} 2s^{2} 2p^{6} 3s^{2} 3p^{6} 4s^{2} 3d^{10} 4p^{6} 5s^{1} $
\end{enumerate}

\item 
\begin{enumerate}[label=(\alph*)]
\item $1s^{2} 2s^{2} 2p^{6} 3s^{2} 3p^{6} 4s^{2} 3d^{2} $
\item $1s^{2} 2s^{2} 2p^{6} 3s^{2} 3p^{4} $
\item $1s^{2} 2s^{2} 2p^{6} 3s^{2} 3p^{6} 4s^{2} 3d^{10} 4p^{6} 5s^{1} $
\end{enumerate}

\item 
\begin{enumerate}[label=(\alph*)]
\item $[Ar] 4s^{2} 3d^{3} $
\item $[Kr] 5s^{2} 4d^{10} 5p^{2} $
\item $[Kr] 5s^{2} 4d^{10} 5p^{1} $
\end{enumerate}

\item 
\begin{enumerate}[label=(\alph*)]
\item $3p$ can only hold 6 electrons
\item $3d$ is higher energy than $4s$, so it fills later
\end{enumerate}

\end{enumerate}


\end{document}  