\documentclass[10pt, oneside]{article}   	% use "amsart" instead of "article" for AMSLaTeX format
\usepackage{geometry}                		% See geometry.pdf to learn the layout options. There are lots.
\geometry{letterpaper}                   		% ... or a4paper or a5paper or ... 
%\geometry{landscape}                		% Activate for rotated page geometry
%\usepackage[parfill]{parskip}    		% Activate to begin paragraphs with an empty line rather than an indent
\usepackage{graphicx}				% Use pdf, png, jpg, or eps§ with pdflatex; use eps in DVI mode
								% TeX will automatically convert eps --> pdf in pdflatex		
\usepackage{amssymb}
\usepackage{mhchem}
\usepackage{hyperref}

\title{Module 11, Part 2}
\author{Mark Peever\\ \texttt{mpeever@gmail.com}}
\date{February 27, 2026}

\begin{document}
\maketitle

\section*{Objectives}
\marginpar{0 minutes}
Refer to \href{https://drive.google.com/file/d/1HwBnpMMAaghrcIhv391fhaqhtSwE00vL/view?usp=sharing}{Module 11 Notes}.\\

By the end of this class, the students should be able to\ldots
\begin{itemize}
\item describe the Perfect Map Paradox
\item use the terms ``endothermic" and ``exothermic" correctly
\item describe Molality
\item describe Freezing Point Depression ($\Delta T = - i \cdot K_{f} \cdot m $)
\item describe Boiling Point Elevation ($\Delta T =  i \cdot K_{b} \cdot m $)
\end{itemize}

\section*{Welcome \& Devotion}
\marginpar{5 minutes}
\begin{itemize}
\item have one student read \href{https://www.biblegateway.com/passage/?search=titus\%203\%3A3-8\&version=LSB}{Titus 3:3--8}
\end{itemize}

\section*{Module 10 Quiz}
\marginpar{20 minutes}
\begin{itemize}
\item have students complete the Module 10 Quiz
\item go over the Quiz correct answers and take in the papers
\end{itemize}

\section*{Perfect Map Paradox}
\marginpar{15 minutes}
\begin{itemize}
\item discuss Perfect Map Paradox
\item a perfect map is impossible
\item if you \emph{could} make a perfect map, it wouldn't be usable as a map
\item maps are useful \emph{because} they omit information
\end{itemize}

\section*{Energy Changes in Dissolution}
\marginpar{10 minutes}
\begin{itemize}
\item most dissolution is \emph{endothermic}
\item some dissolution is \emph{exothermic}
\item some dissolutions involve actual chemical changes, which can have way more thermal effect than the dissolution itself
\end{itemize}

\section*{Molality}
\marginpar{10 minutes}
\begin{itemize}
\item molality is like molarity, but using mass instead of volume
\item $ b = \frac{n}{m} $
\item $1 molal = \frac{1 mole}{1 kg} $
\end{itemize}

\section*{Freezing Point Depression}
\marginpar{5 minutes}
\begin{itemize}
\item solutions generally have lower freezing points. 
\item $\Delta T = - i \cdot K_{f} \cdot m $
\item either two substances have to change phase \emph{or} there has to be a precipitation \emph{and} a phase shift
\item this is why we salt roads 
\item this is why the ocean doesn't generally freeze
\end{itemize}

\section*{Boiling Point Elevation}
\marginpar{5 minutes}
\begin{itemize}
\item solutions generally have higher boiling points 
\item $\Delta T =  i \cdot K_{b} \cdot m $
\item either two substances have to change phase \emph{or} there has to be a precipitation \emph{and} a phase shift
\item this is why we use coolant in cars
\end{itemize}

\section*{Questions for me}
\marginpar{5 minutes}

\section*{Assignment}
\marginpar{5 minutes}
\begin{itemize}
\item Review Problems: p. 381 \# 1--10 (not to be turned in)
\item Practice Problems: p. 382 \# 1--10 (due 2026-03-06)
\end{itemize}



\end{document}  