\documentclass[10pt, oneside]{article}   	% use "amsart" instead of "article" for AMSLaTeX format
\usepackage{geometry}                		% See geometry.pdf to learn the layout options. There are lots.
\geometry{letterpaper}                   		% ... or a4paper or a5paper or ... 
%\geometry{landscape}                		% Activate for rotated page geometry
%\usepackage[parfill]{parskip}    		% Activate to begin paragraphs with an empty line rather than an indent
\usepackage{graphicx}				% Use pdf, png, jpg, or eps§ with pdflatex; use eps in DVI mode
								% TeX will automatically convert eps --> pdf in pdflatex		
\usepackage{amssymb}
\usepackage{mhchem}
\usepackage{hyperref}

\title{Module 11, Part 1}
\author{Mark Peever\\ \texttt{mpeever@gmail.com}}
\date{February 20, 2026}

\begin{document}
\maketitle

\section*{Objectives}
\marginpar{0 minutes}
Refer to \href{https://drive.google.com/file/d/1HwBnpMMAaghrcIhv391fhaqhtSwE00vL/view?usp=sharing}{Module 11 Notes}.\\

By the end of this class, the students should be able to\ldots
\begin{itemize}
\item explain what a \textbf{solution} is (homogeneous mixture)
\item describe a solution in correct terms: ``solute," ``solvent" (solute dissolves into the solvent)
\item describe how ionic compounds dissolve in water \footnote{We've talked about this off and on all year: it should be review by now.}
\item describe how polar covalent compounds dissolve in water 
\end{itemize}

\section*{Welcome \& Devotion}
\marginpar{5 minutes}
\begin{itemize}
\item have one student read \href{https://www.biblegateway.com/passage/?search=titus\%203\%3A3-8\&version=LSB}{Titus 3:3--8}
\item Today is a relatively low math day!
\end{itemize}

\section*{Solutions, Solutes, and Solvents}
\marginpar{5 minutes}
\begin{itemize}
\item a \textbf{solution} is a homogeneous mixture
\item the main ingredient in a solution is the \textbf{solvent}
\item the thing (or things) dissolved into the solvent is (are) the \textbf{solute(s)}
\end{itemize}

\section*{How Ionic Compounds Dissolve in Water}
\marginpar{20 minutes}
\begin{itemize}
\item we're primarily talking about \emph{aqueous} solutions here: water is a polar covalent molecule!
\item ionic compounds split into positive ions (cations) and negative ions (anions) in water \footnote{Make a joke about Schr\"odinger's cat and cations.}
\item not all ionic compounds are ``loose" enough to dissolve in water: introduce \emph{solubility} here
\item \emph{e.g.} \ce{NaCl (aq) -> Na^{+} (aq) + Cl^{-} (aq)}
\item show how \ce{Na^{+}} ions gather around the \ce{O^{-2 \delta}} side of \ce{H2O} and \ce{Cl^{-}} ions gather around \ce{H^{+ \delta}} side
\item if you have time, mention \emph{differentials} at this point in connection with $\pm \delta$ charges, tell the kids not to write this down
\end{itemize}

\section*{How Polar Covalent Compounds Dissolve in Water}
\marginpar{15 minutes}
\begin{itemize}
\item polar covalents do something similar, but they don't pry apart into ions
\item remember our friend \ce{NH3 + H2O -> NH4^{+} (aq) + OH^{-} (aq)}
\item see? \emph{some} physical changes drag chemical changes along with them
\item show how \ce{PH3} aligns the with a $-3 \delta$ charge at the \ce{P} with the \ce{H^{+ \delta}} side of \ce{H2O}
\item just like with ionics, polar covalent compounds ``disappear"\footnote{The kids are to young to remember INXS, but it goes \emph{so well} here. "De de de de De de."}  into the water
\end{itemize}

\section*{How Liquids and Gases Dissolve}
\marginpar{5 minutes}

\begin{itemize}
\item liquids are less internally attracted than solids, so the solvent has much less to do prying ions and/or molecules apart
\item gases are less ``held together" even than liquids
\item but in general, dissolution works the same as for solids
\end{itemize}

\section*{Solubility}
\marginpar{20 minutes}

\begin{itemize}
\item give definition of \textbf{solubility} from the notes
\item give definition of \textbf{saturated solution} from the notes
\item discuss solubility increase/decrease with temperature and pressure
\item \emph{e.g.} soda goes flat when it has been opened (not as soluble in lower pressure)
\item \emph{e.g.} salmon have trouble spawning in streams that are too warm (not enough \ce{O2} in solution)
\item \emph{e.g.} hot chocolate mix dissolves better in hot water
\item \emph{e.g.} espresso requires both high temperature and high pressure to dissolve coffee oils
\end{itemize}


\section*{Questions for me}
\marginpar{5 minutes}

\section*{Announcements}
\marginpar{5 minutes}

\begin{itemize}
\item I will remind everyone of the announcements via email, but we can't announce them too often!
\item Module 10 Practice Problems are due (put them into Mr. Peever's folder)
\item Experiment 10.1 or 10.2 are due (put them into Mr. Peever's folder)
\item We'll plan to have an acid/base quiz (Module 10 quiz) next week
\end{itemize}

%\section*{Assignment}
%\marginpar{5 minutes}
%\begin{itemize}
%\item Review Problems: p. 317 \# 1--10 (not to be turned in)
%\item Practice Problems: p. 318 \# 1--10 (due 2026-02-06)
%\item Experiment 9.1 (due 2026-02-06)
%\end{itemize}


\end{document}  