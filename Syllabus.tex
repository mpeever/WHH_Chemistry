\documentclass[11pt, oneside]{article}   	% use "amsart" instead of "article" for AMSLaTeX format
\usepackage{geometry}                		% See geometry.pdf to learn the layout options. There are lots.
\geometry{letterpaper}                   		% ... or a4paper or a5paper or ... 

\usepackage[parfill]{parskip}    		% Activate to begin paragraphs with an empty line rather than an indent
\usepackage{graphicx}				% Use pdf, png, jpg, or eps§ with pdflatex; use eps in DVI mode
								% TeX will automatically convert eps --> pdf in pdflatex		
\usepackage{amssymb}
\usepackage{cite}

\usepackage{gb4e}
\usepackage{enumitem}
\usepackage{cancel}
\usepackage{amsmath}

\title{Chemistry}
\author{Mark Peever \texttt{mpeever@gmail.com}}
\date{August 22, 2025}

\begin{document}
\maketitle

\begin{center}
Psalm 107:23--32
\end{center}

\section{Overview}
This is an introductory Chemistry class. Our goal is to lay a solid foundation for studying ``hard" sciences\footnote{As opposed to social sciences like Sociology.} through high school and college. 

We'll keep Psalm 107:23--32 in mind as we begin to exploring chemistry together. Chemistry can feel a bit overwhelming, like being lost at sea, but those who take risks see the wonders of God.

Our goals for this course are:
\begin{itemize}
\item to cultivate a love of ``hard" sciences, especially chemistry,
\item to build good math habits, especially as they pertain to physical sciences, and
\item to build a good foundation of skills for problem-solving
\end{itemize}

\section{Prerequisites/Corequisites}
There will be some overlap with \textbf{Physical Science}, don't be discouraged if there's a lot of review to start!

We will use basic algebra a \emph{lot}. 
We'll go as far as logarithms for pH calculations.  
I plan to use class time to review/refresh math topics as we go, so don't let that scare you.
If you have finished \textbf{Algebra I}, you should be fine. 
If you haven't yet finished Algebra I \emph{but are willing to put in the effort}, you should be fine.

If you don't understand the math, you need to ask.

\section{Course Pace and Class Time}
There are 16 modules in the text book, so we will cover one module every two weeks, with some exceptions (see Table \ref{table:class-schedule}). \emph{Please plan to have read and/or worked through each module in preparation for class.}  We will review the module's material in the class, answering student questions as a priority. The pace might be a bit different from what you've done in the past, so give yourself some time to adjust.

At the end of each module are practice problems: I plan to assign every problem in the book as home-work. You have been warned.

We will have several quizzes through the class to ensure we're all up to date. At the end of each quarter there will be a take-home exam.

\section{Expectations}
This will be a math-heavy class. Come with a \textbf{scientific calculator}, a solid \textbf{three-ring binder}, plenty of \textbf{looseleaf paper}, \textbf{pencils} and plenty of \textbf{erasers}. 

I relish questions, especially questions that show you have been listening. Bring plenty of those.

\section{Meeting Times}
TRC Classroom 2       \\
8:55 A.M. -- 10:20 A.M. Fridays \\ 

\section{Topics}
I have included a weekly break-down of topics in Table \ref{table:class-schedule}.
We'll refine our schedule as we go: I anticipate some topics will take less time to cover than others.

\begin{table}[h]
\centering
\begin{tabular}{ l | l }
Date & Topic \\
\hline
2025-08-22 & Module 1:  Measurement and Units \\
2025-08-29 & Module 1:  Measurement and Units \\
2025-09-05 & Module 2: Energy, Heat, and Temperature \\
2025-09-12 & Module 2: Energy, Heat, and Temperature \\
2025-09-19 & Module 3: Atoms and Molecules \\
2025-09-26 & Module 3: Atoms and Molecules \\
2025-10-03 & Module 4: Classifying Matter and Its Changes \\
2025-10-10 & Module 4: Classifying Matter and Its Changes * \\

2025-10-17 & \textbf{Fall Break} \\

2025-10-24 & Module 5: Counting Molecules and Atoms in Chemical Equations \\
2025-10-31 & Module 5: Counting Molecules and Atoms in Chemical Equations \\
2025-11-07 & Module 6: Stoichiometry \\
2025-11-14 & Module 6: Stoichiometry \\
2025-11-21 & Module 7: Atomic Structure \\
2025-12-05 & Module 7: Atomic Structure \\
2025-12-12 & Module 8: Molecular Structure \\
2025-12-19 & Module 8: Molecular Structure \\

2025-12-26  & \textbf{Winter Break} \\
2026-01-02  & \textbf{Winter Break} \\
2026-01-09  & \textbf{Winter Break} \\

2026-01-23 & Module 9: Polyatomic Ions and Molecular Geometry \\
2026-01-30 & Module 9: Polyatomic Ions and Molecular Geometry * \\
2026-02-06 & Module 10: Acid/Base Chemistry \\
2026-02-13 & Module 10: Acid/Base Chemistry \\
2026-02-20 & Module 11: The Chemistry of Solutions \\
2026-02-27 & Module 12: The Gas Phase \\
2026-03-06 & Module 13: Thermodynamics \\
2026-03-13 & Module 13: Thermodynamics * \\
2026-03-20 & \textbf{Spring Break} \\
2026-03-27 & Module 14: Kinetics \\
2026-04-03 & \textbf{Good Friday} \\
2026-04-10 & Module 14: Kinetics \\
2026-04-17 & Module 15: Chemical Equilibrium \\
2026-04-24 & Module 15: Chemical Equilibrium \\
2026-05-01 & Module 16: Reduction/Oxidation Reactions \\
2026-05-08 & Final Review \\
\end{tabular}
\caption{(Tentative) Class Schedule}\label{table:class-schedule}
\end{table}



%
%\section{Module Overview}
%See calendar (Table \ref{table:class-schedule})
%
%% First Quarter
%\subsection{Module 1:  Measurement and Units} 
%Introduction to measurement, units, unit conversion, and significant figures.
%
%\subsection{Module 2: Energy, Heat, and Temperature}
%Introduction to energy and the Law of Conservation of energy.
%
%\subsection{Module 3: Atoms and Molecules}
%Introduction to atoms and molecules, the Law of Conservation of Matter and a deeper dive into Dalton's atomic model.
%
%\subsection{Module 4: Classifying Matter and Its Changes}
%Introduction to physical and chemical changes; beginning chemical reactions and chemical equations.
%
%% Second Quarter
%\subsection{Module 5: Counting Molecules and Atoms in Chemical Equations}
%TBD
%
%\subsection{Module 6: Stoichiometry}
%TBD
%
%
%\subsection{Module 7: Atomic Structure}
%TBD
%
%
%\subsection{Module 8: Molecular Structure}
%TBD
%
%
%% Third Quarter
%\subsection{Module 9: Polyatomic Ions and Molecular Geometry}
%TBD
%
%\subsection{Module 10: Acid/Base Chemistry}
%TBD
%
%\subsection{Module 11: The Chemistry of Solutions}
%TBD
%
%\subsection{Module 12: The Gas Phase}
%TBD
%
%
%% Fourth Quarter 
%\subsection{Module 13: Thermodynamics}
%TBD
%
%\subsection{Module 14: Kinetics}
%TBD
%
%\subsection{Module 15: Chemical Equilibrium}
%TBD
%
%\subsection{Module 16: Reduction/Oxidation Reactions}
%TBD



\end{document}