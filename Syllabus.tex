\documentclass[11pt, oneside]{article}   	% use "amsart" instead of "article" for AMSLaTeX format
\usepackage{geometry}                		% See geometry.pdf to learn the layout options. There are lots.
\geometry{letterpaper}                   		% ... or a4paper or a5paper or ... 

\usepackage[parfill]{parskip}    		% Activate to begin paragraphs with an empty line rather than an indent
\usepackage{graphicx}				% Use pdf, png, jpg, or eps§ with pdflatex; use eps in DVI mode
								% TeX will automatically convert eps --> pdf in pdflatex		
\usepackage{amssymb}
\usepackage{cite}

\usepackage{gb4e}
\usepackage{enumitem}
\usepackage{cancel}
\usepackage{amsmath}

\usepackage{nunito}
\usepackage[T1]{fontenc}

\title{Chemistry Syllabus}
\author{Mark Peever\\ \texttt{mpeever@gmail.com}}
%\date{August 15, 2025}

\begin{document}
\maketitle

\begin{center}
Psalm 107:23--32
\end{center}

\section{Overview}
This is an introductory chemistry class. Our goal is to lay a strong foundation for studying ``hard" sciences\footnote{As opposed to social sciences like Sociology.} through high school and college. 

We'll keep \textbf{Psalm 107:23--32} in mind as we begin exploring chemistry together. Chemistry can feel a bit overwhelming --- a bit like being lost at sea --- but it's only as we take risks that we are in a position to see the wonders of God.

Our goals for this course are:
\begin{itemize}
\item to cultivate a love of ``hard" sciences, especially chemistry,
\item to build good math habits, especially as they pertain to physical sciences, and
\item to build a good foundation of skills for problem-solving
\end{itemize}

\section{Contact Information}
Mark Peever\\
1164 Ruby Creek Rd.\\
Harvard, ID 83834 \\
email: \texttt{mpeever@gmail.com} \\
text: \texttt{253-328-2197}\footnote{Cell phone reception in Harvard is terrible\ldots please bias for text over voice.} \\


\section{Meeting Times}
TRC Classroom 2       \\
8:55 A.M. -- 10:20 A.M. Fridays \\ 


\section{Text Book}
 \emph{Exploring Creation with Chemistry} (second edition, 2003) by Dr. J. L. Wile.


\section{Prerequisites/Corequisites}
There will be some overlap with \textbf{Physical Science}, don't be discouraged if there's a lot of review to start!

We will use basic algebra a \emph{lot}. 
I plan to use class time to review/refresh math topics as we go, so don't let that scare you.
If you have finished \textbf{Algebra I}, you should be fine. 
If you are still working on Algebra I \emph{and are willing to put in the effort}, you should be fine.

If you don't understand the math, you need to ask.

\section{Course Pace and Class Time}
There are 16 modules in the text book, so we will cover one module every two weeks, with some exceptions (see Table \ref{table:class-schedule-year}, page \pageref{table:class-schedule-year}). 
\emph{Please plan to have read and/or worked through each module in preparation for class.}  
We will review the module's material in the class, answering student questions as a priority. 
If the module is split between two weeks, then plan to have worked through the first half on the first week, the second half on the second. 

At the end of each module are practice problems: I plan to have assigned every problem in the book eventually.\footnote{You have been warned.}

There are several labs scheduled for this course. As much as possible, we'll do labs as homework, but we'll dedicate some class time to learning how to write them up.
Some labs will require class time: we'll schedule those as we go.

\section{Expectations}
This will be a math-heavy class. 
Come with a \textbf{scientific calculator}, a sturdy \textbf{three-ring binder}, plenty of \textbf{looseleaf paper} to go in it, \textbf{pencils} and plenty of \textbf{erasers}.

Please don't attempt to turn in any work written in ink: this will be a \textbf{pencil-only course}.\footnote{No one will get everything right the first time: you \emph{will} need to be able to erase your answers.}

I relish \textbf{questions}, especially questions that show you have been paying attention. Bring plenty of those.


\section{Homework and Grading}
I have listed homework questions for each Module in Table \ref{table:class-schedule-q1} (page \pageref{table:class-schedule-q1}).
Those will be due the next class (\emph{e.g.} p. 36 \# 1--10 will be due September 5).

We will have quizzes on a more-or-less three-week cadence (see Table \ref{table:class-schedule-q1}).
The quizzes will be given in class, as a check to ensure we're all mastering the material as we go.

There will be one cumulative exam per quarter, which will be a take-home exam, to be proctored by a parent or guardian.

The grades will be calculated as:
\begin{itemize}
\item Homework: 40\%
\item Labs: 30\%
\item Exams: 20\%
\item Quizzes: 10\%
\end{itemize}

\subsection{A Note on Studying}
The inimitable Greg Mitchell used to say, ``You don't study chemistry by reading, you study chemistry by solving problems."

If you're not using a pencil, eraser, and calculator you might be reading your chemistry book, but you're not studying chemistry.

\subsection{A Note on Grading}
 ``The right answer" is of very little interest to me when it comes to Chemistry problems on homework, quizzes, or exams. 
I am much more interested in the process than I am in the result. 

I've included an example problem in Section  \ref{grading-example}  (page \pageref{grading-example}) so you can see how I grade chemistry problems.
In the example, you can see I grade the problem out of a possible five points, and \emph{only one point is for the correct answer}.  We will go over this in class.


\section{Topic Schedule}
\subsection*{Quarter 1 Topic Schedule}
The schedule for the first quarter is given in Table \ref{table:class-schedule-q1} (p. \pageref{table:class-schedule-q1}).
Please note I've included both labs and homework problems in the Q1 schedule.\footnote{We'll assume the labs are to be done at home and the results brought to class on the listed day. There will be labs to be done in class, but we'll plan on doing the labs at home to start.}

\begin{table}[h]
\centering
\begin{tabular}{ l | l }
Date & Topic \& Assignments \\
\hline
2025-08-22 & Module 1:  Measurement and Units \\
                   & Experiment 1.1 \& 1.2 (to be done in class, as time permits) \\

\hline

2025-08-29 & Module 1:  Measurement and Units \\
                   & Experiment 1.4 (do at home) \\
                   & Assignment: p. 36 \# 1--10 \\
\hline

2025-09-05 & Module 2: Energy, Heat, and Temperature \\
                   & Experiment 2.1 (at home) \\
\hline

2025-09-12 & Module 2: Energy, Heat, and Temperature \\
	           & Quiz \\
	           & Assignment: p. 68 \# 1--10 \\ 
\hline

2025-09-19 & Module 3: Atoms and Molecules \\
                   & Experiment 3.2 \\
\hline

2025-09-26 & Module 3: Atoms and Molecules \\
                   & Assignment: p. 98 \# 1--10 \\ 
\hline

2025-10-03 & Module 4: Classifying Matter and Its Changes \\
                    & Experiment 4.3 \\
                    & Quiz \\
\hline

2025-10-10 & Module 4: Classifying Matter and Its Changes * \\
                   & p. 132 \# 1--10 \\
                   & Take-home Quarter 1 Exam \\
\hline

2025-10-17 & \textbf{Fall Break} \\

\end{tabular}
\caption{First Quarter Class Schedule}\label{table:class-schedule-q1}
\end{table}

\subsection*{Quarter 2 Topic Schedule}
The schedule for the second quarter is given in Table \ref{table:class-schedule-q2} (p. \pageref{table:class-schedule-q2}).
Please note I've included both labs and homework problems in the Q2 schedule.

Also, please note that I've scheduled more quizzes for Quarter 2.


\begin{table}[h]
\centering
\begin{tabular}{ l | l }
Date & Topic \& Assignments \\
\hline
2025-10-24 & Module 5:  Counting Molecules and Atoms in Chemical Equations \\
                   & Experiment 5.1 (to be done at home) \\

\hline
2025-10-31 & Module 5:  Counting Molecules and Atoms in Chemical Equations \\
                   & Assignment: p. 162 \# 1--10 (due 2025-11-07) \\
                   & Quiz \\
                   
\hline
2025-11-07 & Module 6: Stoichiometry \\
                   & Experiment 6.1 (at home) (due 2025-11-14) \\

\hline
2025-11-14 & Module 6: Stoichiometry \\
                   & Assignment: p. 199 \# 1--10 (due 2025-11-21) \\
                   & Quiz \\

\hline
2025-11-21 & Module 7: Atomic Structure \\
                   & Experiment 7.1 (at home) (due 2025-12-05)\\

\hline
2025-11-28 & Thanksgiving \\

\hline
2025-12-05 & Module 7: Atomic Structure \\
                   & Assignment: p. 246 \# 1--10 (due 2025-12-12)\\
                   & Quiz \\

\hline
2025-12-12 & Module 8: Molecular Structure \\
                   & Take-home Q2 Exam (due 2025-12-19)\\

\hline
2025-12-19 & Module 8: Molecular Structure \\
                   & Assignment: p. 284 \# 1--10 (due 2026-01-23)\\
                   & Quiz \\
                   & \textbf{Q2 Exam due} \\

\hline
2025-12-26 & \textbf{Winter Break} \\

\hline
2026-01-02 & \textbf{Winter Break} \\

\hline
2026-01-09 & \textbf{Winter Break} \\
\hline
2026-01-16 & \textbf{Winter Break*} \\

\end{tabular}
\caption{Second Quarter Class Schedule}\label{table:class-schedule-q2}
\end{table}

\subsection*{Year-long Overview}
I have included a weekly break-down of topics for the whole year in Table \ref{table:class-schedule-year} (page \pageref{table:class-schedule-year} ).
We'll refine our schedule as we go: I anticipate some topics will take less time to cover than others.

\begin{table}[h]
\centering
\begin{tabular}{ l | l }
Date & Topic \\
\hline
2025-08-22 & Module 1:  Measurement and Units \\
2025-08-29 & Module 1:  Measurement and Units \\
2025-09-05 & Module 2: Energy, Heat, and Temperature \\
2025-09-12 & Module 2: Energy, Heat, and Temperature \\
2025-09-19 & Module 3: Atoms and Molecules \\
2025-09-26 & Module 3: Atoms and Molecules \\
2025-10-03 & Module 4: Classifying Matter and Its Changes \\
2025-10-10 & Module 4: Classifying Matter and Its Changes * \\

2025-10-17 & \textbf{Fall Break} \\

2025-10-24 & Module 5: Counting Molecules and Atoms in Chemical Equations \\
2025-10-31 & Module 5: Counting Molecules and Atoms in Chemical Equations \\
2025-11-07 & Module 6: Stoichiometry \\
2025-11-14 & Module 6: Stoichiometry \\
2025-11-21 & Module 7: Atomic Structure \\
2025-12-05 & Module 7: Atomic Structure \\
2025-12-12 & Module 8: Molecular Structure \\
2025-12-19 & Module 8: Molecular Structure \\

2025-12-26  & \textbf{Winter Break} \\
2026-01-02  & \textbf{Winter Break} \\
2026-01-09  & \textbf{Winter Break} \\
2026-01-16  & \textbf{Winter Break} \\

2026-01-23 & Module 9: Polyatomic Ions and Molecular Geometry \\
2026-01-30 & Module 9: Polyatomic Ions and Molecular Geometry * \\
2026-02-06 & Module 10: Acid/Base Chemistry \\
2026-02-13 & Module 10: Acid/Base Chemistry \\
2026-02-20 & Module 11: The Chemistry of Solutions \\
2026-02-27 & Module 12: The Gas Phase \\
2026-03-06 & Module 13: Thermodynamics \\
2026-03-13 & Module 13: Thermodynamics * \\
2026-03-20 & \textbf{Spring Break} \\
2026-03-27 & Module 14: Kinetics \\
2026-04-03 & \textbf{Good Friday} \\
2026-04-10 & Module 14: Kinetics \\
2026-04-17 & Module 15: Chemical Equilibrium \\
2026-04-24 & Module 15: Chemical Equilibrium \\
2026-05-01 & Module 16: Reduction/Oxidation Reactions \\
2026-05-08 & Final Review \\
\end{tabular}
\caption{(Tentative) Year-long Class Schedule}\label{table:class-schedule-year}
\end{table}

\pagebreak
\section{Example Problem}
\label{grading-example}
\subsection*{Question:} 
Lead has a density of $11.4 g/mL$. If we cast a statue from $3.45 L$ of lead, what is the statue's mass? (p. 36 \# 9)

\subsection*{Answer:}
\begin{enumerate}
\item identify what we're asked to find: \emph{we're looking for the statue's mass} (1 point)
\item organize the information we've been given (1 point):
\begin{itemize}
\item let $\rho$ be the density of lead, $\rho = 11.4 \frac{g}{mL}$
\item let $V$ be the volume of the statue, $V = 3.45 L$
\item let $m$ be the mass of the statue, $m = ?$
\end{itemize}
\item identify the correct equation to relate what we have to what we want (1 point):
\begin{equation} 
\begin{split}
       \rho = \frac{m}{V}
 \end{split}
 \end{equation}
 
 \item solve the equation for the quantity we want (1 point):
 \begin{equation} 
\begin{split}
       \rho        &= \frac{m}{V} \\
       (V)(\rho) &=  (\frac{m}{V})(V) \\
       \rho V     &=  (\frac{m}{\cancel{V}})(\cancel{V}) \\
       \rho V     &=  m \\
       m            &= \rho V
 \end{split}
 \end{equation}
 
 \item substitute the correct values into the resulting equation, being careful with units of measure and significant figures (1 point):
\begin{equation} 
\begin{split}
       m            &= \rho V \\
                      &= (V)(\rho) \\
                      &= (3.45 L)(11.4 \frac{g}{mL}) \\
%                      &= (3.45 L)(11.4 \frac{g}{mL})(\frac{1000 mL}{1 L}) \\
                      &= (3.45 \cancel{L})(11.4 \frac{g}{\cancel{mL}})(\frac{1000 \cancel{mL}}{1 \cancel{L}}) \\
                      &= 39330 g \\
       m            &= 39300 g 
 \end{split}
 \end{equation}
 \end{enumerate}


\end{document}