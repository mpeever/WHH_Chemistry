\documentclass[10pt, oneside]{article}   	% use "amsart" instead of "article" for AMSLaTeX format
\usepackage{geometry}                		% See geometry.pdf to learn the layout options. There are lots.
\geometry{letterpaper}                   		% ... or a4paper or a5paper or ... 
%\geometry{landscape}                		% Activate for rotated page geometry
%\usepackage[parfill]{parskip}    		% Activate to begin paragraphs with an empty line rather than an indent
\usepackage{graphicx}				% Use pdf, png, jpg, or eps§ with pdflatex; use eps in DVI mode
								% TeX will automatically convert eps --> pdf in pdflatex		
\usepackage{amssymb}
\usepackage{mhchem}
\usepackage{hyperref}

\title{Module 8, Part 2}
\author{Mark Peever\\ \texttt{mpeever@gmail.com}}
\date{January 23, 2026}

\begin{document}
\maketitle

\section*{Objectives}
\marginpar{0 minutes}
Refer to \href{https://drive.google.com/file/d/1V3TltK4FD87iRrGvrlL5Io6wkPMI0ZaF/view?usp=sharing}{Module 8 Notes}.\\

By the end of this class, the students should be able to\ldots
\begin{itemize}
\item describe the concept of valence electrons
\item describe ionic compounds in terms of Lewis Structures
\item describe ionization, ionization potential, electronegativity, and atomic radius in terms of periodic properties
\item describe covalent bonds in terms of Lewis Structures
\item use Lewis Structures to analyze covalent compounds
\end{itemize}

This lesson is almost entirely a repetition of the December 19 lesson, which I announced on December 19.

\section*{Welcome \& Devotion}
\marginpar{5 minutes}
\begin{itemize}
\item have one student read \href{https://www.biblegateway.com/passage/?search=Colossians\%201\&version=LSB}{Colossians 1:15--20}
\end{itemize}

\section*{Valence Electrons}
\marginpar{20 minutes}
\begin{itemize}
\item \textbf{Valence electrons} are electrons in the outermost energy level
\item how many valence electrons does \ce{H} have? how many does \ce{He} have?
\item how many valence electrons does \ce{K} have? how many does \ce{Kr} have?
\item how many valence electrons does \ce{Cs} have? how many does \ce{Rn} have?
\item go through each column and comment on valence electron count
\item it turns out atoms are trying to fill their valence electrons!
\end{itemize}

\section*{Lewis Structures}
\marginpar{15 minutes}
\begin{itemize}
\item introduce Lewis Structures
\item work out Lewis Structures for: \ce{H}, \ce{He}, \ce{K}, \ce{Kr}, \ce{N}, \ce{Cl}, \ce{O}, etc.
\end{itemize}

\section*{Ionic Bonds}
\marginpar{15 minutes}
\begin{itemize}
\item show how \ce{NaCl} is formed with a Lewis Structure
\item show how \ce{MgCl2} is formed with a Lewis Structure
\item go over electronegativity, ionization, ionization potential, and atomic radius
\end{itemize}

\section*{Covalent Bonds}
\marginpar{15 minutes}
\begin{itemize}
\item go over \ce{N2} with Lewis Structures
\item go over \ce{O2} with Lewis Structures
\item go over \ce{H2O} with Lewis Structures
\item go over \ce{CH4} with Lewis Structures
\item go over \ce{H2SO4} with Lewis Structures
\end{itemize}

\section*{Periodic Properties }
\marginpar{15 minutes}

\begin{itemize}
\item ionization potential (how hard it is to pry an electron from an atom) increases left-to-right, bottom-to-top
\item electronegativity (how hard an atom attracts an extra electron) increases left-to-right, bottom-to-top
\item atomic radius increases right-to-left, top-to-bottom
\end{itemize}


\section*{Questions for me}
\marginpar{10 minutes}

\section*{Assignment}
\marginpar{5 minutes}
\begin{itemize}
\item Review Problems: p. 283 \# 1--10 (not to be turned in)
\item Practice Problems: p. 284 \# 1--10 (due 2026-01-30)
\end{itemize}


\end{document}  