\documentclass[11pt, oneside]{article}   	% use "amsart" instead of "article" for AMSLaTeX format
\usepackage{geometry}                		% See geometry.pdf to learn the layout options. There are lots.
\geometry{letterpaper}                   		% ... or a4paper or a5paper or ... 

\usepackage[parfill]{parskip}    		% Activate to begin paragraphs with an empty line rather than an indent
\usepackage{graphicx}				% Use pdf, png, jpg, or eps§ with pdflatex; use eps in DVI mode
								% TeX will automatically convert eps --> pdf in pdflatex		
\usepackage{amssymb}
\usepackage{cite}

% numbered examples
\usepackage{gb4e}
\usepackage{enumitem}
\usepackage{cancel}
\usepackage{amsmath}

\usepackage{mhchem}

\usepackage{hyperref}

% Scientific Laws
\newtheorem{definition}{Definition}
\newtheorem{law}{Law}
\newtheorem{theory}{Theory}
\newtheorem{hint}{Hint}

\title{Module 8: Molecular Structure }
\author{Mark Peever\\ \texttt{mpeever@gmail.com}}
\date{December 12 -- December 19, 2025}

\begin{document}
\maketitle

\begin{center}

\end{center}

\section{Overview}
\begin{enumerate}
\item \textbf{Valence Electrons} are the outermost electrons on an atom.
\item molecules are formed almost entirely by interactions in the valence electrons of its constituent atoms
\item atoms in the same columns in the Periodic Table have the same number of valence electrons
\item \textbf{Inert} substances are substances that don't generally react with other substances
\item most elements follow the \textbf{Octet Rule}, attempting to gain or lose electrons so they they have $8$ valence electrons
\item \ce{H} and \ce{He} don't follow the octet rule
\item \textbf{Lewis Structures} are a notation to help us envision valence electrons
\item \textbf{Ions} are electrically-charged atoms with either a surplus or a deficit of electrons
\item \textbf{Electronegativity} is a measure of how strongly an atom attracts extra electrons
\end{enumerate}

\subsection{Valence Electrons}
\begin{definition}[Valence Electrons]\label{definition:valence-electrons}
The electrons furthest from an atom's nucleus. They are generally the electrons with the highest energy level number.
\end{definition}

\begin{itemize}
\item in almost every case, the electron configuration of an atom will end with an orbital that has a capacity of eight electrons
\item \ce{H} and \ce{He} are exceptions: their electron configurations are $1s^{1}$ and $1s^{2}$ 
\item their valences only hold two electrons
\item this leads us to the Octet Rule: atoms ``try" to fill their valence electrons by giving or taking electrons so they end up with eight in their outermost layers
\item in general, atoms on the left side of the Periodic Table find it easier to ``give away" electrons, while atoms on the right side of the Periodic Table ``take" electrons
\item so metals essentially try for an empty valence layer, while non-metals look to fill up their valences
\end{itemize}



\section{Homework}
Review Problems: p. 245 \# 1--10 (not to be turned in)\\
Practice Problems: p. 246 \# 1--10 (due 2025-12-12)\\
Experiment 7.1, p. 203 (due 2025-12-05)\\

\nocite{wile-chem-2}
\bibliography{../Chemistry}{}
\bibliographystyle{apalike}

\clearpage
\subsection{Answers to Practice Problems}
Answers to Practice Problems p. 246 \# 1--10:
\begin{enumerate}
\item bleah, bleah, bleah
\end{enumerate}


\end{document}  