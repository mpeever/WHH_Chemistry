\documentclass[11pt, oneside]{article}   	% use "amsart" instead of "article" for AMSLaTeX format
\usepackage{geometry}                		% See geometry.pdf to learn the layout options. There are lots.
\geometry{letterpaper}                   		% ... or a4paper or a5paper or ... 

\usepackage[parfill]{parskip}    		% Activate to begin paragraphs with an empty line rather than an indent
\usepackage{graphicx}				% Use pdf, png, jpg, or eps§ with pdflatex; use eps in DVI mode
								% TeX will automatically convert eps --> pdf in pdflatex		
\usepackage{amssymb}
\usepackage{cite}

% numbered examples
\usepackage{gb4e}
\usepackage{enumitem}
\usepackage{cancel}
\usepackage{amsmath}

\usepackage{mhchem}

\usepackage{hyperref}

% Scientific Laws
\newtheorem{definition}{Definition}
\newtheorem{law}{Law}
\newtheorem{theory}{Theory}
\newtheorem{hint}{Hint}

\title{Module 8: Molecular Structure }
\author{Mark Peever\\ \texttt{mpeever@gmail.com}}
\date{December 12 -- December 19, 2025}

\begin{document}
\maketitle

\begin{center}

\end{center}

\section{Overview}
\begin{enumerate}
\item \textbf{Valence Electrons} are the outermost electrons on an atom.
\item molecules are formed almost entirely by interactions in the valence electrons of its constituent atoms
\item atoms in the same columns in the Periodic Table have the same number of valence electrons
\item \textbf{Inert} substances are substances that don't generally react with other substances
\item most elements follow the \textbf{Octet Rule}, attempting to gain or lose electrons so they they have $8$ valence electrons
\item \ce{H} and \ce{He} don't follow the octet rule
\item \textbf{Lewis Structures} are a notation to help us envision valence electrons
\item \textbf{Ions} are electrically-charged atoms with either a surplus or a deficit of electrons
\item \textbf{Electronegativity} is a measure of how strongly an atom attracts extra electrons
\end{enumerate}

\subsection{Valence Electrons}
\begin{definition}[Valence Electrons]\label{definition:valence-electrons}
The electrons furthest from an atom's nucleus. They are generally the electrons with the highest energy level number.
\end{definition}

\begin{itemize}
\item waves are a disturbance in a medium that carries energy while the medium does not\footnote{This can be confusing to students, because the waves we see in real life aren't always the best example of waves. Surf, for example, is actually a \emph{decaying} waveform.}
\item so the idea is that a medium might be ``sitting still" as a whole, but a wave can still move through it, carrying energy
\item waves have a:
\begin{itemize}
\item wavelength ($\lambda$) (measured in meters, centimeters, etc.)
\item frequency ($f$ or $\nu$) (measured in Hertz (Hz), $1 Hz = 1 s^{-1}$)
\item amplitude ($A$)
\item a speed ($v$) (measured in ``meters per second" ($\frac{m}{s}$))\footnote{In the case of electromagnetic waves, we use the symbol $c$ to indicate the speed of light ($3.00 \times 10^{8} \frac{m}{s}$).}
\end{itemize}
\item generally we take the speed of a wave to be: $v = \lambda  f$
\item so waves with higher frequency have lower wavelength
\item the energy of a wave is related to its amplitude: waves with higher amplitude have higher energy
\item since we're talking mainly about light here, we'll mention the speed of light:\\ $$c = 3.00 \times 10^{8} \frac{m}{s}$$
\end{itemize}



\section{Homework}
Review Problems: p. 245 \# 1--10 (not to be turned in)\\
Practice Problems: p. 246 \# 1--10 (due 2025-12-12)\\
Experiment 7.1, p. 203 (due 2025-12-05)\\

\nocite{wile-chem-2}
\bibliography{../Chemistry}{}
\bibliographystyle{apalike}

\clearpage
\subsection{Answers to Practice Problems}
Answers to Practice Problems p. 246 \# 1--10:
\begin{enumerate}
\item bleah, bleah, bleah
\end{enumerate}


\end{document}  