\documentclass[11pt, oneside]{article}   	% use "amsart" instead of "article" for AMSLaTeX format
\usepackage{geometry}                		% See geometry.pdf to learn the layout options. There are lots.
\geometry{letterpaper}                   		% ... or a4paper or a5paper or ... 

\usepackage[parfill]{parskip}    		% Activate to begin paragraphs with an empty line rather than an indent
\usepackage{graphicx}				% Use pdf, png, jpg, or eps§ with pdflatex; use eps in DVI mode
								% TeX will automatically convert eps --> pdf in pdflatex		
\usepackage{amssymb}
\usepackage{cite}

% numbered examples
\usepackage{gb4e}
\usepackage{enumitem}
\usepackage{cancel}
\usepackage{amsmath}

\usepackage{mhchem}

% Scientific Laws
\newtheorem{definition}{Definition}
\newtheorem{law}{Law}
\newtheorem{theory}{Theory}
\newtheorem{hint}{Hint}

\title{Module 4: Classifying Matter and Its Changes }
\author{Mark Peever \texttt{mpeever@gmail.com}}
\date{October 3 -- 10, 2025}

\begin{document}
\maketitle

\begin{center}

\end{center}

\section{Overview}
\begin{enumerate}
\item \textbf{Mixtures} contain multiple types of elements and/or compounds.
\item Matter is made up of moving atoms and/or molecules: the higher its temperature, the faster those atoms and/or molecules are moving.
\item \textbf{Physical Changes} are when a substance is changed, but remains the same substance.
\item \textbf{Chemical Changes} are when a substance changes from one type of thing to another type of thing.
\end{enumerate}

\section{Classifying Matter}

\begin{definition}[Pure Substance]\label{defn:pure-substance}
A substance that contains only one element and/or compound
\end{definition}

\begin{definition}[Mixture]\label{defn:mixture}
A substance that contains different elements and/or compounds
\end{definition}

\begin{itemize}
\item a mixture contains multiple elements and/or compounds, but they haven't lost their individual identities and properties
\item \emph{e.g.} air is a mixture of many different gases\footnote{See the table in the textbook, p. 105}: each gas in air is still whatever kind of gas it would be without the others\ldots air is not a compound
\item it's possible to separate parts of a mixture \emph{physically} (\emph{e.g.} by filtering)
\end{itemize}

\begin{definition}[Homogenous Mixture]\label{defn:mixture:homo}
A mixture with a composition that is always the same, regardless of which part of the sample you are observing
\end{definition}

\begin{definition}[Heterogenous Mixture]\label{defn:mixture:hetero}
A mixture with a composition that differs depending on which part of the sample you are observing
\end{definition}

\begin{itemize}
\item air \emph{seems like} a homogeneous mixture: it pretty much looks the same everywhere\ldots if you go past a dairy farm, you'll realize it's actually heterogeneous on a large enough scale
\item paint is a heterogenous mixture, which is why you have to mix it up before you can use it
\end{itemize}

\section{Classifying Changes in Matter}

\begin{definition}[Chemical Change]\label{defn:change-chemical}
a change that affects the type of atoms or molecules in a substance
\end{definition}

\begin{definition}[Physical Change]\label{defn:change-physical}
a change in which the atoms or molecules in a substance stay the same
\end{definition}

\begin{itemize}
\item the idea here is that if we physically alter a substance, that's not a chemical change
\item dissolving one substance into another is a physical change\footnote{I realize we could argue differently based on our definitions, but let's just go with it. Check out p. 107 for more.}, because we haven't actually changed the kinds of atoms and/or molecules involved
\item chewing a steak is a physical change, but digesting a steak is a chemical change
\end{itemize}

\section{Phase Changes}
\begin{itemize}
\item one interesting type of physical change is the \emph{phase change}
\item in general, there are three phases of matter: solid, liquid, and gas\footnote{Let's not get hung up on plasma right now.}.
\item we can convert matter between these phases with heat \footnote{And pressure!} (see Table \ref{table:phase-changes}, page \pageref{table:phase-changes})
\end{itemize}

\begin{table}[h]
\centering
\begin{tabular}[b]{| l | l | l | l |}
\textbf{Name} & \textbf{Phase Change} & \textbf{Beginning Phase} & \textbf{Ending Phase} \\
\hline
freeze & add heat & liquid & solid \\
melt & take heat away & solid & liquid \\
evaporate & add heat & liquid & gas \\
condense & take heat away & gas & liquid \\
sublimate\footnote{This one isn't in the book, but we see it a lot in the winter around Moscow.} & change pressure & solid & gas \\
\end{tabular}
\caption{Phase Changes for Matter}
\label{table:phase-changes}
\end{table}

\section{The Kinetic Theory of Matter}
\begin{theory}[The Kinetic Theory of Matter]\label{theory:kinetic-theory-matter}
Molecules and atoms are in constant motion, and the higher the temperature, the greater their speed.
\end{theory}

\begin{itemize}
\item we already know all matter is made of \emph{atoms} and/or \emph{molecules}
\item those atoms and molecules are in constant motion:\footnote{This doesn't break the Law of Conservation of Momentum, as long as they all sort of cancel each other out.}
\begin{enumerate}
\item in \emph{solids}, they move more-or-less in place, vibrating
\item in \emph{liquids} they move faster, and don't really stay in place
\item in \emph{gases}, they move much faster, and are free to travel really far from each other
\end{enumerate}
\item since motion requires energy,\footnote{Remember energy is the ability to do work, and work is related to force and distance moved.} we can say that the atoms and/or molecules in liquid have more energy than in a solid.
\item so we can think of \textbf{thermal energy} as being the energy associated with the movement of the atoms and/or molecules in a substance
\item similarly, we can think of \textbf{temperature} as being the average thermal energy of the substance
\end{itemize}


\section{Chemical Reactions and Chemical Equations}
\begin{definition}[Homonuclear Diatomic]\label{defn:homonuclear-diatomic}
Homonuclear diatomic molecules are made up of two atoms of the same type.
\end{definition}

\begin{itemize}
\item homonuclear diatomic molecules are the natural form\footnote{``Natural" here means, ``found in nature."} of several elements: \ce{N2}, \ce{O2}, \ce{Cl2}, \ce{F2}, \ce{Br2}, \ce{I2}, \ce{At2}, \ce{H2}
\item you'll find all these at the right-hand side of the periodic table
\item we only care about homonuclear diatomic molecules when we discuss the substances in terms of their natural form: we don't care about them when we deal with these substances in terms of their roles in molecules (\emph{e.g.} water (\ce{H2O}) contains a \emph{single} \ce{O}, not two \ce{O} atoms)
\end{itemize}

\begin{itemize}
\item we think of chemical changes in terms of \textbf{chemical reactions}, \emph{e.g.} ``carbon plus oxygen yields carbon dioxide"
\item we write chemical reactions like an equation in math: 
\begin{center}
\ce{C + O2 -> CO2}
\end{center}
\item when we write a chemical reaction as an equation, we call it a \textbf{chemical equation}
\item we write the \textbf{reactants} on the left-hand side, and the \textbf{products} on the right-hand side
\item we can add phase information too: 
\begin{center}
\ce{C (s) + O2 (g) -> CO2 (g) }
\end{center}
\end{itemize}

\subsection{Balancing Chemical Equations}
\begin{itemize}
\item let's consider the burning of methane\cite[p. 117ff]{wile-chem-2}: ``methane plus oxygen yields carbon dioxide plus water"
\item we can write that one as:\\
\begin{center}
 \ce{CH4 (g) + O2 (g) -> CO2 (g) + H2O (g)}
 \end{center}
\item written that way, we have matter both created and destroyed:
\begin{enumerate}
\item we have one \ce{C} atom on the left, and one on the right\ldots that works, but
\item we have four \ce{H} atoms on the left, and two on the right\ldots so, did we destroy atoms?
\item we have two \ce{O} atoms on the left, and three on the right\ldots so, are we creating atoms?
\end{enumerate}
\item in order to tell the truth\footnote{Chemistry is not an excuse for lying.} about not creating atoms, we need to conserve the atoms on each side
\item so our equation is:\\  
\begin{center}
\ce{CH4 (g) + O2 (g) + O2 (g) -> CO2 (g) + H2O (g) + H2O (g)}
\end{center}
\item now we have:
\begin{enumerate}
\item one \ce{C} atom on the left, and one on the right
\item four \ce{H} atoms on the left, and four on the right
\item four \ce{O} atoms on the left, and four on the right
\end{enumerate}
\item so now we're not creating or destroying atoms!
\item we can write that more simply\footnote{Just like we would in a math equation.}:\\
\begin{center}
 \ce{CH4 (g) + 2O2 (g) -> CO2 (g) + 2H2O (g)}
 \end{center}
\item this is called a \textbf{balanced chemical equation}
\item and notice what it tells us:
\begin{enumerate}
\item \ce{CH4} takes twice as many \ce{O2} molecules to burn as \ce{CH4} molecules 
\item for every \ce{CH4} molecule we burn, we'll get one \ce{CO2} molecule
\item for every \ce{CH4} molecule we burn, we'll get one \ce{H2O} molecule
\end{enumerate}
\end{itemize}

\begin{hint}[Balancing Chemical Equations]
A chemical equation is balanced when the same number of atoms of each type are on both the left-hand-side and the right-hand-side.
\end{hint}

\section{Homework}
Review Problems: p. 131 \# 1--10 (not to be turned in)\\
Practice Problems: p. 132 \# 1--10 (due 2025-10-17)\\
Experiment 4.3, p. 110 (due 2025-10-10)\\
\textbf{Take-home Exam: Quarter 1 Cumulative Exam} (due 2025-10-10) \\



\nocite{wile-chem-2}
\bibliography{../Chemistry}{}
\bibliographystyle{apalike}
\end{document}  