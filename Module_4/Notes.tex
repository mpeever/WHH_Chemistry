\documentclass[11pt, oneside]{article}   	% use "amsart" instead of "article" for AMSLaTeX format
\usepackage{geometry}                		% See geometry.pdf to learn the layout options. There are lots.
\geometry{letterpaper}                   		% ... or a4paper or a5paper or ... 

\usepackage[parfill]{parskip}    		% Activate to begin paragraphs with an empty line rather than an indent
\usepackage{graphicx}				% Use pdf, png, jpg, or eps§ with pdflatex; use eps in DVI mode
								% TeX will automatically convert eps --> pdf in pdflatex		
\usepackage{amssymb}
\usepackage{cite}

% numbered examples
\usepackage{gb4e}
\usepackage{enumitem}
\usepackage{cancel}
\usepackage{amsmath}

\usepackage{mhchem}

% Scientific Laws
\newtheorem{definition}{Definition}
\newtheorem{law}{Law}

\title{Module 4: Classifying Matter and Its Changes }
\author{Mark Peever \texttt{mpeever@gmail.com}}

\begin{document}
\maketitle

\begin{center}

\end{center}

\section{Overview}
\begin{enumerate}
\item \textbf{Mixtures} contain multiple types of elements and/or compounds
\end{enumerate}

\section{Classifying Matter}

\begin{definition}[Pure Substance]\label{defn:pure-substance}
A substance that contains only one element and/or compound
\end{definition}

\begin{definition}[Mixture]\label{defn:mixture}
A substance that contains different elements and/or compounds
\end{definition}

\begin{itemize}
\item a mixture contains multiple elements and/or compounds, but they haven't lost their individual identities and properties
\item \emph{e.g.} air is a mixture of many different gases\footnote{See the table in the textbook, p. 105}: each gas in air is still whatever kind of gas it would be without the others\ldots air is not a compound
\item it's possible to separate parts of a mixture \emph{physically} (\emph{e.g.} by filtering)
\end{itemize}

\begin{definition}[Homogenous Mixture]\label{defn:mixture:homo}
A mixture with a composition that is always the same, regardless of which part of the sample you are observing
\end{definition}

\begin{definition}[Heterogenous Mixture]\label{defn:mixture:hetero}
A mixture with a composition that differs depending on which part of the sample you are observing
\end{definition}

\begin{itemize}
\item air \emph{seems like} a homogeneous mixture: it pretty much looks the same everywhere\ldots if you go past a dairy farm, you'll realize it's actually heterogeneous on a large enough scale
\item paint is a heterogenous mixture, which is why you have to mix it up before you can use it
\end{itemize}

\section{Classifying Changes in Matter}

\begin{definition}[Chemical Change]\label{defn:change-chemical}
a change that affects the type of atoms or molecules in a substance
\end{definition}

\begin{definition}[Physical Change]\label{defn:change-physical}
a change in which the atoms or molecules in a substance stay the same
\end{definition}

\begin{itemize}
\item the idea here is that if we physically alter a substance, that's not a chemical change
\item dissolving one substance into another is a physical change\footnote{I realize we could argue differently based on our definitions, but let's just go with it. Check out p. 107 for more.}, because we haven't actually changed the kinds of atoms and/or molecules involved
\item chewing a steak is a physical change, but digesting a steak is a chemical change
\end{itemize}

\section{Phase Changes}
\begin{itemize}
\item one interesting type of physical change is the \emph{phase change}
\item in general, there are three phases of matter: solid, liquid, and gas\footnote{Let's not get hung up on plasma right now.}.
\item we can convert matter between these phases with heat \footnote{And pressure!} (see Table \label{table:phase-changes})
\end{itemize}

\begin{table}
\centering
\begin{tabular}[b]{| l | l | l | l |}
\textbf{Name} & \textbf{Phase Change} & \textbf{Beginning Phase} & \textbf{Ending Phase} \\
\hline
freeze & add heat & liquid & solid \\
melt & take heat away & solid & liquid \\
evaporate & add heat & liquid & gas \\
condense & take heat away & gas & liquid \\
sublimate\footnote{This one isn't in the book, but we see it a lot in the winter around Moscow.} & change pressure & solid & gas \\
\end{tabular}
\caption{Phase Changes for Matter}
\label{table:phase-changes}
\end{table}




\nocite{wile-chem-2}
\bibliography{../Chemistry}{}
\bibliographystyle{apalike}
\end{document}  