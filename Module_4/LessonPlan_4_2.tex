\documentclass[10pt, oneside]{article}   	% use "amsart" instead of "article" for AMSLaTeX format
\usepackage{geometry}                		% See geometry.pdf to learn the layout options. There are lots.
\geometry{letterpaper}                   		% ... or a4paper or a5paper or ... 
%\geometry{landscape}                		% Activate for rotated page geometry
%\usepackage[parfill]{parskip}    		% Activate to begin paragraphs with an empty line rather than an indent
\usepackage{graphicx}				% Use pdf, png, jpg, or eps§ with pdflatex; use eps in DVI mode
								% TeX will automatically convert eps --> pdf in pdflatex		
\usepackage{amssymb}
\usepackage{mhchem}
\usepackage{hyperref}

\title{Module 4, Part 2}
\author{Mark Peever\\ \texttt{mpeever@gmail.com}}
\date{October 10, 2025}

\begin{document}
\maketitle

\section*{Objectives}
\marginpar{0 minutes}
Refer to \href{https://drive.google.com/file/d/1-HTy9W-cQOkNabdz7FzxH-j4LVV51c--/view?usp=sharing}{Module 4 Notes.pdf}.\\


By the end of this class, the students should be able to\ldots
\begin{itemize}
\item give a succinct definition of ``Chemical Change"
\item give a succinct definition of ``Physical Change"
\item give a succinct definition of ``Kinetic Theory of Matter"
\item balance simple chemical reactions
\end{itemize}

\section*{Welcome \& Devotion}
\marginpar{5 minutes}
\begin{itemize}
\item have one student read \href{https://tinyurl.com/58tn9bhx}{Ezekiel 18:11--19}
\end{itemize}

\section*{Review of Chemical and Physical Changes}
\marginpar{10 minutes}
\begin{itemize}
\item Chemical changes affect the types of atoms or molecules in a substance
\item Physical changes alter a substance, but don't change what kind of thing it is
\item Phase changes are purely physical (solid \ce{<=>} liquid \ce{<=>} gas)
\end{itemize}

\section*{Phase Changes}
\marginpar{10 minutes}
\begin{itemize}
\item phase changes are physical: the substance isn't changing \emph{what} it is, only the form it takes
\item go over phase changes from Notes (p. 3)
\end{itemize}

\section*{Kinetic Theory of Matter}
\marginpar{10 minutes}
\begin{itemize}
\item \emph{Molecules and atoms are in constant motion, and the higher the temperature, the greater their speed.}
\item cover kinetic theory differences between Solid, Liquid, and Gas
\end{itemize}

\section*{Chemical Reactions}
\marginpar{25 minutes}
\begin{itemize}
\item cover the \emph{Homonuclear Diatomics} on Notes, p. 3
\item cover chemical reactions with examples:
\begin{itemize}
\item \ce{C + O2 -> CO2}
\item \ce{H + O2 -> H2O}
\item \ce{CH4 + O2 -> CO2}
\end{itemize}
\item chemical reactions need to be \emph{balanced}
\item we need to balance these:
\begin{itemize}
\item \ce{C + O2 -> CO2}
\item \ce{H + O2 -> H2O}
\item \ce{CH4 + O2 -> CO2}
\item \ce{Na + H2O -> NaOH + H2}
\item \ce{HCl + NaOH -> NaCl + H2O}
\item \ce{H2SO4 + NaOH -> Na2SO4 + H2O}
\item \ce{NaHCO3 + HC2H3O2 -> NaC2H3O2 + H2O + CO2}
\end{itemize}
\end{itemize}

\section*{Questions}
\marginpar{5 minutes}

\section*{Assignment}
\begin{itemize}
\item Review Problems: p. 131 \# 1--10 (not to be turned in)
\item Practice Problems: p. 132 \# 1--10 (due 2025-10-24)
\end{itemize}



\end{document}  